\chapter{SWT}
\label{sec:swt}
\index{SWT}

\href{http://www.eclipse.org/swt}{SWT} steht f�r \enquote{Standard Widget
Toolkit} und wird in \xirp~f�r die
Programmierung der graphischen Benutzeroberfl�che verwendet. Auch die Oberfl�che
von \texttt{eclipse} ist beispielsweise in \seegls{SWT} geschrieben.

Der Einstieg in die Programmierung mit \seegls{SWT} ist deutlich einfacher als bei den
von Java standardm��ig bereitgestellten \seegls{AWT} und \seegls{Swing}. Ein gutes 
\index{SWT!Tutorial}\seegls{Tutorial} zu \seegls{SWT}
ist unter \href{\swttutorialurl}{\swttutorialurl} zu finden. Eine weitere gute
Fundgrube sind die \index{SWT!Snippets}\href{\swtsnippeturl}{SWT Snippets}.

\section{dispose}
\label{sec:swt:dispose}
\seegls{SWT} benutzt zur Darstellung der Oberfl�chenelemente 
(\index{SWT!Widget|see {Widget}}Widgets) nach M�glichkeit die nativen
vom aktuellen Betriebssystem zur Verf�gung gestellten Elemente. Im Gegensatz zur
normalen Programmierung mit Java k�nnen diese Widgets nicht vom \seegls{Garbage
Collector} freigegeben werden, da dies vom Betriebssystem durchgef�hrt werden
muss. Daher hat jedes Element von \seegls{SWT} welches vom Betriebssystem bereitgestellt
wird eine Methode \index{SWT!dispose()}\codeQuote{dispose()} zum Freigeben.

Normalerweise muss diese Methode nicht programmatisch aufgerufen werden, da wenn
ein Widget \seegls{disposed} wird, automatisch auch die darauf liegenden
Kindelemente freigegeben werden und das \codeQuote{dispose()} einer
\codeQuote{Shell} beispielsweise durch das Schlie�en dieser aufgerufen wird.

Eine Ausnahme davon bilden die Klassen, die von
\codeQuote{org.eclipse.swt.graphics.Resource} erben. Werden diese von der
Applikation erstellt, m�ssen sie auch wieder von dieser \seegls{disposed}
werden, da sonst Speicherlecks entstehen w�rden.

\xirp~bietet daher f�r \index{Manager!ColorManager}Farben,
\index{Manager!FontManager}Schriften und \index{Manager!ImageManager}Bilder \codeQuote{Manager} an,
bei deren Benutzung das \seegls{disposen} entf�llt. Sie liegen alle im Paket
\codeQuote{de.unibremen.rr.xirp.ui.util.ressource} und hei�en:
\begin{itemize}
  \item \codeQuote{ColorManager}
  \item \codeQuote{FontManager}
  \item \codeQuote{ImageManager}
\end{itemize}
\section{Thread}
\label{sec:swt:thread}
\seegls{SWT} hat nur einen \index{SWT!Thread}Thread, welchem der Zugriff auf die \seegls{SWT}
Oberfl�chen Elemente erlaubt ist. Alle anderen Threads bekommen bei einem
Zugriff eine \codeQuote{SWTException} mit der Meldung \enquote{Invalid thread
access}. Um aus anderen Threads heraus Oberfl�chenelemente manipulieren zu k�nnen muss der Code, der auf ein Oberfl�chenelement
zugreift, mit einem \index{SWT!SWTUtil!asyncExec()}\codeQuote{SWTUtil.asyncExec()} 
oder \index{SWT!SWTUtil!syncExec()}\codeQuote{SWTUtil.syncExec()} gekapselt werden.
Der auszuf�hrende Code wird dabei in einem \codeQuote{Runnable} �bergeben und in
die \codeQuote{Queue} des \seegls{SWT}-Thread zur Ausf�hrung eingeh�ngt.

W�hrend \index{SWT!SWTUtil!syncExec()}\codeQuote{SWTUtil.syncExec()} wartet bis der
Code tats�chlich ausgef�hrt wurde, kehrt der Aufruf von
\index{SWT!SWTUtil!asyncExec()}\codeQuote{SWTUtil.asyncExec()} sofort zur�ck, w�hrend
der �bergebene Code erst in der Zukunft ausgef�hrt wird.

Ein Beispiel ist in \autoref{sec:sensor} auf \autopageref{sec:sensor} zu finden.

Wichtig ist den Code, der an \codeQuote{SWTUtil.asyncExec()} oder
\codeQuote{SWTUtil.syncExec()} �bergeben wird, so minimal und wenig zeitaufwendig
wie m�glich zu halten, da l�nger dauernde Codest�cke die ganze Oberfl�che lahm
legen k�nnen (siehe auch \autoref{sub:swtcalc} auf \autopageref{sub:swtcalc}).
\section{Helferlein}

\xirp~stellt f�r den leichteren Umgang mit \seegls{SWT} ein paar Hilfsklassen bereit:
\begin{itemize}
  \item \index{Manager!ColorManager}\codeQuote{ColorManager} (\longRefSec{sec:swt:dispose})
  \item \index{Manager!FontManager}\codeQuote{FontManager} (\longRefSec{sec:swt:dispose})
  \item \index{Manager!ImageManager}\codeQuote{ImageManager} (\longRefSec{sec:swt:dispose})
  \item \index{SWT!SWTUtil}\codeQuote{SWTUtil} 
  \item \index{SWT!PointUtil}\codeQuote{PointUtil}: Methode zum addieren,
  dividieren etc mit \codeQuote{org.eclipse.swt.graphics.Point}
\end{itemize}

\subsection{SWTUtil}

Klasse mit allgemeinen statischen Hilfsmethoden f�r \seegls{SWT}:
\begin{itemize}
  \item \index{SWT!SWTUtil!asyncExec()}\codeQuote{asyncExec(Runnable)}: \longRefSec{sec:swt:thread}
  \item \index{SWT!SWTUtil!syncExec()}\codeQuote{syncExec(Runnable)}: \longRefSec{sec:swt:thread}
  \item \index{SWT!SWTUtil!centerDialog()}\codeQuote{centerDialog(Shell)}: Zentriert die �bergebene \codeQuote{Shell} auf dem Bildschirm.
  \item \index{SWT!SWTUtil!setGridLayout()}\codeQuote{setGridLayout(Composite,
  int, boolean)}: Setzt ein \codeQuote{GridLayout} f�r das gegebene \codeQuote{Composite}
  \item \index{SWT!SWTUtil!resetMargins()}\codeQuote{resetMargins(GridLayout)}:
  Entfernt R�nder vom gegebenen \codeQuote{GridLayout}
  \item
  \index{SWT!SWTUtil!resetSpacings()}\codeQuote{resetSpacings(GridLayout)}: 
  Entfernt Abst�nde zwischen Zeilen und Spalten vom gegebenen 
  \codeQuote{GridLayout}
  \item \index{SWT!SWTUtil!setGridData()}\codeQuote{setGridData(...)}: Setzt
  die \codeQuote{GridData} f�r ein \codeQuote{Composite}.
  \item \index{SWT!SWTUtil!convertPoint()}\codeQuote{convertPoint(...)}:
  Koordinaten z.B. von Mausklicks werden relativ zum Parent-Widget angegeben und
  m�ssen daher teilweise umgerechnet werden. Hierzu kann diese Methode benutzt
  werden oder aber direkt \codeQuote{toDisplay()} des \codeQuote{Composite}.
  \item
  \index{SWT!SWTUtil!blockDialogFromReturning()}\codeQuote{blockDialogFromReturning(Shell)}: siehe
  unten
  \item \index{SWT!SWTUtil!showBusyWhile()}\codeQuote{showBusyWhile(Shell,
  Runnable)}: Zeigt einen wartenden Mauszeiger solange wie der �bergebene Task l�uft.
  \item \index{SWT!SWTUtil!packTable()}\codeQuote{packTable(Table)}: Ruft
  \codeQuote{pack} f�r alle Tabellenspalten der Tabelle auf.
  \item \index{SWT!SWTUtil!checkStyle()}\codeQuote{checkStyle(int, int)}: \longRefSec{sec:checkstyle}
  \item \index{SWT!SWTUtil!secureDispose()}\codeQuote{secureDispose(...)}:
  Pr�ft mit \codeQuote{swtAssert()} und \codeQuote{SWT!dispose()}\seegls{disposed} den
  Parameter wenn erfolgreich.
  \item \index{SWT!SWTUtil!swtAssert()}\codeQuote{swtAssert(...)}: Pr�ft ob
  der �bergebene Parameter nicht \codeQuote{null} und nicht \seegls{disposed}
  ist.
  \item \index{SWT!SWTUtil!rotate()}\codeQuote{rotate(...)}: Dreht einen
  Punkt um seinen Mittelpunkt in einem gegeben Winkeln (im Bogenma�).
  \item \label{itm:swtutil:openfile}\index{SWT!SWTUtil!openFile()}\codeQuote{openFile(...)}: Versucht die
  angegebene \index{Datei!�ffnen}Datei mit dem im System f�r dessen Endung registrierten Editor zu �ffnen.
\end{itemize}

\subsubsection{blockDialogFromReturning()}
\index{SWT!SWTUtil!blockDialogFromReturning()}
Damit ein erstellter Dialog erste bei einem Tastendruck nicht aber direkt nach
dem �ffnen wieder schlie�t, kann \codeQuote{blockDialogFromReturning()} genutzt
werden.

Im folgenden ein Beispiel welches in Zeile 68
\index{SWT!SWTUtil!centerDialog()}\codeQuote{centerDialog()} und in Zeile 71
\codeQuote{blockDialogFromReturning()} benutzt.

\begin{java}[numbers=left,caption=Ein einfacher Dialog mit R�ckgabe,label=code:swt:util:block]
import org.eclipse.swt.SWT;
import org.eclipse.swt.events.SelectionAdapter;
import org.eclipse.swt.events.SelectionEvent;
import org.eclipse.swt.widgets.Dialog;
import org.eclipse.swt.widgets.Display;
import org.eclipse.swt.widgets.Shell;

import de.unibremen.rr.xirp.ui.util.SWTUtil;
import de.unibremen.rr.xirp.ui.widgets.custom.XButton;
import de.unibremen.rr.xirp.ui.widgets.custom.XLabel;
import de.unibremen.rr.xirp.ui.widgets.custom.XShell;
import de.unibremen.rr.xirp.ui.widgets.custom.XStyledSpinner;
import de.unibremen.rr.xirp.ui.widgets.custom.XButton.XButtonType;

public class SimpleDialog extends Dialog {

	private static final int HEIGHT = 130;
	private static final int WIDTH = 300;
	private XShell dialogShell;
	private int returnValue = -1;

	public SimpleDialog(Shell parent) {
		super(parent);
	}

	public int open() {
		dialogShell = new XShell(getParent( ), SWT.DIALOG_TRIM |
				SWT.APPLICATION_MODAL);
		dialogShell.setText(getText( ));
		dialogShell.setSize(WIDTH, HEIGHT);
		dialogShell.setMinimumSize(WIDTH / 2 + 50, HEIGHT / 2);
		dialogShell.setText("SimpleDialog");

		SWTUtil.setGridLayout(dialogShell, 2, true);

		XLabel label = new XLabel(dialogShell, SWT.NONE);
		label.setText("Zahl eingeben");
		SWTUtil.setGridData(label, true, true, SWT.BEGINNING, SWT.CENTER, 1, 1);

		final XStyledSpinner theta = new XStyledSpinner(dialogShell, SWT.NONE);
		SWTUtil.setGridData(theta, true, true, SWT.FILL, SWT.CENTER, 1, 1);

		XButton ok = new XButton(dialogShell, XButtonType.OK);
		SWTUtil.setGridData(ok, true, false, SWT.FILL, SWT.CENTER, 1, 1);
		ok.addSelectionListener(new SelectionAdapter( ) {

			@Override
			public void widgetSelected(SelectionEvent e) {

				returnValue = theta.getSelection( );
				dialogShell.close( );
			}
		});

		XButton cancel = new XButton(dialogShell, XButtonType.CANCEL);
		SWTUtil.setGridData(cancel, true, false, SWT.FILL, SWT.CENTER, 1, 1);
		cancel.addSelectionListener(new SelectionAdapter( ) {

			@Override
			public void widgetSelected(SelectionEvent e) {
				dialogShell.close( );
			}
		});

		dialogShell.setDefaultButton(ok);
		dialogShell.pack( );
		dialogShell.layout( );
		SWTUtil.centerDialog(dialogShell);
		dialogShell.open( );

		SWTUtil.blockDialogFromReturning(dialogShell);

		return returnValue;
	}

	public static void main(String[] args) {
		final Display display = new Display( );
		final Shell shell = new Shell(display);
		shell.pack( );
		shell.open( );

		SimpleDialog dialog = new SimpleDialog(shell);
		int value = dialog.open( );

		System.out.println(value);

		while (!shell.isDisposed( )) {
			if (!display.readAndDispatch( )) {
				display.sleep( );
			}
		}
		display.dispose( );
	}
}
\end{java}
\lstset{
numbers=none,
}

\subsubsection{checkStyle()}
\label{sec:checkstyle}
\index{SWT!SWTUtil!checkStyle()}

Mit \codeQuote{checkStyle()} l�sst sich �berpr�fen ob ein bestimmter Stil z.B.
von \seegls{SWT} gesetzt ist.

Im folgenden Beispiel ist dies gezeigt:

\begin{java}[caption=Pr�fen ob Stil gesetzt ist, label=code:swt:util:checkstyle]
public static void main(String[] args) {
	int combinedStyle = SWT.APPLICATION_MODAL | SWT.DIALOG_TRIM;
	boolean applicationModal = SWTUtil.checkStyle(combinedStyle,
			SWT.APPLICATION_MODAL);
	boolean systemModal = SWTUtil.checkStyle(combinedStyle,
			SWT.SYSTEM_MODAL);
	System.out.println("Modalit�t");
	System.out.println("Applikation: " + applicationModal);
	System.out.println("System: " + systemModal);
}
\end{java}
\section{Custom}
\label{sec:swt:custom}
\index{Widget!Custom}

\xirp~stellt f�r die Nutzung Custom Widgets bereit, die anstatt der \seegls{SWT} Widgets
benutzt werden sollten. Diese Custom Widgets im Paket
\codeQuote{de.unibremen.rr.xirp.ui.widgets.custom} �bernehmen die aktuellen
\codeQuote{Farbe!\xirp}Farb- und \index{Internationalisierung}Spracheinstellungen von \xirp~automatisch.

Sie bieten je nach Anwendbarkeit die Methoden
\codeQuote{setTextForLocaleKey(String key, Object... objects)} und
\codeQuote{setToolTipTextForLocaleKey(String key, Object... objects)} sowie
Konstruktoren die zus�tzlich zu den Standardkonstruktoren einen
\index{Internationalisierung!II18nHandler}�bersetzungshandler (\autoref{cha:i18n}, \autopageref{cha:i18n}) als Parameter 
haben.

\subsection{Neue Widgets}

Im Laufe der Entwicklung von \xirp~reichten die von \seegls{SWT} bereitgestellten
\index{Widget}Widgets nicht immer aus, so dass drei neue hinzugekommen sind:
\begin{itemize}
  \item \index{Widget!Custom!XNegativeSlider}\codeQuote{XNegativeSlider} (siehe
  \autopageref{sec:XNegativeSlider})
  \item \index{Widget!Custom!XStyledSpinner}\codeQuote{XStyledSpinner} (siehe
  \autopageref{sec:XStyledSpinner})
  \item
  \index{Widget!Custom!XTimeIntervalChooser}\codeQuote{XTimeIntervalChooser} (siehe
  \autopageref{sec:XTimeIntervalChooser})
  \item \codeQuote{XImageLabel} (siehe
  \autopageref{sec:XImageLabel})
  \item \codeQuote{XPList} und \codeQuote{XPListItem} (siehe
  \autopageref{sec:XPList})
  \item \codeQuote{XPCombo} (siehe
  \autopageref{sec:XPCombo})
\end{itemize}


\subsubsection{XNegativeSlider}
\label{sec:XNegativeSlider}
\index{Widget!Custom!XNegativeSlider}
\kfig{swt_custom_xnegativeslider}{.4}{XNegativeSlider}{img:swt:custom:negativeslider}
Die \seegls{SWT} \codeQuote{Slider} unterst�tzen nur positive Werte. Bei der
Robotersteuerung ist es jedoch oft n�tig auch negative Werte einstellen zu
k�nnen. Aus diesem Grund ist der \codeQuote{XNegativeSlider} entstanden.

Im Hintergrund wird der Original \codeQuote{Slider} von \seegls{SWT} benutzt der
Minimalwert des \codeQuote{XNegativeSlider} darf jedoch im Gegensatz zu dem des
\codeQuote{Slider} auch negativ sein.

Da die Werte intern umgerechnet werden m�ssen, ist es erforderlich dass ein
Aufruf von \codeQuote{setMinimum} vor einem \codeQuote{setMaximum} Aufruf
durchgef�hrt wird und diese beiden Methoden auch nicht ein zweites mal aufgerufen
werden d�rfen.

Der \index{Tooltip}Tooltip des \codeQuote{XNegativeSlider} weist einige
Besonderheiten auf: 
Ist noch kein Tooltip Text gesetzt worden, so wird der aktuelle Wert des \codeQuote{Slider}s
als Tooltip angezeigt. Ist ein Tooltip Text mittels
\codeQuote{setToolTipTextForLocaleKey()} gesetzt worden und enth�lt die
\index{Internationalisierung!Schl�ssel} �bersetzung zu dem Schl�ssel eine Variable
\index{Internationalisierung!Variable} und wurde kein Parameter f�r die
�bersetzung mitgegeben, so wird der gegebene \index{Internationalisierung!Schl�ssel} Schl�ssel
�bersetzt und die Variable in der �bersetzung mit dem aktuellen Wert ersetzt.

\textbf{Beispiel:}\par
Ist der aktuelle Wert des \codeQuote{XNegativeSlider} \texttt{-11} und die
deutsche �bersetzung f�r den als Tooltip gesetzten Schl�ssel
\enquote{XNegativeSlider.tooltip.text} ist \enquote{Der aktuelle Wert ist
\{0\}} so erscheint als Tooltip am \codeQuote{XNegativeSlider} (siehe
\autoref{img:swt:custom:negativeslider} auf 
 \autopageref{img:swt:custom:negativeslider})
\begin{lstlisting}
Der aktuelle Wert ist -11
\end{lstlisting}


\begin{java}[caption=Beispiel XNegativeSlider,label=code:swt:custom:xnegativeslider]
import java.util.Locale;

import org.eclipse.swt.SWT;
import org.eclipse.swt.layout.FillLayout;
import org.eclipse.swt.widgets.Display;
import org.eclipse.swt.widgets.Shell;

import de.unibremen.rr.xirp.ui.widgets.custom.XNegativeSlider;
import de.unibremen.rr.xirp.util.I18n;

public class XNegativeSliderExample {

	public static void main(String[] args) {
		final Display display = new Display( );
		final Shell shell = new Shell(display);
		shell.setLayout(new FillLayout( ));
		shell.setText("XNegativeSlider");

		I18n.setLocale(Locale.GERMAN);

		XNegativeSlider s = new XNegativeSlider(shell, SWT.NONE);
		s.setMinimum(-50);
		s.setMaximum(50);
		s.setSelection(0);
		s.setToolTipTextForLocaleKey("XNegativeSlider.tooltip.text");

		shell.pack( );
		shell.open( );
		while (!shell.isDisposed( )) {
			if (!display.readAndDispatch( )) {
				display.sleep( );
			}
		}
		display.dispose( );
	}
}
\end{java}

\subsubsection{XStyledSpinner}
\label{sec:XStyledSpinner}
\index{Widget!Custom!XStyledSpinner}
\kfig{swt_custom_xstyledspinner}{.4}{XStyledSpinner}{img:swt:custom:xstyledspinner}
Der \codeQuote{XStyledSpinner} hatte zu Beginn den selben Entstehungshintergrund
wie der \codeQuote{XNegativeSlider}, jedoch kam hier zu der Anforderung auch
negative Werte darstellen zu k�nnen auch noch die auch Flie�komma-Werte
(\codeQuote{double}) Einstellen zu k�nnen.

Im Gegensatz zum \codeQuote{XNegativeSlider} basiert der
\codeQuote{XStyledSpinner} auf keinem \seegls{SWT}-Widget sondern setzt sich aus
einzelnen Textfeldern und Buttons auf einem Composite zusammen.

Um den verschiedenen Anforderungen gerecht zu werden unterst�tzt der
\codeQuote{XStyledSpinner} verschiedene Stile welche dem Konstruktor �bergeben
werden k�nnen:
\begin{itemize}
  \item \codeQuote{SpinnerStyle.NORMAL}: Ein normaler Spinner
  \item \codeQuote{SpinnerStyle.NEGATIVE}: Ein Spinner welcher auch negative
  Werte erlaubt
  \item \codeQuote{SpinnerStyle.DOUBLE}: Ein Spinner welcher die Eingabe von
  Double Werte erlaubt
  \item \codeQuote{SpinnerStyle.CHOOSABLE_INCREMENT}: Ein Spinner bei dem sich
  Ausw�hlen l�sst welche Stelle erh�ht werden soll. Sinnvoll f�r den Stil \codeQuote{SpinnerStyle.DOUBLE}
  \item \codeQuote{SpinnerStyle.ALL}: Zusammenfassung aus
  \codeQuote{SpinnerStyle.NEGATIVE}, \codeQuote{SpinnerStyle.DOUBLE} und
  \codeQuote{SpinnerStyle.CHOOSABLE_INCREMENT}
\end{itemize}

Der \codeQuote{XStyledSpinner} bietet neben einem \codeQuote{SelectionListener}
auch die M�glichkeit einen\newline \codeQuote{ValueChangedListener} zu registrieren
welcher im Gegensatz zum \codeQuote{SelectionListener} direkt einen
\codeQuote{double} als selektierten Wert enth�lt.

Werte die in die Textfelder eingegebene werden, beachten die gesetzten Minimum
und Maximum-Grenzen des Spinner, werden aber nur bei einem \codeQuote{focusLost}
�bernommen.

Mit \codeQuote{setNumberOfDecimals()} kann die Anzahl der angezeigten
Dezimalstellen f�r Spinner des Stils \codeQuote{SpinnerStyle.DOUBLE} eingestellt
werden.

Minimum und Maximum-Werte k�nnen nur negativ sein wenn der Stil
\codeQuote{SpinnerStyle.NEGATIVE} ist. Der Minimum-Wert muss immer vor dem
Maximum-Wert gesetzt werden, und die Werte k�nnen nicht wieder �berschrieben
werden. Ist der Maximum-Wert kleiner als der Minimum-Wert so werden die Werte
getauscht. 

Der mit \codeQuote{setIncrement()} gesetzte Wert wird beim Stil
\codeQuote{SpinnerStyle.DOUBLE} plus
\codeQuote{SpinnerStyle.CHOOSABLE_INCREMENT} Anteilig auf die gerade zu
ver�ndernde Stelle angerechnet.

Steht der Increment auf \texttt{1} und bei der Zahl \texttt{15,65} soll die erste
Dezimalstelle editiert werden, so ergibt eine Erh�hung den Wert \texttt{15,75}
(siehe \autoref{img:swt:custom:xstyledspinner} auf \autopageref{img:swt:custom:xstyledspinner}).

\begin{java}[caption=Beispiel XStyledSpinner,label=code:swt:custom:xstyledspiner]
import org.eclipse.swt.SWT;
import org.eclipse.swt.layout.FillLayout;
import org.eclipse.swt.widgets.Display;
import org.eclipse.swt.widgets.Shell;

import de.unibremen.rr.xirp.ui.widgets.custom.XStyledSpinner;
import de.unibremen.rr.xirp.ui.widgets.custom.XStyledSpinner.SpinnerStyle;

public class XStyledSpinnerExample {

	public static void main(String[] args) {
		final Display display = new Display( );
		final Shell shell = new Shell(display);
		shell.setLayout(new FillLayout( ));
		shell.setText("XStyledSpinner");

		XStyledSpinner s = new XStyledSpinner(shell,
				SWT.NONE,
				SpinnerStyle.CHOOSABLE_INCREMENT | SpinnerStyle.DOUBLE |
						SpinnerStyle.NEGATIVE);
		s.setMinimum(-50);
		s.setMaximum(100);
		s.setSelection(60.65);
		s.setIncrement(10);

		shell.pack( );
		shell.open( );
		while (!shell.isDisposed( )) {
			if (!display.readAndDispatch( )) {
				display.sleep( );
			}
		}
		display.dispose( );
	}
}
\end{java}

\subsubsection{XTimeIntervalChooser}
\label{sec:XTimeIntervalChooser}
\index{Widget!Custom!XTimeIntervalChooser}
\kfig{swt_custom_xtimeintervalchooser}{.75}{XTimeIntervalChooser}{img:swt:custom:xtimeintervalchooser}

Der \codeQuote{XTimeIntervalChooser} erm�glicht ist Zeitspannen auf einem
Zeitstrahl auszuw�hlen.

Mittels \codeQuote{setStart(java.util.Date)} und
\codeQuote{setStop(java.util.Date)} werden die Grenzen des Zeitstrahls auf die
angegebenen Zeiten festgelegt. Innerhalb dieser Grenzen kann dann eine beliebige
Zeitspanne ausgew�hlt werden.

Durch \codeQuote{setRange(java.util.Date)} kann zus�tzlich eine feste Zeitspanne
von z.B. 10 min festgelegt werden, bei der sich nur noch Start oder Endpunkt
festlegen l�sst.

�ber dem Zeitstrahl werden die Start- und Endzeit des Gesamten Zeitstrahls
angezeigt, darunter die Zeiten der Gew�hlten Zeitspanne.

Die Farben des \codeQuote{XTimeIntervalChooser} sind einstellbar:
\begin{itemize}
  \item \codeQuote{setBackground()}: Die Hintergrundfarbe
  \item \codeQuote{setChosenTimeTextColor()}: Die Farbe f�r die Zeiten der
  gew�hlten Zeitspanne
  \item \codeQuote{setLineColor()}: Die Farbe des Zeitstrahls
  \item \codeQuote{setMarkerColor()}: Die Farbe der Zeitspannenmarkierung
\end{itemize}

Weiterhin l�sst sich das Format der angezeigten Zeiten mit
\codeQuote{setFormat()} festlegen. Dieses Format entspricht denen in
\codeQuote{java.text.SimpleDateFormat} beschriebenen Formaten und ist Initial
\enquote{dd.MM.yyyy HH:mm:ss,SSS} zeigt also Tage, Monate, Jahr, Stunden, Minuten,
Sekunden und Millisekunden an.

Das folgende Beispiel zeigt die Erstellung eines
\codeQuote{XTimeIntervalChooser} mit einem Zeitstrahl von 30 Minuten ab jetzt
und einer Zeitspannen-Auswahl von 10 Minuten mit einem Zeitformat welches
Stunden, Minuten und Sekunden darstellt (siehe
\autoref{img:swt:custom:xtimeintervalchooser} auf \autopageref{img:swt:custom:xtimeintervalchooser}).

\begin{java}[caption=Beispiel XTimeIntervalChooser,label=code:swt:custom:xtimeintervalchooser]
import java.util.Date;
import java.util.concurrent.TimeUnit;

import org.eclipse.swt.SWT;
import org.eclipse.swt.layout.FillLayout;
import org.eclipse.swt.widgets.Display;
import org.eclipse.swt.widgets.Shell;

import de.unibremen.rr.xirp.ui.widgets.custom.XTimeIntervalChooser;

public class XTimeIntervalChooserExample {

	public static void main(String[] args) {
		final Display display = new Display( );
		final Shell shell = new Shell(display);
		shell.setLayout(new FillLayout( ));
		shell.setText("XTimeIntervalChooser");

		XTimeIntervalChooser s = new XTimeIntervalChooser(shell, SWT.NONE);
		// Startzeit = Jetzt
		s.setStart(new Date( ));
		// Stopzeit in einer halben Stunde
		Date stop = new Date(System.currentTimeMillis( ) +
				TimeUnit.MINUTES.toMillis(30));
		s.setStop(stop);

		// Farbe der ausgew�hlten zeiten soll rot sein
		s.setChosenTimeTextColor(display.getSystemColor(SWT.COLOR_RED));
		// Farbe des Zeitstrahls soll blau sein
		s.setLineColor(display.getSystemColor(SWT.COLOR_BLUE));
		// Farbe der Zeitspannen Markierungen soll Magenta sein
		s.setMarkerColor(display.getSystemColor(SWT.COLOR_DARK_MAGENTA));

		// Feste Zeitspanne von 10 Minuten
		s.setRange(new Date(TimeUnit.MINUTES.toMillis(10)));
		// Stunde, Minuten und Sekunden anzeigen
		s.setFormat("HH:mm:ss");

		shell.pack( );
		shell.setSize(500, shell.getSize( ).y);
		shell.open( );
		while (!shell.isDisposed( )) {
			if (!display.readAndDispatch( )) {
				display.sleep( );
			}
		}
		display.dispose( );
	}
}
\end{java}

\subsubsection{XImageLabel}
\label{sec:XImageLabel}
\index{Widget!Custom!XImageLabel}

Das \codeQuote{XImageLabel} bietet die M�glichkeit einen roten oder gr�nen Punkt
zum Beispiel f�r die Komplettierung einer Aufgabe anzuzeigen. Als
\index{Tooltip}Tooltip erscheint entsprechend des Punktes \enquote{Dieser Teil
ist fertig} (gr�n) oder
\enquote{Dieser Teil ist nicht fertig} (rot). Die Angabe eines eigenen
\index{Tooltip}Tooltip oder Textes ist nicht m�glich.

Der Status kann mittels der Methode \codeQuote{setDone(boolean)} gesetzt und mit
\codeQuote{isDone()} abgerufen werden.

\subsubsection{XPList und XPListItem}
\label{sec:XPList}
\index{Widget!Custom!XPList}
\index{Widget!Custom!XPListItem}
\kfig{swt_custom_xplist}{.75}{XPList mit ListType.TITLE$\_$AND$\_$DESC in der 
Kontaktverwaltung}{img:swt:custom:xplist}
Die \codeQuote{XPList} hat als Grundlage nicht ein \seegls{SWT}-Widget sondern
\codeQuote{PList} aus \index{SWTPlus!PList}\seegls{SWTPlus}.

Diese Liste unterst�tzt drei verschiedene Typen:
\begin{itemize}
  \item \codeQuote{ListType.SIMPLE}: Einfache Liste von Items mit Bild und Text
  \item \codeQuote{ListType.LIST_BAR}: Liste von Items mit gr��eren Bildern
  \item \codeQuote{ListType.TITLE_AND_DESC}: Liste von Items mit Bild, Titel und
  Beschreibung (siehe \autoref{img:swt:custom:xplist})
\end{itemize}

Ist der Type der Liste auf der ein \seegls{Item} liegt
\codeQuote{ListType.TITLE_AND_DESC} so unterst�tzen die \codeQuote{XPListItem}s
zus�tzlich die Methoden \codeQuote{setDescription(String)},
\codeQuote{setDescriptionForLocaleKey(String, Object...)} und
\codeQuote{getDescription()}.

Die \codeQuote{setText()}-Methoden setzen in diesem Fall den Titel des \seegls{Item}.

\subsubsection{XPCombo}
\label{sec:XPCombo}
\index{Widget!Custom!XPCombo}
\kfig{swt_custom_xpcombo}{.3}{XPCombo mit ComboType.COLOR}{img:swt:custom:xpcombo}
Die \codeQuote{XPCombo} hat als Grundlage nicht ein \seegls{SWT}-Widget sondern
\codeQuote{PCombo} aus \index{SWTPlus!PCombo}\seegls{SWTPlus}.

Diese Liste unterst�tzt drei verschiedene Typen:
\begin{itemize}
  \item \codeQuote{ComboType.COLOR}: Eine \index{Widget!Combobox}Combobox zur
  Farbauswahl (siehe \autoref{img:swt:custom:xpcombo})
  \item \codeQuote{ComboType.LIST}: Eine \index{Widget!Combobox}Combobox mit einer
  Liste zur Auswahl
  \item \codeQuote{ComboType.TREE}: Eine \index{Widget!Combobox}Combobox mit einem
  Baum zur Auswahl
  \item \codeQuote{ComboType.TABLE}: Eine \index{Widget!Combobox}Combobox mit einer
  Tabelle zur Auswahl
\end{itemize}

Die Methode \codeQuote{setI18nPrefix(String)} wird Intern f�r die
\index{Internationalisierung}Internationalisierung der Farbnamen f�r den Stil
\codeQuote{ComboType.COLOR} und sollte nicht extern verwendet werden.

\subsection{Besondere Funktionen}

Einige \index{Widget!Custom!Besondere Funktionen} Custom-Widgets haben neben
Internationalisierung und Farb�bernahme neben den Original-Funktionen des
SWT-Widget noch zus�tzliche Funktionen die die Benutzung vereinfachen sollen.
Zur Zeit sind dies:
\begin{itemize}
  \item \codeQuote{XCombo}
  \item \codeQuote{XList}
  \item \codeQuote{XPCombo}
  \item \codeQuote{XTabFolder}
  \item \codeQuote{XTable}
  \item \codeQuote{XTableColumn}
  \item \codeQuote{XTreeItem}
  \item \codeQuote{XButton}
\end{itemize} 

\subsubsection{XCombo}

Die Methode \codeQuote{select(String)} der \codeQuote{XCombo} selektiert den
gegebenen \codeQuote{String} der Teil der \seegls{Items} der \index{Widget!Combobox}Combobox sein muss, jedoch
kein \index{Internationalisierung!Schl�ssel}Schl�ssel sein darf.

\subsubsection{XList}
Die Methode \codeQuote{setSelectionWithEvent} der \codeQuote{XList} selektiert
den gegebenen Index und wirft ein entsprechendes \codeQuote{SelectionEvent}.

\subsubsection{XTabFolder}
Der \codeQuote{XTabFolder} hat keine besonderen Methoden, sorgt aber daf�r dass
\seegls{Items} die auf diesem Folder liegen mit eine Klick des Mittleren Mausbuttons auf
das \seegls{Item} geschlossen werden k�nnen.

\subsubsection{XTable und XTableColumn}
\codeQuote{XTable} und \codeQuote{XTableColumn} unterst�tzen die Sortierung von
Spalten wenn sie zusammen verwendet werden.

\codeQuote{XTable} bietet folgende Methode:
\begin{itemize}
  \item \codeQuote{sortColumn(int, int)}: Sortiert die angegebene Spalte wenn
  diese vorhanden und vom Typ \codeQuote{XTableColumn} ist in der angegebenen
  Sortierreihenfolge \codeQuote{SWT.UP} und \codeQuote{SWT.DOWN}.
  \item \codeQuote{resortColumn(int)}: Sortiert die angegebene Spalte wenn
  diese vorhanden und vom Typ \codeQuote{XTableColumn} ist in der aktuell
  gesetzten Sortierreihenfolge.
  \item \codeQuote{setSelectionAndForceFocus(int)}: Selektiert die angegebene
  Zeile, wirft ein passendes \codeQuote{SelectionEvent} und gibt der Tabelle den
  Focus.
  \item \codeQuote{selectLastAndForceFocus()}: Wie
  \codeQuote{setSelectionAndForceFocus(int)} selektiert aber die letzte Zeile
  der Tabelle.
\end{itemize}

\codeQuote{XTableColumn} bietet folgende Methode:
\begin{itemize}
  \item \codeQuote{setSortable(Comparator<String>)}: Sortiert die Spalte mit dem
  gegebenen \codeQuote{Comparator} f�r \codeQuote{String}s
  \item \codeQuote{setSortable(SortType)}: Stellt die Sortierung auf den
  gegebenen Typ:
  \begin{itemize}
    \item \codeQuote{SortType.ALPHA}: Alphanumerische Sortierung
    \item \codeQuote{SortType.NUMERIC_INT}: Sortierung f�r Ganzzahlen
    \item \codeQuote{SortType.NUMERIC_DOUBLE}: Sortierung f�r Flie�kommazahlen
  \end{itemize}
  \item \codeQuote{sort(int)}: Sortiert diese Spalte anhand des gegebenen
  Sortierungstyps \codeQuote{SWT.UP} (aufsteigend) oder \codeQuote{SWT.DOWN} (absteigend).
  \item \codeQuote{disableSortable()}: Deaktiviert die Sortierung f�r diese Spalte.
\end{itemize}

Die Sortierung von \codeQuote{XTableColumn} ist Standardm��ig an und steht auf
alphanumerischer Sortierung.

\subsubsection{XTreeItem}
Da Standardm��ig nur die Methode \codeQuote{getChecked()} existiert kommt beim
\codeQuote{XTreeItem} zur besseren Auffindbarkeit noch die Methode
\codeQuote{isChecked()} hinzu, welche \codeQuote{getChecked()} zur�ck gibt.

\subsubsection{XButton}
Der \codeQuote{XButton} ist ein Normaler Button mit dem Stil \codeQuote{SWT.PUSH}
dem Konstruktor kann stattdessen aber noch ein Typ mitgegeben werden welcher
einen \index{Internationalisierung}internationalisierten Standardtext f�r den
Button setzt (hier die deutschen �bersetzungen):
\begin{itemize}
  \item \codeQuote{XButtonType.NULL}: Text und Tooltip kann manuell gesetzt
  werden. Dies entspricht der �bergabe keines Typ.
  \item \codeQuote{XButtonType.APPLY}: Text \enquote{�bernehmen}, Tooltip \enquote{�nderungen �bernehmen}
  \item \codeQuote{XButtonType.SAVE}: Text \enquote{Speichern}, Tooltip \enquote{Daten �bernehmen und speichern}
  \item \codeQuote{XButtonType.RESET}: Text \enquote{Zur�cksetzen}, Tooltip \enquote{Feldinhalte zur�cksetzen}
  \item \codeQuote{XButtonType.DEFAULT}: Text \enquote{Standard}, Tooltip \enquote{Standard wiederherstellen}
  \item \codeQuote{XButtonType.CLOSE}: Text \enquote{Schlie�en}, Tooltip \enquote{Dialog schlie�en}
  \item \codeQuote{XButtonType.OK}: Text \enquote{OK}, Tooltip \enquote{Daten best�tigen}
  \item \codeQuote{XButtonType.CANCEL}: Text \enquote{Abbrechen}, Tooltip \enquote{Operation abbrechen}
  \item \codeQuote{XButtonType.YES}: Text \enquote{Ja}, Tooltip \enquote{Ja ich will}
  \item \codeQuote{XButtonType.NO}: Text \enquote{Nein}, Tooltip \enquote{Nein ich will nicht}
  \item \codeQuote{XButtonType.LOOKUP}: Text \enquote{\ldots}, Tooltip \enquote{Daten ausw�hlen}
\end{itemize}

\subsection{Eigene Widgets}
\index{Widget!eigene}
Werden zus�tzlich zu den von \xirp~bereitgestellten noch weitere Widgets
ben�tigt, welche ebenfalls \index{Internationalisierung}Internationalisierung
und \index{Farbe!\xirp} Farbgebung von \xirp~
entsprechen, so k�nnen diese selber erstellt werden.

Dies soll am Beispiel von \codeQuote{XText} gezeigt werden.

Als erste wird daf�r die Klasse selbst ben�tigt. Diese erweitert
\codeQuote{Text}: 
\begin{java}[caption=Eigenes Widget: XText (1),label=code:swt:custom:xtext1]
import org.eclipse.swt.widgets.Composite;
import org.eclipse.swt.widgets.Text;

import de.unibremen.rr.xirp.util.I18n;
import de.unibremen.rr.xirp.util.II18nHandler;

public class XText extends Text {

	private II18nHandler handler;

	public XText(Composite parent, int style) {
		super(parent, style);
		handler = I18n.getGenericI18n( );
	}

	public XText(Composite parent, int style, II18nHandler handler) {
		super(parent, style);
		this.handler = handler;
	}

}
\end{java}

F�r die �bersetzungen wird ein \index{Internationalisierung!II18nHandler}
\codeQuote{II18nHandler} ben�tigt. Der Standardkonstruktor nutzt die
Standard�bersetzungen von \xirp~w�hrend der zweite Konstruktor einen \seegls{Handler}
�bergeben bekommt. Diese Variante wird von \seegls{Plugins} genutzt.

Um die �bersetzungen zu vereinfachen kann der Handler direkt an eine Instanz von
\newline\index{Internationalisierung!XWidgetLocaleHandler} 
\codeQuote{XWidgetLocaleHandler} �bergeben werden.
\begin{java}[caption=Eigenes Widget: XText (2),label=code:swt:custom:xtext2]
public class XText extends Text {

	private final XWidgetLocaleHandler localeHandler;

	public XText(Composite parent, int style) {
		super(parent, style);
		localeHandler = new XWidgetLocaleHandler(this, I18n.getGenericI18n( ));
		initListeners( );
	}

	public XText(Composite parent, int style, II18nHandler handler) {
		super(parent, style);
		localeHandler = new XWidgetLocaleHandler(this, handler);
		initListeners( );
	}

}
\end{java}

Zu sehen ist hier schon die Methode \codeQuote{initListener()} welche f�r die
Farb�nderungen ben�tigt wird.

\subsubsection{Farbe}

Diese sollen nun zuerst implementiert werden.
\index{Manager!ColorManager}
\begin{java}[caption=Eigenes Widget: XText (3),label=code:swt:custom:xtext3]
private AppearanceChangedListener appearanceListener;

private void initListeners() {
	// Hintegrundfarbe initial setzen
	setBackground(ColorManager.getColor(Variables.BACKGROUND_COLOR));

	appearanceListener = new AppearanceChangedListener( ) {

		// Bei Farb�nderung neue Farbe setzen
		public void appearanceChanged(AppearanceChangedEvent event) {
			setBackground(ColorManager.getColor(Variables.BACKGROUND_COLOR));
		}
	};

	// Listener registrieren
	ApplicationManager.addAppearanceChangedListener(appearanceListener);

	addDisposeListener(new DisposeListener( ) {

		public void widgetDisposed(DisposeEvent e) {
			// Listener abmelden
			ApplicationManager.removeAppearanceChangedListener(appearanceListener);
		}
	});
}
\end{java}

Um �ber \codeQuote{Farbe!\xirp!addAppearanceChangedListener()}Farb�nderungen informiert zu werden, meldet sich diese Klasse bei
\xirp~auf die Farb�nderungen an (und beim \codeQuote{dispose} wieder ab).

�ndert sich die Farbe, so wird diese neu gesetzt.

\subsubsection{Internationalisierung}

Damit das neue Widget nun auch immer die aktuelle Sprache von
\xirp~widerspiegelt muss nun der schon angelegte \codeQuote{localeHandler} noch
benutzt werden.

Dazu werden zwei neue Methoden angelegt:
\begin{java}[caption=Eigenes Widget: XText (4),label=code:swt:custom:xtext4]
public void setTextForLocaleKey(String key, Object... objects) {
	localeHandler.setText(key, objects);
}

public void setToolTipTextForLocaleKey(String key, Object... objects) {
	localeHandler.setToolTipText(key, objects);
}
\end{java}

Diese Methoden delegieren ihre Aufrufe an den \codeQuote{localeHandler} welcher
dann daf�r sorgt dass Text und Tooltip immer in der gerade gesetzten Sprache
erscheinen.
\section{.. oder doch Swing?}
Bibliotheken wie zum Beispiel JMF (\enquote{Java Media Framework}) bieten zur
Darstellung eines Videostream nur die M�glichkeit an diesen auf einem \seegls{Swing}
Composite darzustellen. Dieses ist kein Hinderungsgrund \xirp~bzw. \seegls{SWT} zu
nutzen, da \seegls{SWT} �ber die Klasse \codeQuote{SWT_AWT} die M�glichkeit bietet \seegls{Swing}
oder \seegls{AWT} Komponenten auf \seegls{SWT} Komponenten abzulegen.

Ein Beispiel f�r JMF (\codeQuote{this} ist ein SWT \codeQuote{Composite}):
\begin{java}
private void initCameraView() {
		if (player != null) {
			player.start( );
			Component comp;
			if ((comp = player.getVisualComponent( )) != null) {
				this.setSize(700, 480);
				Frame f = SWT_AWT.new_Frame(this);
				if (f != null) {
					GridBagConstraints gridBagMain = new GridBagConstraints( );
					gridBagMain.gridx = 0;
					gridBagMain.fill = java.awt.GridBagConstraints.BOTH;
					gridBagMain.weightx = 1.0D;
					gridBagMain.weighty = 1.0D;
					gridBagMain.gridy = 1;

					GridBagConstraints gridBagNavi = new GridBagConstraints( );
					gridBagNavi.gridx = 0;
					gridBagNavi.fill = java.awt.GridBagConstraints.HORIZONTAL;
					gridBagNavi.weightx = 1.0D;
					gridBagNavi.weighty = 0.0D;
					gridBagNavi.gridy = 0;

					f.setLayout(new GridBagLayout( ));
					f.setSize(300, 200);
					f.add(getJPanelNavi( ), gridBagNavi);
					f.add(comp, gridBagMain);

					f.setVisible(true);
				}
				else {
					logClass.debug(camera.getRobotName( ),
							"Could not get java.awt.Frame to Display Camera." + Constants.LINE_SEPARATOR); //$NON-NLS-1$
			}
		}
	}
}
\end{java}
