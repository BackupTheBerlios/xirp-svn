\chapter{TesterBot}
\index{TesterBot}
Um entwickelte \seegls{Sensor}plugins ohne einen realen Roboter testen zu k�nnen, stellt
\xirp~den \robotQuote{TesterBot} zur Verf�gung.

Zum \robotQuote{TesterBot} l�sst sich mittels des \seegls{IP}-Kommunikation und des
\robotQuote{TesterBot}-Protokoll \seegls{Plugins} eine Verbindung herstellen. 

Der \robotQuote{TesterBot} liefert dann folgende Daten (als \codeQuote{Number}
wenn nicht anders beschrieben):
\index{Datenpool!Schl�ssel}
\begin{center}
\begin{longtable}{l|l|l|l}
\bf{Bezeichnung} & \bf{Datenpool-Schl�ssel} & \bf{Wertebereich} & \bf{Einheit}\\
\hline\endhead
Batterie 	& battery1 				& [0;5000]	& Ampere\\
 			& battery2 				& [0;5000]	& Ampere\\
\hline
Infrarot 	& ir\_left\_front 		& [0;80] 	& Zentimeter\\
 			& ir\_right\_front 		& [0;80] 	& Zentimeter\\
 			& ir\_left\_rear 		& [0;80] 	& Zentimeter\\
 			& ir\_right\_rear 		& [0;80] 	& Zentimeter\\
\hline
Gyroscope 	& gyro\_roll\_unique 	& [0;80] 	& \\
		 	& gyro\_nick\_unique 	& [0;80] 	& \\
\hline
Ultraschall & sonic\_front 			& [0;7500] 	& Zentimeter\\
			& sonic\_left 			& [0;7500] 	& Zentimeter\\
			& sonic\_right 			& [0;7500] 	& Zentimeter\\
			& sonic\_rear 			& [0;7500] 	& Zentimeter\\
\hline
Temperatur	& temperature\_unique	& [4;100] 	& Grad Celsius\\
\hline
Kompass		& compass\_unique		& [0;360] 	& Grad\\
\hline
CO$_2$		& co2\_unique			& [0;20] 	& Partikel pro Million\\
\hline
Laserscanner& laser\_unique			& [20;4096] & Zentimeter\\
			& Float Array mit 681 Werten & 		&\\
\hline
Thermopile	& thermopile\_unique	& [4;100] 	& Grad Celsius\\
			& double[8][32]			&			&\label{tab:testerbot}\\
\hline
\caption{Vom TesterBot gelieferte Werte}\\
\end{longtable}
\end{center}

F�r diese Werte sind Standardplugins in \xirp~bereits enthalten.

Das \seegls{Profil} f�r den \robotQuote{TesterBot} wird ebenfalls mitgeliefert.

Wie der \robotQuote{TesterBot} aufgerufen wird, l�sst sich im
\index{Benutzerhandbuch}Benutzerhandbuch
nachlesen. Wie Daten vom \index{Datenpool}Datenpool empfangen werden k�nnen
l�sst sich in \autoref{sec:datapool} ab \autopageref{sec:datapool} nachlesen.
