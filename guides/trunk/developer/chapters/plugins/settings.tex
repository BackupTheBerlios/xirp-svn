\section{Einstellungen}
\label{sec:plugin:settings}
\index{Plugin!Einstellungen!erstellen|(}
\index{Einstellungen}
\seegls{Plugins} k�nnen eigene Einstellungen definieren.

Das Beispiel \seegls{Plugin} kann einfach um solche Einstellungen erweitert werden, indem
die Methode \index{Plugin!IPlugable!getSettings()}\codeQuote{getSettings()} �berschrieben wird.

Die \seegls{API} f�r Einstellungen findet sich im Paket
\codeQuote{de.unibremen.rr.xirp.settings}.

F�r dieses \seegls{Plugin} soll es einstellbar sein, was f�r ein Text ausgegeben wird.


Zun�chst muss dazu noch ein Jar hinzugef�gt werden: das commons configuration
jar, welches sich im Ordner \fileQuote{lib} der Installation befindet.
\index{Manager!PluginManager}
\begin{java}[caption=Einstellungen vom Plugin,
				 label=lst:smplplug:settings_plugin_getSettings]
	@Override
	public Settings getSettings() {
		// Nur neues Objekt erstellen wenn noch nicht vorhanden
		if (settings == null) {
			settings = new Settings(PluginManager.getProperties( ),
					handler,
					"MyPlugin.settings.main",
					getRealKey( ));
			// Neue Seite
			SettingsPage settingsPage = settings.addPage("page",
					null,
					"description");
			// Neue Option mit freiem Text
			optionString = settingsPage.addOption("sysout",
					Option.OptionType.TEXTFIELD);
			// Default ist Hello World!
			optionString.addStringValue("sysout", "Hello World!");

			settings.addSettingsChangedListener(new SettingsChangedListener( ) {

				@Override
				public void settingsChanged(SettingsChangedEvent event) {
					// Auslesen der gemachten Einstellungen
					readSettings( );
				}

			});
		}
		return settings;
	}
\end{java}

Damit sind wieder ein paar weitere Schl�ssel f�r die 
\index{Internationalisierung!Plugin}\index{Internationalisierung!Variable}�bersetzungen 
hinzugekommen:\\
\newpage\textbf{de}
\begin{properties}[caption=Deutsche �bersetzungen,
				 label=lst:smplplug:settings_props_de]
MyPlugin.plugin.name=Mein Plugin
MyPlugin.plugin.description=Ein Testplugin

MyPlugin.toolbar.helloworld=Hallo Welt ausgeben
MyPlugin.print.helloworld={0} sagt: "{1}"

MyPlugin.settings.main=Mein Plugin
MyPlugin.settings.main.page=Mein Plugin
MyPlugin.settings.main.page.sysout=Ausgabe
MyPlugin.settings.main.page.description=Das Plugin gibt den eingestellten String aus
\end{properties}

\textbf{en}
\begin{properties}[caption=Englische �bersetzungen,
				 label=lst:smplplug:settings_props_en]
MyPlugin.plugin.name=My Plugin
MyPlugin.plugin.description=A test plugin

MyPlugin.toolbar.helloworld=Print out Hello World
MyPlugin.print.helloworld={0} says: "{1}"

MyPlugin.settings.main=My Plugin
MyPlugin.settings.main.page=My Plugin
MyPlugin.settings.main.page.sysout=Printout
MyPlugin.settings.main.page.description=The plugin prints the given string
\end{properties}

Die Methode \codeQuote{readSettings()} wurde zum Auslesen der \index{Einstellungen!Optionen!auslesen}Einstellungen in
lokale Variablen hinzugef�gt, welche immer aufgerufen wird, wenn ge�nderte
Einstellungen gespeichert werden.

\begin{java}[caption=Einstellungen vom Plugin auslesen,
				 label=lst:smplplug:settings_plugin_readSettings]
	private void readSettings() {
		getSettings( );
		for (IValue value : optionString.getSelectedValues( )) {
			toPrint = value.getDisplayValue( );
		}
	}
\end{java}

Hier wird der Wert der in den Einstellungen erstellten und ge�nderten \index{Einstellungen!Optionen!auslesen}Option
ausgelesen und in das Feld \codeQuote{toPrint} �bertragen.

Damit in \codeQuote{toPrint} auch zu Beginn der richtige Wert steht, wird
\codeQuote{readSettings()} auch in \index{Plugin!IPlugable!runInternal()}\codeQuote{runInternal()} aufgerufen.
\begin{java}[caption=Einstellungen vom Plugin auslesen,
				 label=lst:smplplug:settings_plugin_runInternal]
	@Override
	protected void runInternal() {
		readSettings( );
	}
\end{java}
\index{Toolbar}
Nun muss in \index{Plugin!IPlugable!getToolbar()}\codeQuote{getToolbar()} nur noch das Feld bei der Ausgabe mit
�bergeben werden. Oben ist zu sehen, dass f�r die �bersetzungen des Ausgabeschl�ssels
nun ein zweiter Parameter ben�tigt wird. Dies wird das Feld \codeQuote{toPrint}
sein. 
\index{Manager!ImageManager}
\begin{java}[caption=Log-Ausgabe eines Plugin konfigurieren,
				 label=lst:smplplug:settings_plugin_getToolBar]
	@Override
	public XToolBar getToolBar(CoolItem parent) {
		XToolBar tools = new XToolBar(parent.getParent( ), SWT.FLAT);
		// �bersetzungshandler dieses Plugins mit �bergeben
		XToolItem helloWorld = new XToolItem(tools,
				SWT.FLAT | SWT.PUSH,
				handler);
		// Key f�r den Tooltip setzen
		helloWorld.setToolTipTextForLocaleKey("MyPlugin.toolbar.helloworld"); //$NON-NLS-1$
		// Bild vom ImageManager holen
		helloWorld.setImage(ImageManager.getPluginImage(this, "helloworld.png")); //$NON-NLS-1$
		helloWorld.addSelectionListener(new SelectionAdapter( ) {

			/*
			 * (non-Javadoc)
			 * 
			 * @see org.eclipse.swt.events.SelectionAdapter#widgetSelected(org.eclipse.swt.events.SelectionEvent)
			 */
			@Override
			public void widgetSelected(SelectionEvent e) {
				// �bersetzte Logausgabe mit einer Variablen
				LOGGER.info(robotName,
						getI18nString("MyPlugin.print.helloworld",
								MyPlugin.this.getRobotName( ),
								toPrint)
								+ Constants.LINE_SEPARATOR);
			}
		});

		return tools;
	}
\end{java}

Erstellt man das \seegls{Plugin} nun neu und f�hrt \xirp~damit aus, so kann man nun im
Men� unter \menuQuote{Plugins} und \index{Men�!Plugins!Einstellungen}\menuQuote{Einstellungen} bei dem Plugin den
Ausgabestring einstellen (\refFig{settings}).

\kfig{plugin_settings}{1}{Einstellungen eines Plugins}{settings}

Die Ausgabe k�nnte nun so aussehen:
\begin{lstlisting}
TesterBot: TesterBot sagt: "Ich bin ein eingestellter String"
\end{lstlisting}

Weitere Information finden sich in \autoref{cha:settings} ab Seite \autopageref{cha:settings}.

\index{Plugin!Einstellungen!erstellen|)}