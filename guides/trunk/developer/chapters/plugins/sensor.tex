\section{Sensordaten}
\label{sec:sensor}
\index{Sensordaten}

Das \index{Kommunikation}Kommunikationsplugin des Roboters sorgt daf�r, dass \seegls{Sensor}daten die vom
Roboter empfangen werden im \index{Datenpool}Datenpool (siehe auch 
\autoref{sec:datapool} auf \autopageref{sec:datapool}) unter einem
bestimmten \index{Datenpool!Schl�ssel}Schl�ssel (Key) abgelegt werden.

\begin{xml}[caption=Roboter Profil Sensorgruppe,label=lst:smplplug:sensordata:sensorgroup]
<sensorgroup datapoolKey="compass" longName="Compass" visible="true">
    <sensor subKey="unique" unit="DEGREE" id="0">
        <sensorspecs>
            <position y="250" x="250" side="INSIDE" attached="TORSO"/>
            <minimum>0.0</minimum>
            <maximum>360.0</maximum>
        </sensorspecs>
    </sensor>
</sensorgroup>
\end{xml}

Dies ist ein Beispiel aus dem \index{Profil!Roboter}Roboterprofil des \robotQuote{TesterBot} und zeigt
die Definition eines Kompasssensors. Der \index{Datenpool!Schl�ssel}Datenpool-Schl�ssel unter welchem die
Daten im \index{Datenpool}Datenpool abgelegt werden setzt sich dabei aus dem Attribut
\codeQuote{compass} des Elements \codeQuote{sensorgroup} und \codeQuote{subKey}
des Elements \codeQuote{sensor} zusammen. Um den vollst�ndigen Schl�ssel zu
erhalten kann die Klasse \index{Datenpool!DatapoolUtil}\codeQuote{DatapoolUtil} 
genutzt werden.

Dem \index{Plugin!Sensor}\index{Profil!Roboter!Plugin}\seegls{Plugin} kann ein so definierter \index{Profil!Roboter!Sensor}\seegls{Sensor} wie folgt
im Roboter \seegls{Profil} zugeordnet werden:
\begin{xml}[caption=Roboter Profil Sensor-Plugin-Zuordnung,label=lst:smplplug:sensordata:plugin]
<plugin name="Compass">
    <class>de.unibremen.rr.plugins.sensors.compass.CompassDisplay</class>
    <sensorname>Compass</sensorname>
    <usemultimedia>false</usemultimedia>
</plugin>
\end{xml}

F�r die Zuordnung wir das Attribut \codeQuote{longName} des Elements
\codeQuote{sensorgroup} genutzt.

Das Auslesen der Daten eines \seegls{Sensor}s aus dem \index{Profil!Roboter}Roboter \seegls{Profil} wird \seegls{Sensor}plugins
vereinfacht indem sie statt
\index{Plugin!IPlugable!AbstractPlugin}\codeQuote{AbstractPlugin}
\index{Plugin!IPlugable!AbstractSensorPlugin}\codeQuote{AbstractSensorPlugin}
erweitern. Dies stellt die Methoden
\begin{itemize}
  \item \codeQuote{getSensorgroups()}
  \item \codeQuote{getSensorgroup()}
  \item \codeQuote{getSensor()}
  \item \codeQuote{readSensorDefaults()} (f�r \index{Plugin!Plugindaten!DefaultSensorData}\codeQuote{DefaultSensorData})
\end{itemize} 
bereit.

Das Auslesen aus dem Roboter \seegls{Profil} sollte in 
\index{Plugin!IPlugable!runInternal()}\codeQuote{runInternal()}
erfolgen, damit dies f�r jedes \seegls{Plugin} nur genau einmal ausgef�hrt wird.

F�r das Kompass-\seegls{Plugin} sieht die Methode so aus:
\begin{java}[caption=Auslesen von Sensorspezifikation aus Roboter Profil (kurz),label=lst:smplplug:sensordata:compassplugin:short]
@Override
protected void runInternal() {
	getSensor( );
	initListener( );
}

private void initListener() {
	addDatapoolReceiveListener(new DatapoolAdapter( ) {

		public void valueChanged(DatapoolEvent e) {
			Object data = e.getValue( );
			if (data instanceof Number) {
				final Number angle = (Number) data;
				SWTUtil.asyncExec(new Runnable( ) {

					@Override
					public void run() {
						pluginData.setAngle(angle);
					}

				});

			}
		}

	});
}
\end{java}

Die Methode \codeQuote{addDatapoolReceiveListener()} wird ebenfalls vom
\index{Plugin!IPlugable!AbstractSensorPlugin}\codeQuote{AbstractSensorPlugin}
bereitgestellt. In diesem Fall liest \codeQuote{getSensor()} direkt den
\index{Datenpool!Schl�ssel}Datenpool-Schl�ssel des Kompass-\seegls{Sensor}s aus dem Roboter \seegls{Profil} mit aus. Dieser
Schl�ssel wird als Feld in der Klasse abgelegt und beim Aufruf von
\codeQuote{addDatapoolReceiveListener()} zum anmelden des gegebenen
\index{Datenpool!Listener}
\codeQuote{DatapoolListener} oder \codeQuote{DatapoolAdapter} genutzt.

Der Vorteil bei der Nutzung von \codeQuote{addDatapoolReceiveListener()} ist,
dass ein Abmelden vom Datenpool beim Beenden des \seegls{Plugins} automatisch ausgef�hrt
wird. 

Ansonsten w�rde der Code deutlich l�nger sein:
\index{Manager!ProfileManager}
\index{Manager!DatapoolManager}
\begin{java}[caption=Auslesen von Sensorspezifikation aus Roboter Profil (lang),label=lst:smplplug:sensordata:compassplugin:long]
private String datapoolKey;

@Override
public void runInternal() {
	try {
		// Roboterdaten vom Profil holen
		Robot robot = ProfileManager.getRobot(robotName);
		// Sensorgruppen die diesem Plugin zugeordnet sind
		// abrufen
		List<Sensorgroup> sensorgroups = robot.getSensorgroups(this.getClass( )
				.getName( ),
				this.uniqueIdentifier);
		if (!sensorgroups.isEmpty( )) {
			if (sensorgroups.size( ) > 1) {

				logClass.info(robotName,
						"Mehr als eine Sensogruppe vorhanden, es wird aber nur die erste genutzt." +
								Constants.LINE_SEPARATOR);
			}
			// Erste Sensorgruppe benutzen
			Sensorgroup group = sensorgroups.get(0);
			if (group != null) {
				// Sensoren der Gruppe auslesen und ersten
				// benutzen
				List<Sensor> sensors = group.getSensors( );
				if (!sensors.isEmpty( )) {
					if (sensors.size( ) > 1) {
						logClass.info(robotName,
								"Mehr als ein Sensor vorhanden, es wird aber nur die erste genutzt." +
										Constants.LINE_SEPARATOR);
					}
					Sensor sensor = sensors.get(0);
					if (sensor != null) {
						// Datenpoolschl�ssel holen
						this.datapoolKey = DatapoolUtil.createDatapoolKey(group,
								sensor);
					}
				}
			}
		}
	}
	catch (RobotNotFoundException e) {
		logClass.error(this.robotName, "Error " + e.getMessage( ) //$NON-NLS-1$
				+ Constants.LINE_SEPARATOR, e);
	}
}

private DatapoolListener listener;

private void initListener() {
	//Listener erstellen
	listener = new DatapoolListener( ) {

		@Override
		public boolean notifyOnlyWhenChanged() {
			return true;
		}

		@Override
		public void valueChanged(DatapoolEvent e) {
			Object data = e.getValue( );
			if (data instanceof Number) {
				final Number angle = (Number) data;
				SWTUtil.asyncExec(new Runnable( ) {

					@Override
					public void run() {
						pluginData.setAngle(angle);
					}

				});

			}
		}

	};

	//Datenpool des Roboters holen und Listener anmelden
	Datapool datapool = DatapoolManager.getDatapool(robotName);
	datapool.addDatapoolReceiveListener(datapoolKey, listener);
}

@Override
protected boolean stopInternal() {
	//Datenpool des Roboters holen und Listener abmelden
	Datapool datapool = DatapoolManager.getDatapool(robotName);
	datapool.addDatapoolReceiveListener(datapoolKey, listener);
	return true;
}
\end{java}

Wichtig ist in jedem Fall das saubere Pr�fen der �ber den
\index{Datenpool!Listener}Datenpool-Listener empfangenen Daten mit
\codeQuote{instanceof} vor dem \codeQuote{casten}.

Warum hier \index{SWTUtil!asyncExec()} genutzt wird kann im Abschnitt
\refSec{sec:tipstricks} und Kapitel
\refSec{sec:swt} nachgelesen werden.

Beispiele f�r \seegls{Sensor}plugins finden sich beim \robotQuote{TesterBot} die in
dessen Roboter \seegls{Profil} \fileQuote{conf/profiles/robots/testerbot.bot}
aufgelistet sind.