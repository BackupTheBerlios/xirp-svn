\section{Ein erstes Plugin}
\label{sec:firstplugin}
\index{Plugin!erstellen|(}
Voraussetzungen:
\begin{itemize}
\item Java (\seegls{JDK}) ab 1.6
\item Entwicklungsumgebung wie eclipse
\item \xirp~Installation
\end{itemize}

Der erste Schritt ist das erstellen eines neuen Projektes und hinzuf�gen des
\xirp~jar aus dem Installationsordner sowie
des \seegls{SWT} Jar aus dem Unterordner \fileQuote{lib/windows} oder \fileQuote{lib/linux} und hinzuf�gen.

Dann kann eine neue Klasse erstellt werden, welche
\index{Plugin!IPlugable!AbstractPlugin}
\newline\codeQuote{de.unibremen.rr.xirp.plugin.AbstractPlugin} erweitert (\refFig{newPluginClass}). \par
\kfig{plugin_plugin_newPluginClass}{1}{Erstellen einer neuen Pluginklasse}{newPluginClass}

Damit die Applikation wei�, dass diese neu erstellte Klasse die \index{Plugin!Hauptklasse}Hauptklasse 
des Plugins ist, muss eine \index{Plugin!Properties}properties Datei erstellt werden (\refFig{properties}).

Eine Vorlage dieser Properties Datei befindet sich bei der Installation im Ordner 
\fileQuote{development} als \fileQuote{plugin.properties.template}. Diese Datei in das Paket der 
\seegls{Plugin}-Klasse kopieren und die Endung \texttt{.template} entfernen.

\kfig{plugin_properties}{1}{Eine \enquote{plugin.properties}-Datei}{properties}

Nun k�nnen die Methoden mit Inhalten gef�llt werden.

Die Methode \index{Plugin!IPlugable!runInternal()}\codeQuote{runInternal()} 
wird ausgef�hrt wenn das Plugin gestartet wird. Um eine Ausgabe im 
Systemprotokoll von \xirp~zu erhalten muss noch eine Konstante f�r das 
\seegls{Logging} definiert werden.

Dies sieht wie folgt aus:
\codeQuote{private static final RobotLogger LOGGER = RobotLogger.getLogger(MyPlugin.class);}

Der \index{Roboter!Logging}\index{Logging!Roboter}\codeQuote{RobotLogger} ist im Paket 
\codeQuote{de.unibremen.rr.xirp.io.logging} zu finden. 

Die Ausgabe in 
\codeQuote{runInternal()} erfolgt dann �ber 
\codeQuote{LOGGER.info(robotName,''Ausgabe'');}. 

Die Methoden 
\index{Plugin!IPlugable!getPluginType()}\codeQuote{getPluginType()} gibt den 
Typ des Plugins zur�ck. Die \seegls{Plugin} Typen kommen aus der Klasse 
\index{Plugin!PluginType}\codeQuote{de.unibremen.rr.xirp.plugin.PluginType}. �hnliches
gilt f�r den 
\index{Plugin!VisualizationType}Darstellungstyp den die Methoden 
\index{Plugin!IPlugable!getVisualizationType()}\codeQuote{getVisualizationType()} 
zur�ckgibt.

Die Methoden 
\index{Plugin!IPlugable!getDescriptionKey()}\codeQuote{getDescriptionKey()} und 
\index{Plugin!IPlugable!getNameKey()}\codeQuote{getNameKey()} geben einen Schl�ssel 
zur \index{Internationalisierung!Schl�ssel}�bersetzung des \seegls{Plugin} Namens und der \seegls{Plugin} Beschreibung zur�ck.

\kfig{plugin_basicProperties}{1}{�bersetzungsdateien f�r ein Plugin}{basicProperties}

F�r die �bersetzung kommen nun noch zwei weitere Dateien 
(\refFig{basicProperties}) hinzu:\\
\index{Internationalisierung!Plugin}\fileQuote{messages$\_$de.properties} f�r 
die Deutsche �bersetzung und 
\index{Internationalisierung!Plugin}\fileQuote{messages$\_$en.properties} f�r 
die Englische �bersetzung.

Die entstandene Java-Klasse ist in \autoref{stubsFilled} zu sehen.

Das erste \seegls{Plugin} ist nun fast fertig. Nun muss nur noch ein Jar daraus erstellt
werden. Dazu kann aus dem Ordner \fileQuote{development} der Installation die \seegls{Ant}-\fileQuote{build.xml}
in den Hauptordner des Projekts kopiert werden.

Die Pfade oben in der \index{Ant!build.xml}\fileQuote{build.xml} m�ssen entsprechend angepasst werden. Dies 
betrifft normalerweise nur den Eintrag \codeQuote{plugins.pkg}.

Dann kann der Task \index{Ant!build.xml!singlePlugin}\codeQuote{singlePlugin} ausgef�hrt werden. Im Ordner \fileQuote{jar} liegt nun ein 
jar des \seegls{Plugins}. Dieses kann nun in die \xirp~Installation in den Ordner \fileQuote{plugins} 
kopiert werden.

Startet man nun \xirp~wird das \seegls{Plugin} geladen. Ob dies funktioniert hat kann man 
�ber das Men� \menuQuote{?} und den Eintrag \index{Men�!?!�ber Plugins}\menuQuote{�ber Plugins}
herausfinden. Wurde das \seegls{Plugin} 
geladen so ist es dort in der Tabelle aufgelistet (\refFig{aboutPlugins}).

\kfig{plugin_stubsFilled}{1}{Eine einfache Pluginklasse: MyPlugin.java}{stubsFilled}

\kfig{plugin_aboutPlugins}{1}{Men�: �ber Plugins}{aboutPlugins}

Damit das \seegls{Plugin} ausgef�hrt wird, muss es einem Roboter hinzugef�gt werden.

Dazu kann der Roboter \robotQuote{TesterBot} genutzt werden. Die Definition dieses Roboters 
befindet sich in der Installation \index{Profil!Roboter}\fileQuote{conf/profiles/robots/testerbot.bot}.

\newpage
Um das \seegls{Plugin} einzubinden muss einfach nur ein Element zum Element \codeQuote{plugins} ganz 
unten hinzugef�gt werden welches wie folgt aussieht:\index{Profil!Roboter!Plugin}

\begin{xml}[caption=Plugin in Roboterprofil eintragen,label=lst:smplplug:first:xml]
<plugin name="MeinPlugin">
	<class>xirp.plugins.MyPlugin</class>
	<usemultimedia>false</usemultimedia>
</plugin>
\end{xml}

Startet man \xirp~nun wieder kann man den Roboter \robotQuote{TesterBot} ausw�hlen und findet 
im Men� unter \index{Men�!Plugins}\menuQuote{Plugins} das eigene \seegls{Plugin} wieder. Klickt man darauf so wird das 
\index{Plugin!ausf�hren}\seegls{Plugin} ausgef�hrt und es sollte eine entsprechende Ausgabe im \index{Logging!Systemprotokoll}Systemprotokoll 
erscheinen.

\index{Plugin!erstellen|)}