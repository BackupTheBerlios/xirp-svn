\section{Reporte Erstellen}
\label{sec:plugin:report}
\index{Report}

Die \seegls{Plugin}-\seegls{API} bietet auch die M�glichkeit so genannte Reports zu erstellen. Dies
sind \seegls{PDFs} welche selbst zusammengestellt werden k�nnen.

F�r das Beispielplugin kann so zum Beispiel die Anzahl der erfolgten Klicks
zusammen mit einem Screenshot der aktuellen Oberfl�che ausgegeben werden.

Dazu muss das \seegls{Plugin} als Typ zus�tzlich \codeQuote{PluginType.REPORT}
zur�ckgeben. Weiterhin m�ssen die Methoden \codeQuote{getReport()} und
\codeQuote{hasReport()} implementiert werden:
\index{Manager!ProfileManager}
\begin{java}[caption=Reporterstellung eines Plugins,label=code:plugin:report]
@Override
public int getPluginType() {
	return PluginType.TOOLBAR | PluginType.REPORT;
}

@Override
public boolean hasReport() {
	return pluginData.getClickCount( ) > 0;
}

@Override
public Report getReport() {
	Report r = null;
	try {
		Header h = new Header("Klickz�hler",
				ProfileManager.getRobot(robotName),
				this);
		Content c = new Content( );
		ContentPart part = new ContentPart( );
		ContentPartText text = new ContentPartText( );
		ContentPartTextParagraph para = new ContentPartTextParagraph( );
		para.setHeader("Klickanzahl");
		para.setText("In diesem Plugin wurde " +
				pluginData.getClickCount( ) + " mal geklickt.");
		text.addParagraph(para);
		part.addReportText(text);

		if (!guis.isEmpty( )) {
			MyPluginGUI gui = guis.firstElement( );
			Image image = gui.captureScreenshot( );
			ImageLoader loader = new ImageLoader( );
			loader.data = new ImageData[] {image.getImageData( )};
			String imagePath = Constants.TMP_DIR + Constants.FS +
					getClass( ).getName( ) + ".png";
			loader.save(imagePath, SWT.IMAGE_PNG);
			image.dispose( );

			ContentPartImage imagePart = new ContentPartImage(imagePath,
					"Screenshot der aktuellen GUI");

			part.addReportImage(imagePart);
		}

		c.addReportPart(part);
		r = new Report(h, c, h.getTitle( ));
	}
	catch (Exception e) {
		LOGGER.error(robotName, "Error: " + e.getMessage( ) +
				Constants.LINE_SEPARATOR, e);
	}
	return r;
}
\end{java}

Der Report soll nur erstellt werden, wenn �berhaupt schon einmal geklickt wurde.

F�r die Erstellung des Screenshot kann die Methode \codeQuote{captureScreenshot}
der Oberfl�che genutzt werden. Es reicht dabei einen Screenshot von einer
Oberfl�che zu nehmen (in diesem Fall der ersten), da diese ja alle das selbe
anzeigen.

\kfig{plugin_report_menu}{.5}{Plugin im Reportmen�}{img:plugin:report}

Das Bild muss vor der Einbindung in das \seegls{PDF} mit dem \codeQuote{ImageLoader} 
gespeichert werden, damit es
von einem Pfad direkt bei der \seegls{PDF} Erstellung eingebunden werden kann.

Im Men� unter \index{Men�!Reporte}Reporte (siehe \autoref{img:plugin:report} l�sst sich dann der
Report f�r das \seegls{Plugin} erstellen.
Dieser �ffnet sich direkt und ist im Ordner \fileQuote{reports} sowie �ber die
Reportsuche zu finden.


In \autoref{img:plugin:report:cutout} ist ein Ausschnitt des erstellten \seegls{PDFs} zu sehen.

\kfig{plugin_report_cutout}{1}{Ausschnitt eines Pluginreports}{img:plugin:report:cutout}

Mehr �ber die Reporterstellung kann im \autoref{sec:report} auf
\autopageref{sec:report} nachgelesen werden.