\section{Bibliotheken}
\index{Plugin!Bibliotheken}

\seegls{Plugins} bieten die M�glichkeit zus�tzliche \seegls{Bibliotheken} zu nutzen. Dazu m�ssen
diese im \seegls{Plugin} in einem Ordner \fileQuote{lib} liegen. Dieser Ordner enth�lt
die Unterordner \fileQuote{windows},\fileQuote{linux},\fileQuote{dll} und
\fileQuote{so}. 

Platformunabh�ngige Jars k�nnen dabei direkt im Ordner \fileQuote{lib} abgelegt
werden. Jars die speziell f�r ein Betriebssystem sind entsprechend in den
Unterordner \fileQuote{windows} und \fileQuote{linux}. Native \seegls{Bibliotheken} wie
\seegls{DLL}'s und \seegls{SO}'s k�nnen in den anderen Ordner abgelegt werden.

Vor der Ausf�hrung des \seegls{Plugins} ermittelt \xirp~das aktuelle Betriebssystem und
l�dt die entsprechenden Jars. Native \seegls{Bibliotheken} werden in den library path
kopiert m�ssen aber zur Benutzung mit \codeQuote{System.loadLibrary()} ggfs. 
selber geladen werden.

Um das Vorhandensein bestimmten \seegls{Bibliothek}en oder anderer \seegls{Plugins} sicherzustellen kann ein Plugin die Methode
\index{Plugin!IPlugable!requiredLibs()}\codeQuote{requiredLibs()} �berschreiben.
Diese Methode gibt eine Liste von \codeQuote{String}s zur�ck. Bei der Suche nach
\seegls{Plugins} wird diese Liste �berpr�ft. Ist eine der
\index{Plugin!Abh�ngigkeiten}Abh�ngigkeiten des \seegls{Plugins} nicht vorhanden, wird
das \seegls{Plugin} nicht geladen. Entsprechend werden \seegls{Plugins} die dieses \seegls{Plugin} in ihren
Abh�ngigkeiten aufgef�hrt haben auch nicht geladen. Im \index{Logging!Systemprotokoll}\seegls{Log} 
sind in diesem Fall Ausschriften zu finden die �ber die Gr�nde Auskunft geben.

Die Liste von Strings kann folgenden Inhalt haben:
\begin{itemize}
  \item Endet ein String mit \enquote{.jar} so wird der Classpath nach einem Jar
  mit dem gegebenen Namen durchsucht
  \item Andernfalls wird der String als Klassenname eines \seegls{Plugins} interpretiert
  und versucht das \seegls{Plugin} zu finden
\end{itemize}