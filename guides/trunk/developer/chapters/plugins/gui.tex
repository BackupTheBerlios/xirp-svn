\section{Benutzeroberfl�che}
\index{Plugin!Benutzeroberfl�che!erstellen|(}
Bisher hat das kleine Beispielplugin noch keine eigene Oberfl�che, wenn man von
dem \index{Bild!Icon}\seegls{Icon} in der \index{Toolbar}Toolbar einmal absieht. Dies soll sich nun �ndern.

F�r die Oberfl�che wird eine neue Klasse ben�tigt die
\index{Plugin!Benutzeroberfl�che!AbstractPluginGUI}\codeQuote{AbstractPluginGUI} erweitert.

Die \index{Plugin!Hauptklasse}Basisklasse des \seegls{Plugins} \codeQuote{MyPlugin} ist eine Klasse die
\index{generisch}\seegls{generisch} ist und eigentlich \index{generisch!Typparameter}\seegls{Typparameter} ben�tigt. Diese wurden bisher zur
Vereinfachung weggelassen, werden nun aber ben�tigt. Die Klasse
\codeQuote{MyPlugin} ist wie in \autoref{basicGUI} zu sehen zu erweitern.

\kfig{plugin_basicGUI}{1}{Generische Typen in der Pluginklasse einstellen}{basicGUI}

Nun muss in \codeQuote{MyPlugin} die Methode \index{Plugin!IPlugable!getGUIInternal()}\codeQuote{getGUIInternal()}
�berschrieben werden, so dass sie eine neue Instanz von \codeQuote{MyPluginGUI} zur�ckgibt.

\begin{java}[caption=Oberfl�che aus Plugin zur�ckgeben,
				 label=lst:smplplug:gui_gui_getGUIInternal]
	@Override
	protected MyPluginGUI getGUIInternal(Composite parentPanel) {
		return new MyPluginGUI(parentPanel, this);
	}
\end{java}

Nun muss noch \xirp~mitgeteilt werden, dass dieses \seegls{Plugin} nun eine eigene
Oberfl�che besitzt. Dazu wird die R�ckgabe von
\index{Plugin!IPlugable!getVisualizationType()}\codeQuote{getVisualizationType()} ver�ndert:

\begin{java}[caption=Visualisierungstyp eines Plugin setzen,
				 label=lst:smplplug:gui_gui_getVisualizationType]
	@Override
	public int getVisualizationType() {
		return VisualizationType.ROBOT_TOOLBAR | VisualizationType.WINDOW;
	}
\end{java}
\index{Manager!ColorManager}
\index{Manager!FontManager}
Nun muss nur noch die Oberfl�che selbst ein wenig mit Leben gef�llt werden:
\begin{java}[caption=Beispiel Plugin-Oberfl�che,
				 label=lst:smplplug:gui_gui_sample]
package xirp.plugins;

import org.eclipse.swt.SWT;
import org.eclipse.swt.layout.GridData;
import org.eclipse.swt.widgets.Composite;

import de.unibremen.rr.xirp.plugin.AbstractPluginGUI;
import de.unibremen.rr.xirp.ui.util.SWTUtil;
import de.unibremen.rr.xirp.ui.util.ressource.ColorManager;
import de.unibremen.rr.xirp.ui.util.ressource.FontManager;
import de.unibremen.rr.xirp.ui.widgets.custom.XLabel;

public class MyPluginGUI extends AbstractPluginGUI {

	private MyPlugin plugin;

	public MyPluginGUI(Composite parent, MyPlugin plugin) {
		super(parent, plugin);
		this.plugin = plugin;
		init( );
	}

	private void init() {
		SWTUtil.setGridLayout(this, 3, true);
		setBackground(ColorManager.getColor(255, 0, 0));
		XLabel first = new XLabel(this, SWT.NONE, plugin.getHandler( ));
		first.setTextForLocaleKey("MyPluginGUI.firstLabel");
		SWTUtil.setGridData(first,
				true,
				true,
				GridData.CENTER,
				GridData.CENTER,
				1,
				1);

		XLabel second = new XLabel(this, SWT.NONE, plugin.getHandler( ));
		second.setTextForLocaleKey("MyPluginGUI.secondLabel");
		second.setFont(FontManager.getFont("Arial", 12, SWT.BOLD));
		SWTUtil.setGridData(second,
				true,
				true,
				GridData.FILL,
				GridData.CENTER,
				2,
				2);
	}
}
\end{java}

Hier werden 4 weitere neue Methoden gezeigt:
\begin{itemize}
\item Das erstellen von Farben mit dem \index{Manager!ColorManager}\codeQuote{ColorManager}. Diese m�ssen
nicht \index{SWT!dispose}\seegls{disposed} werden.
\item Das erstellen von Schriften mit dem \index{Manager!FontManager}\codeQuote{FontManager}. Diese m�ssen
ebenfalls nicht \seegls{disposed} werden.
\item Zuweisen eines Layout mit \index{SWTUtil!setGridLayout()}\codeQuote{SWTUtil.setGridLayout()}
\item Das zuweisen von Layout Daten mit \index{SWTUtil!setGridData()}\codeQuote{SWTUtil.setGridData()}
\end{itemize}

Das \seegls{Plugin} ist nun im \index{Men�!Plugins}\menuQuote{Plugins}-Men� zu finden und zeigt einen roten
Hintergrund mit zwei Texten (\refFig{GUITest}).

\kfig{plugin_GUITest}{1}{Eine kleine Benutzeroberfl�che f�r ein Plugin}{GUITest}

Damit das \seegls{Plugin} innerhalb des \index{Workspace}\seegls{Workspaces} des Roboters angezeigt wird
und nicht �ber das Men� aufgerufen werden muss, muss nur der Visualisierungtyp
von \codeQuote{WINDOW} auf \codeQuote{EMBEDDED} ge�ndert werden.
\index{Plugin!Benutzeroberfl�che!erstellen|)}