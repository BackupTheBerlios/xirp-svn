\section{Daten und Oberfl�che}
\index{Plugin!Plugindaten!erstellen|(}
Ein \seegls{Plugin} kann mehrere Oberfl�chen haben, die jedoch immer das selbe anzeigen
sollen. Dies resultiert daraus, dass eine Oberfl�che z.B. in der �bersicht des
Roboters und eine Zweite in der Gesamt�bersicht aller Roboter zu sehen ist.

Damit diese nun das gleiche Anzeigen besteht die M�glichkeit diese �ber eine
Datenklasse zu synchronisieren.

Auch f�r das \seegls{Plugin} selbst bietet sich so eine bessere Steuerung der Oberfl�che.
Es braucht nur die Daten zu setzen und die Oberfl�che zeigt diese dann
an. 

Um Daten zu nutzen, muss eine neue Klasse \codeQuote{MyPluginData} erstellt 
werden, welche \index{Plugin!Plugindaten!AbstractData}\codeQuote{AbstractData} 
erweitert.

Dem Konstruktor sollte direkt einen Parameter f�r die Plugin Basisklasse
hinzugef�gt werden:
\begin{java}[caption=Plugindaten erstellen,label=lst:smplplug:data:newclass]
package xirp.plugins;

import de.unibremen.rr.xirp.plugin.AbstractData;

public class MyPluginData extends AbstractData {

	private MyPlugin plugin;

	public MyPluginData(MyPlugin plugin) {
		super( );
		this.plugin = plugin;
	}
}
\end{java}

In der Klasse \codeQuote{MyPlugin} kann nun der \index{generisch!Typparameter}\seegls{generische} \seegls{Typparamter} \codeQuote{AbstractData}
durch \codeQuote{MyPluginData} ersetzt werden:
\begin{java}[caption=Plugindaten in Plugin eintragen,label=lst:smplplug:data:plugin_extends]
public class MyPlugin extends AbstractPlugin<MyPluginData, MyPluginGUI> {
...
}
\end{java}

Nun kann die Methode \index{Plugin!IPlugable!getPluginDataInternal()}\codeQuote{getPluginDataInternal()} �berschrieben werden:
\begin{java}[caption=Plugindaten aus Plugin zur�ckgeben,label=lst:smplplug:data:plugin:getPluginDataInternal]
	@Override
	protected MyPluginData getPluginDataInternal() {
		return new MyPluginData(this);
	}
\end{java}

Nun wird der \codeQuote{MyPluginData} ein Feld \codeQuote{toPrint} und
entsprechende Getter und Setter hinzugef�gt. Der eigentliche Setter muss nun ein wenig
modifiziert werden, so dass er beim Aufruf eine \index{Plugin!Plugindaten!notifyUI()}Benachrichtigung sendet, dass
sich Daten ge�ndert haben.

\begin{java}[caption=Methoden in Plugindaten hinzuf�gen
,label=lst:smplplug:data:data:methods]
public class MyPluginData extends AbstractData {

	private MyPlugin plugin;
	private String toPrint;

	public MyPluginData(MyPlugin plugin) {
		super( );
		this.plugin = plugin;
	}

	public String getToPrint() {
		return toPrint;
	}

	public void setToPrint(String toPrint) {
		plugin.notifyUI("toPrint", this.toPrint, this.toPrint = toPrint);
	}
}
\end{java}

Die Oberfl�che soll nun ein Textfeld anzeigen:
\begin{java}[caption=Oberfl�che zur Plugindatendarstellung,label=lst:smplplug:data:gui:init]
	private void init() {
		GridLayout gl = SWTUtil.setGridLayout(this, 1, true);
		SWTUtil.resetMargins(gl);

		this.setBackground(ColorManager.getColor(SWT.COLOR_RED));

		text = new XText(this, SWT.READ_ONLY | SWT.WRAP, false);
		SWTUtil.setGridData(text,
				true,
				true,
				GridData.FILL,
				GridData.FILL,
				1,
				1);

	}
\end{java}

Der Hintergrund der Oberfl�che ist eigentlich Rot, jedoch nimmt das Textfeld
durch die \codeQuote{FILL}-Attribute und das entfernen von R�ndern mit
\index{SWTUtil!resetMargins()}\codeQuote{SWTUtil.resetMargins()} den gesamten verf�gbaren Platz ein.
\par
Damit dieses Textfeld �nderungen der Daten verarbeitet fehlt noch eine neue
Methode welche im Konstruktor aufgerufen wird: 
\begin{java}[caption=Listener der Oberfl�che auf Plugindaten,label=lst:smplplug:data:gui:listener]
	public MyPluginGUI(Composite parent, MyPlugin plugin) {
		super(parent, plugin);
		this.plugin = plugin;
		init( );
		initListener( );
	}

	private void initListener() {
		MyPluginData pluginData = this.plugin.getPluginData( );
		pluginData.addPropertyChangeListener("toPrint",
				new PropertyChangeListener( ) {

					@Override
					public void propertyChange(PropertyChangeEvent event) {
						text.append(event.getNewValue( ).toString( ));
					}

				});
	}
\end{java}

In dieser Methode erfolgt die Anmeldung auf �nderungen der Eigenschaft
\codeQuote{toPrint} der \seegls{Plugin} Daten. �ndern sich die Daten, so wird dies im
Textfeld ausgegeben.

Nun fehlt noch das schreiben der Daten beim klicken auf das \index{Toolbar}Toolbar 
\index{Bild!Icon}\seegls{Icon}. Hierzu muss die Methode 
\index{Plugin!IPlugable!getToolbar()}\codeQuote{getToolbar()} ge�ndert werden.

\begin{java}[caption=Setzen von Plugindaten aus der Oberfl�che,label=lst:smplplug:data:gui:toolbar]
	@Override
	public XToolBar getToolBar(CoolItem parent) {
		XToolBar tools = new XToolBar(parent.getParent( ), SWT.FLAT);
		XToolItem helloWorld = new XToolItem(tools,
				SWT.FLAT | SWT.PUSH,
				handler);
		helloWorld.setToolTipTextForLocaleKey("MyPlugin.toolbar.helloworld"); //$NON-NLS-1$
		helloWorld.setImage(ImageManager.getPluginImage(this, "helloworld.png")); //$NON-NLS-1$
		helloWorld.addSelectionListener(new SelectionAdapter( ) {

			/*
			 * (non-Javadoc)
			 * 
			 * @see org.eclipse.swt.events.SelectionAdapter#widgetSelected(org.eclipse.swt.events.SelectionEvent)
			 */
			@Override
			public void widgetSelected(SelectionEvent e) {
				String string = getI18nString("MyPlugin.print.helloworld",
						MyPlugin.this.getRobotName( ),
						toPrint)
						+ Constants.LINE_SEPARATOR;
				LOGGER.info(robotName, string);
				pluginData.setToPrint(string);
			}
		});

		return tools;
	}
\end{java}

Immer wenn die Sprache umgestellt wurde oder in den \index{Einstellungen}Einstellungen ein neuer Text
f�r das \seegls{Plugin} gesetzt wurde, erscheint die Ausgabe beim Klick auf das Toolbar
\seegls{Icon} nun auch in der Oberfl�che des \seegls{Plugins} (\refFig{dataOut}). 

\kfig{plugin_dataOut}{.7}{Ausgabe in der Plugin Oberfl�che}{dataOut}
\index{Plugin!Plugindaten!erstellen|)}