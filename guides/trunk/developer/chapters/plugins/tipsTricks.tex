\section{Tips und Tricks}
\label{sec:tipstricks}
\index{Plugin!Tips und Tricks}


\subsection{Datenpool}
\index{Datenpool}
Beim anmelden auf den Datenpool ist bei \seegls{Sensor}en darauf zu achten, dass der
\index{Datenpool!Schl�ssel}Datenpoolschl�ssel durch die Klasse 
\index{Datenpool!DatapoolUtil}\codeQuote{de.unibremen.rr.xirp.io.comm.data.DatapoolUtil}
ermittelt wird.

\seegls{Plugins} die Werte vom \index{Datenpool}Datenpool empfangen sollten 
diese Werte nicht direkten an die \seegls{GUI} sondern an eine 
\index{Plugin!Plugindaten!AbstractData}\codeQuote{AbstractData} Implementierung 
weitergeben. Von dort aus kann die Weitergabe mittels 
\codeQuote{plugin.notifyUI} erfolgen.

Werte die vom \index{Datenpool}Datenpool Empfangen werden kommen nicht aus dem 
\index{SWT!Thread}\seegls{SWT}-\seegls{Thread}. Da in \seegls{SWT} nur der 
\seegls{SWT}-Thread Zugriff auf die \seegls{GUI}-Komponenten hat, m�ssen Zugriffe aus 
der \index{Datenpool!Listener}Listener Implementierung heraus auf die \seegls{GUI} in 
einem \index{SWT!Thread!asyncExec()}\codeQuote{de.unibremen.rr.xirp. 
ui.util.SWTUtil.asyncExec()} ausgef�hrt werden. Dies gilt auch f�r Listener auf 
den Daten eines \seegls{Plugins}.

Weitere Informationen finden sich in \autoref{sec:datapool} ab 
\autopageref{sec:datapool}.

\subsection{Berechnungen bei Plugins mit Oberfl�che}
\label{sub:swtcalc}
Da es nur einen \index{SWT!Thread}\seegls{SWT}-\seegls{Thread} gibt sollten Berechnungen nicht in Zeichenmethoden
eines \seegls{Plugins} untergebracht werden. Berechnungen m�ssen nur dann durchgef�hrt
werden wenn sich die zugrunde liegenden Daten des \seegls{Plugins} oder die Gr��e der
Oberfl�che ge�ndert hat. Die Berechnungen k�nnen dann in geeigneten Feldern der
\seegls{GUI} abgelegt werden und m�ssen beim Zeichnen nur noch ausgelesen werden.

Wird die \seegls{GUI} ausgeblendet und tritt in der Zwischenzeit sonst keine Ver�nderung
auf, so muss dann nur gezeichnet aber nichts berechnet werden. Dies spart Zeit.

\subsection{Roboter zeichnen}
Zum Zeichnen des Roboters und seiner \seegls{Sensor}en gibt es die Klasse
\index{Roboter!zeichnen!RobotDrawHelper}\codeQuote{de.unibremen.rr.xirp.
ui.util.RobotDrawHelper}. Dieser �bergibt man
den zur Verf�gung stehenden Platz (die Gr��e des \seegls{GUI}-Composites) und den
gew�nschten Rand und es werden die Ma�e eines Roboterrechtecks und auf Anfrage
die Position eines \seegls{Sensor}s am Roboter berechnet (siehe
\autoref{code:robotdrawhelper} auf \autopageref{code:robotdrawhelper}).

\subsection{Einstellungen in Profil}
\index{Plugin!Profil!Optionen}
\index{Einstellungen!Profil}
Es ist m�glich, dass das \seegls{Plugin} Optionen anbieten m�chte, die nicht in den
Einstellungen vorkommen sollen, sondern nur einmal konfiguriert werden sollen.

Diese k�nnen im \seegls{Profil} im Bereich des \codeQuote{plugin}-Elements als Optionen
angegeben werden.

Informationen dazu finden sich in \autoref{sec:profile:plugins} ab
\autopageref{sec:profile:plugins}.