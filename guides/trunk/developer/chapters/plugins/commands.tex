\section{Kommandos}
\label{sec:plugin:commands}
\index{Kommando}
\index{Kommando!Manager}
Um ein \seegls{Plugin} per Tastatur oder \index{Gamepad}\seegls{Gamepad} steuerbar zu machen kann zus�tzlich noch
das Interface \codeQuote{de.unibremen.rr.xirp.io.command.ICommandable.java}
implementieren.

Im Beispielplugin soll so per Tastaturdruck der Klickz�hler erh�ht werden.

Nach folgendem Schema k�nnen Kommandos sehr einfach f�r \seegls{Plugins} implementiert werden:
\index{Manager!CommandManager}
\begin{java}[caption=Kommandodefinition im Beispielplugin,label=lst:smplplug:commands]
	@Override
	protected void runInternal() {
		readSettings( );
		CommandManager.register(this, robotName);
	}

	private static final String INCREMENT_CLICK_KEY = "MyPlugin.command.incrementClick";

	@Override
	public boolean executeCommand(Command command) {
		String key = command.getCommandDefinition( ).getKey( );
		boolean executed = false;
		if (key.equalsIgnoreCase(INCREMENT_CLICK_KEY)) {
			pluginData.incrementClickCount( );
			executed = true;
		}
		return executed;
	}


	private CommandDefinition getCommandDefinition(String key) {
		List<String> parameterNames = Collections.emptyList( );
		List<GeneratorType> types = Arrays.asList(GeneratorType.values( ));
		CommandDefinition def = new CommandDefinition(this,
				key,
				parameterNames,
				types,
				this.handler);
		return def;
	}

	@Override
	public List<CommandDefinition> getCommandDefinitions() {
		List<CommandDefinition> commandDefs = new ArrayList<CommandDefinition>( );
		commandDefs.add(getCommandDefinition(INCREMENT_CLICK_KEY));

		return commandDefs;
	}
\end{java}

Der Schl�ssel muss zu den \index{Internationalisierung!Schl�ssel}�bersetzungen hinzugef�gt werden. Er beschreibt in Kurzform
was das Kommando tut. Wichtig ist, das \seegls{Plugin} in \codeQuote{runInternal()} beim
\codeQuote{CommandManager} zu registrieren.

\index{Hotkey}
In \xirp~l�sst sich nun in den \index{Einstellungen!Hotkey}Einstellungen im Bereich \menuQuote{Tastenk�rzel}
auf dem Tab \menuQuote{Kommando Tastenk�rzel} auf der Tastatur eine Taste
eintippen die f�r das Kommando benutzt werden soll (\refFig{commandView}). 
Speichert und schlie�t man diese Einstellungen, so l�sst sich das \seegls{Plugin} nun mit
der eingestellten Taste tats�chlich steuern.

\kfig{plugin_commandView}{1}{Hotkeys f�r Kommandos einstellen}{commandView}