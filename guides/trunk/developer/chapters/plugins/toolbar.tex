\section{Die Toolbar}
\label{sec:plugin:toolbar}
\index{Toolbar}
Ein \seegls{Plugin} kann Funktionen auf die Toolbar des Roboters legen. Dazu 
muss der \seegls{Plugin} Typ auf 
\index{Plugin!PluginType}\codeQuote{PluginType.TOOLBAR} und der Darstellungstyp 
auf 
\index{Plugin!VisualizationType}\codeQuote{VisualizationType.ROBOT\_TOOLBAR} 
ge�ndert werden und die Methode 
\index{Plugin!IPlugable!getToolbar()}\codeQuote{getToolbar} �berschrieben 
werden.

Als Beispiel wird nun ein Button erstellt, welcher beim klicken einen Text
ausgibt.

Der Button soll ein \index{Bild!Icon}\seegls{Icon} statt Text haben. Daf�r wird ein neuer Ordner 
Namens \index{Plugin!Ordnerstruktur!images}\fileQuote{images} neben dem Ordner 
\index{Plugin!Ordnerstruktur!lib}\codeQuote{lib} im Plugin Paket angelegt in 
welchen ein \index{Bild!im Plugin}Bild \fileQuote{hallowelt.png} gelegt wird.
\index{Manager!ImageManager}
\begin{java}[caption=Eine Plugin Toolbar,
				 label=lst:smplplug:toolbar_plugin_getToolBar]
	@Override
	public XToolBar getToolBar(CoolItem parent) {
		XToolBar tools = new XToolBar(parent.getParent( ), SWT.FLAT);
		// �bersetzungshandler dieses Plugins mit �bergeben
		XToolItem helloWorld = new XToolItem(tools,
				SWT.FLAT | SWT.PUSH,
				handler);
		// Bild vom ImageManager holen
		helloWorld.setImage(ImageManager.getPluginImage(this, "helloworld.png")); //$NON-NLS-1$
		// Key f�r den Tooltip setzen
		helloWorld.setToolTipTextForLocaleKey("MyPlugin.toolbar.helloworld"); //$NON-NLS-1$
		helloWorld.addSelectionListener(new SelectionAdapter( ) {

			/*
			 * (non-Javadoc)
			 * 
			 * @see org.eclipse.swt.events.SelectionAdapter#widgetSelected(org.eclipse.swt.events.SelectionEvent)
			 */
			@Override
			public void widgetSelected(SelectionEvent e) {
				// �bersetzte Logausgabe mit einer Variablen
				LOGGER.info(robotName,
						getI18nString("MyPlugin.print.helloworld",
								MyPlugin.this.getRobotName( ))
								+ Constants.LINE_SEPARATOR);
			}
		});

		return tools;
	}
\end{java}

Wichtig ist das nicht die original \index{Widget}\seegls{SWT} Widgets benutzt 
werden sondern die aus dem Paket 
\newline\index{Widget!Custom}\codeQuote{de.unibremen.rr.xirp.ui.widgets.custom}. Sie 
bieten alle die Methoden 
\newline\index{Widget!Custom!setToolTipTextForLocaleKey()}\codeQuote{setToolTipTextForLocaleKey} 
und \index{Widget!Custom!setTextForLocaleKey()}\codeQuote{setTextForLocaleKey} 
(oder �hnliche) und einen Konstruktor der einen 
\index{Internationalisierung}�bersetzungshandler bekommen kann, mit welchem die 
\index{Internationalisierung!Schl�ssel}Schl�ssel letztendlich �bersetzt werden. 
Weiterhin �bernehmen all diese Custom Widgets immer die \index{Farbe}aktuellen 
Farb- und \index{Internationalisierung}Spracheinstellung auch bei einer 
�nderung im laufenden Betrieb.

Das hinzugef�gte Bild kann vom
\index{Manager!ImageManager}\codeQuote{de.unibremen.rr.xirp.
ui.util.ressource.ImageManager} abgerufen
werden. Dieser handelt auch das eigentlich n�tige \index{SWT!dispose}\seegls{disposen} des Bildes nach der
Benutzung und \seegls{cached} die Bilder, so dass von jedem nur eine Instanz existiert
die mehrfach genutzt werden kann.

\kfig{plugin_toolbar}{.7}{Toolbar Icon eines Plugins}{toolbar}

Auch der Text der beim klicken auf den Button ausgegeben wird, wird �bersetzt.
Dazu dient die Methode \codeQuote{getI18nString()} welche einen Schl�ssel und optional
noch mehrere Parameter zur Ersetzung von Variablen erhalten kann. Wichtig ist
 auch das abschlie�ende \codeQuote{Constants.LINE\_SEPARATOR} welches einen
Zeilenumbruch f�r das aktuelle Betriebssystem enth�lt.

So sehen die ver�nderten \index{Internationalisierung!Plugin}\fileQuote{properties}-Dateien aus:\\
\textbf{de}

\begin{properties}[caption=Deutsche �bersetzungen,
				 label=lst:smplplug:toolbar_props_de]
MyPlugin.plugin.name=Mein Plugin
MyPlugin.plugin.description=Ein Testplugin

MyPlugin.toolbar.helloworld=Hallo Welt ausgeben
MyPlugin.print.helloworld={0} sagt: "Hallo Welt!"
\end{properties}

\newpage
\textbf{en}
\begin{properties}[caption=Englische �bersetzungen,
				 label=lst:smplplug:toolbar_props_en]
MyPlugin.plugin.name=My Plugin
MyPlugin.plugin.description=A test plugin

MyPlugin.toolbar.helloworld=Print out Hello World
MyPlugin.print.helloworld={0} says: "Hello World!"
\end{properties}

Das \seegls{Plugin} kann nun wiederum mit der 
\index{Ant!build.xml}\fileQuote{build.xml} erstellt und in den Ordner 
\fileQuote{plugins} des \xirp~Installationsordners kopiert werden.



Bei der Ausf�hrung erscheint dann f�r den \robotQuote{TesterBot} ein \index{Bild!Icon}\seegls{Icon} auf der
Toolbar welches eine Weltkugel zeigt (\refFig{toolbar}). Dieses gibt bei einem Klick
\begin{lstlisting}
TesterBot: TesterBot sagt: "Hallo Welt!"
\end{lstlisting}

bzw.
\begin{lstlisting}
TesterBot: TesterBot says: "Hello World!"
\end{lstlisting}