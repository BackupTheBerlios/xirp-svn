%
% header.tex
%
\documentclass[%
	pdftex,%              PDFTex verwenden
	a4paper,%             A4 Papier
	twoside,%             Einseitig
	bibtotoc,%    Literaturverzeichnis nummeriert einf�gen
	idxtotoc,%            Index ins Verzeichnis einf�gen
	halfparskip,%         Europ�ischer Satz mit abstand zwischen Abs�tzen
	chapterprefix,%       Kapitel anschreiben als Kapitel
	headsepline,%         Linie nach Kopfzeile
	%footsepline,%         Linie vor Fusszeile
	12pt,%                 Gr�ssere Schrift, besser lesbar am bildschrim
	pointlessnumbers,
	openright
]{scrbook}

\newcommand{\xirp}{$\chi$irp\index{xirp@$\chi$irp}}
\newcommand{\xirpname}{\enquote{eXtendable interface for robotic purposes}}
\newcommand{\version}{2.4.0}
\newcommand{\devguideurl}{http://developer.berlios.de/docman/display_doc.php?docid=1711&group_id=8442}
\newcommand{\devguideurlname}{xirp.berlios.de}
\newcommand{\userguideurl}{http://developer.berlios.de/docman/display_doc.php?docid=1710&group_id=8442}
\newcommand{\swttutorialurl}{http://www.cs.umanitoba.ca/~eclipse/}
\newcommand{\swtsnippeturl}{http://www.eclipse.org/swt/snippets/}
\newcommand{\jakartacommonsurl}{http://jakarta.apache.org/commons}

\usepackage[ngerman]{varioref}


\usepackage{scrpage2}
\usepackage{multicol}
% \renewcommand{\partformat}{\partname~\thepart}
% \renewcommand{\chapterformat}{\chapapp~\thechapter}
% \renewcommand{\othersectionlevelsformat}[1]{\csname the#1\endcsname\enskip}
% 
% \renewcommand{\partmarkformat}{\thepart\enskip}
% \renewcommand{\chaptermarkformat}{\chapapp~\thechapter\enskip}
% \renewcommand{\sectionmarkformat}{\thesection\enskip}
% \renewcommand{\subsectionmarkformat}{\thesubsection\enskip}
% \renewcommand{\subsubsectionmarkformat}{\thesubsubsection\enskip}
% \renewcommand{\paragraphmarkformat}{\theparagraph\enskip}
% \renewcommand{\subparagraphmarkformat}{\thesubparagraphmarkformat\enskip}

\renewcommand{\chaptermark}[1]{\markboth{\chaptermarkformat{#1}}{\chaptermarkformat{#1}}}

\renewcommand{\chapterpagestyle}{scrheadings}
\renewcommand{\indexpagestyle}{scrheadings}

% kopf und fusszeilen
\pagestyle{scrheadings}

\clearscrheadfoot
\rohead[\headmark~\pagemark]{\headmark~\pagemark}
\lehead[\pagemark~\headmark]{\pagemark~\headmark}
\setheadtopline{.4pt}
\setheadsepline{.4pt}

\setcounter{secnumdepth}{5}
\setcounter{tocdepth}{5}

\usepackage{textcomp}%f�r textuparrow

% ---------------------------------------
% Makro fuer einheitliche figures!
% Bilder immer ins pics verzeichnis!!
% Dateiname, Groesse, caption, label
% ---------------------------------------
\newcommand{\kfig}[4]
		{\figgiNeu{#1}{#2}{#3}{#4}}
%<-------------------------------------->
\newcommand{\figgiNeu}[4]
{
\begin{figure}[!ht]
	\centering
	\includegraphics[width=#2\textwidth]{images/{#1}}
	\caption{#3}
	\label{#4}
\end{figure}
}

\newcommand{\refFig}[1]
{\autoref{#1} auf \autopageref{#1}}

%\newcommand{\codeQuote}[1]
%{\texttt{{#1}}}
\newcommand*{\codeQuote}[1]{\lstinline!#1!}

\newcommand{\fileQuote}[1]
{\texttt{{#1}}}

\newcommand{\menuQuote}[1]
{\texttt{{#1}}}

\newcommand{\seegls}[1]{\textuparrow#1}

\newcommand{\robotQuote}[1]
{\index{#1}\texttt{\seegls{#1}}}

\newcommand{\refSec}[1]
{\autoref{#1} (\autopageref{#1})}

\newcommand{\longRefSec}[1]
{siehe \autoref{#1} auf \autopageref{#1}}



%
% Paket f�r die Indexerstellung.
%
\usepackage{makeidx}
\newcommand{\boldindex}[1]{{\bf #1}}
\newcommand{\mainindex}[1]{\index{#1|boldindex}}

%Formattierungen f�r Indexfile
%Erste Ebene
\newcommand*{\indexdelim}{\ \hspace{0pt plus 1fil}\penalty0\null\nobreak
  \dotfill~}
%Zweite Ebene
\newcommand*{\indexdelimi}{~\dotfill\penalty0\ }
%Dritte Ebene
\newcommand*{\indexdelimii}{~\dotfill\penalty0\ }

%Abschnitts�berschrift
\newcommand*{\indexsection}[1]{%
  \ifx\empty#1\empty\else
  \hspace{0pt plus 2fil}{{\usekomafont{sectioning} #1}}\hspace{0pt plus
    1fil}\nopagebreak
  \fi
}

%
% Paket f�r �bersetzungen ins Deutsche
%
\usepackage[ngerman]{babel}
\addto\extrasngerman{%
\def\partautorefname{Teil}
\def\pageautorefname{\pagename}%
\def\figureautorefname{\figurename}%
\def\tableautorefname{\tablename}%
\def\appendixautorefname{\appendixname}%
\def\chapterautorefname{\chaptername}%
\def\sectionautorefname{Abschnitt}%
\def\subsectionautorefname{Abschnitt}%
\def\subsubsectionautorefname{Abschnitt}%
\def\paragraphautorefname{Abschnitt}%
\def\subparagraphautorefname{Abschnitt}%
\def\equationautorefname{Formel}%
\def\appendixautorefname{\appendixname}
\def\Itemautorefname{\itemname}
\def\Hfootnoteautorefname{\footnotename}
\def\AMSautorefname{\AMSname}
\def\theoremautorefname{\theoremname}
}
%
% Pakete um Latin1 Zeichnens�tze verwenden zu k�nnen und die dazu
%  passenden Schriften.
%
\usepackage[latin1]{inputenc}
\usepackage[T1]{fontenc}

%
% Paket f�r Quotes
%
\usepackage[babel,french=guillemets,german=swiss]{csquotes}

%
% Paket um die Symbole des TS1 Zeichensatzes verwenden zu k�nnen.
%
%\usepackage{textcomp}

%
% Paket zum Erweitern der Tabelleneigenschaften
%
\usepackage{array}

%
% Paket um Grafiken einbetten zu k�nnen
%
\usepackage{graphicx}

%
% Spezielle Schrift verwenden.
%
%\usepackage{goudysans}

%
% Spezielle Schrift im Koma-Script setzen.
%
%\setkomafont{sectioning}{\normalfont\bfseries}
%\setkomafont{captionlabel}{\normalfont\bfseries}
%\setkomafont{pagehead}{\normalfont\itshape}
%\setkomafont{descriptionlabel}{\normalfont\bfseries}

%
% Zeilenumbruch bei Bildbeschreibungen.
%
\setcapindent{1em}

%
% mathematische symbole aus dem AMS Paket.
%
\usepackage{amsmath}
\usepackage{amssymb}

%
% Type 1 Fonts f�r bessere darstellung in PDF verwenden.
%
%\usepackage{mathptmx}           % Times + passende Mathefonts
%\usepackage[scaled=.92]{helvet} % skalierte Helvetica als \sfdefault
%\usepackage{courier}            % Courier als \ttdefault

%
% Paket um Textteile drehen zu k�nnen
%
\usepackage{rotating}

% Package f�r Farben im PDF
\usepackage{color}
\input{rgb}
% Links innerhalb des PDF Dokuments
\definecolor{LinkColor}{rgb}{0,0,0.5}
\usepackage[unicode,
	pdftitle={Xirp - Developer Guide},
	pdfauthor={Rabea Gransberger, Matthias Gernand},
	pdfcreator={MiKTeX, LaTeX with hyperref and KOMA-Script},
	pdfsubject={Das Xirp Entwicklerhandbuch},
	pdfkeywords={Xirp, Robotik, Universit�t, Bremen}]{hyperref}
\hypersetup{colorlinks=true,
	linkcolor=LinkColor,
	citecolor=LinkColor,
	filecolor=LinkColor,
	menucolor=LinkColor,
	%pagecolor=LinkColor,
	urlcolor=LinkColor}

%
% Paket um Listings sauber zu formatieren.
%
\usepackage{listings}
\lstloadlanguages{Java,XML}

%
% ---------------------------------------------------------------------------
% Listing Definationen f�r Java und XML
%
\definecolor{eclipsekeyword}{rgb}{0.5,0,0.0}
\definecolor{eclipsestring}{rgb}{0.16,0,1}
\definecolor{eclipsecomment}{rgb}{0.25,0.5,0.37}

\lstdefinelanguage{Properties}%
  {sensitive=false,%
   moredelim=*[s][\color{eclipsekeyword}]{\{}{\}},
   morecomment=**[l][\ttfamily\color{eclipsestring}]{=},
   morecomment=*[l]{\#},
  }[comments,strings]%

\lstset{
basicstyle=\ttfamily\footnotesize,
breaklines=true, %break long lines
showlines=true,
breakautoindent=true,
postbreak=\space,
numbers=none, %n numbers
numberstyle=\tiny, %numbers in tiny size
numbersep=0pt, %the distance between number and listing.
xleftmargin=0pt, %needed for numbers to be shown
extendedchars=true, %allow german chars
showstringspaces=false, %don't mark spaces
tabsize=2,
frame=tb,
captionpos=b,
showspaces=false,
showstringspaces=false,
backgroundcolor=\color{LightGrey}
}
	
\lstnewenvironment{java}[1][]{\lstset{language=Java,
	tagstyle=\color{darkred},
 	keywordstyle=\color{eclipsekeyword}\bfseries, %red keywords
 	commentstyle=\color{eclipsecomment}, %green comments
 	stringstyle=\color{eclipsestring},
 	usekeywordsintag=false,
 	markfirstintag=false,#1
 }}{}
\lstnewenvironment{xml}[1][]{\lstset{language=XML,
	tagstyle=\color{SeaGreen},
	keywordstyle=\color{DarkBlue},
	commentstyle=\color{DarkRed},
	stringstyle=\color{BlueViolet},
	usekeywordsintag=true,
	markfirstintag=true,#1
	}}{}
\lstnewenvironment{properties}[1][]{\lstset{language=Properties,
	commentstyle=\color{eclipsecomment},
	stringstyle=\color{black},
	usekeywordsintag=true,
	markfirstintag=true,#1
	}}{}
% \newcommand{\Java}[2]{\setJava \begin{lstlisting}[caption=#1,label=#2]}
% \newcommand{\XML}[2]{\setXML \begin{lstlisting}[caption=#1,label=#2]}
% \newcommand{\Properties}[2]{\setProps \begin{lstlisting}[caption=#1,label=#2]}
% ---------------------------------------------------------------------------
%

%
% Neue Umgebungen
% ---------------------------------------------------------------------------
\usepackage[number=none,style=altlist,toc=true]{glossary}
\makeglossary

%
% Index erzeucgen
%
\makeindex

\renewcommand{\lstlistingname}{Auflistung}
\renewcommand{\lstlistlistingname}{Auflistungsverzeichnis}
\renewcommand{\glossaryname}{Glossar}

%\setindexpreamble{Erk�rung zum Index.\par\bigskip} 

\setkomafont{caption}{\sffamily}
\setkomafont{captionlabel}{\sffamily\bfseries}
\setkomafont{footnote}{\sffamily}
\setkomafont{pagehead}{\sffamily}
\setkomafont{pagenumber}{\sffamily\bfseries}

\renewcommand{\glossarypreamble}{\begin{multicols}{2}}
\renewcommand{\glossarypostamble}{\end{multicols}}
%
% EOF
%
