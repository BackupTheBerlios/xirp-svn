\glossary{name={GUI},description={Graphical User Interface: graphische 
Benutzeroberfl�che.},}

\glossary{name={Locale},description={Eine Sprache entsprechend der in
java.util.Locale definierten.},}

\glossary{name={DLL},description={Dynamic Link Library: Native Windows Bibliothek.},}

\glossary{name={SO},description={Shared Object: Native Linux Bibliothek.},}

\glossary{name={Testerbot},description={Ein einfacher Server, welcher
Zufallswerte f�r diverse Sensoren liefert.},}

\glossary{name={SWT},description={Standard Widget Toolkit: Ein Framework zur
Oberfl�chengestaltung wie Swing oder AWT.},}

\glossary{name={JDK},description={Java Development Kit: Enth�lt Bibliotheken,
Compiler und weitere Werkzeuge zur Java-Entwicklung.},}

\glossary{name={Ant},description={Java Werkzeug zum automatischen Erstellen von
Programmen aus Quelltext.},}

\glossary{name={Plugin},description={Ein Programm, welches ein anderes Programm
mit zus�tzlichen Funktionen erweitert.},}

\glossary{name={Item},description={Ein Element einer Liste.},}

\glossary{name={generisch},description={Man bezeichnet etwas als generisch, 
wenn es nicht auf einen speziellen Datentyp angewiesen ist.},}

\glossary{name={Typparameter},description={Wird bei Generics in Java gebraucht.
},}

%\glossary{name={Generics},description={Datentyp mit der M�glichkeit zur Angabe
%von Typparametern},}

\glossary{name={Workspace},description={Arbeitsumgebung.},}

\glossary{name={handeln},description={engl. f�r handhaben oder auch
verarbeiten.},}

\glossary{name={dispose},description={Methode um Programmteile manuell aus dem
Speicher entfernen.},}

\glossary{name={cachen},description={Programmteile im Speicher behalten, um ihn
schneller ausf�hren zu k�nnen.},}

\glossary{name={Framework},description={Anwendungsarchitektur.},}

\glossary{name={Tutorial},description={Eine Anleitung.},}

\glossary{name={Garbage Collector},description={Eine automatische Methode, der
den Speicher von nicht und nicht mehr genutzten Programmteilen befreit.},}

\glossary{name={Swing},description={\textit{siehe} SWT},}

\glossary{name={AWT},description={\textit{siehe} SWT},}

\glossary{name={DTD},description={Document Type Definition: In einer DTD wird
die Reihenfolge die Struktur des XML-Dokuments festgelegt.},} 

\glossary{name={XSD},description={XML Schema Definition, \textit{siehe} DTD},}

\glossary{name={Thread},description={Teil eines Prozesses(Programm), 
Ausf�hrungsreihenfolge eines Programmes},}

\glossary{name={Logging},description={Bezeichnet den Vorgang des automatischen
Aufzeichnens aller oder bestimmter Aktionen von Prozessen auf einem
Computersystem in einer Protokoll-Datei.},}

\glossary{name={Bibliothek},description={Hier Java Bibliotheken (\texttt{jar})
oder native Bibliotheken wie \texttt{dll} f�r Windows oder \texttt{so} f�r
Unix},}

\glossary{name={SWTPlus},description={\textit{siehe} SWT},}

\glossary{name={JAXB},description={Java Architecture for XML Binding: Eine
API in Java, die es erm�glicht, Daten aus XML automatisch an Java-Klassen zu
binden},}

\glossary{name={IP},description={Internet Protokoll, hier eine IP-Adresse zur
eindeutigen Adressierung von Kommunikationspartnern.},}

\glossary{name={Port},description={Adresskomponente einer IP-Adresse. Ist n�tig
um Daten einer Anwendung zuordnen zu k�nnen.},}

\glossary{name={Renderer},description={Ist f�r das Erstellen grafischer 
Elemente zust�ndig.},}

\glossary{name={Hotkey},description={Tastaturk�rzel, Befehle k�nnen Tasten
zugeordnet und somit schneller ausgef�hrt werden.},}

\glossary{name={Gamepad},description={Ein Spielecontroller mit Steuerkreuz und
Tasten, wie man sie von Spielekonsolen kennt.},}

\glossary{name={Joystick},description={Ein Spielecontroller, der dem
Steuerkn�ppel eines Flugzeugs nachempfunden ist.},}

\glossary{name={Systemtray},description={Tray od. Taskbar Notification Area,
ist ein Bereich auf der grafischen Benutzeroberfl�che vieler Desktop
Environments zur Anzeige von Nachrichten.},}

\glossary{name={Profil},description={Ein Profil wird in XML definiert. Es
spezifiziert welche(r) Roboter geladen werden soll und einige weitere
Optionen},} 

%\glossary{name={CSV},description={Comma Separated Values: Ein Datenformat um
%tabellarische Daten in reinem Text ablegen zu k�nnen.},}

\glossary{name={PDF},description={Portable Document Format: Ein Dokumentenformat
zur systemunabh�ngigen Anzeige von Dokumenten.},}

%\glossary{name={Client},description={Ein Computersystem, welches eine Verbindung
%mit einem Server aufnimmt und Nachrichten mit diesem austauscht.},}

%\glossary{name={Server},description={Ein Computersystem, welches Berechnungen
%und Daten f�r seine Clients bereitstellt.},}

\glossary{name={Laserscanner},description={Eigentlich ein
Laser-Entfernungsmesser.},}

\glossary{name={Thermopile},description={Ein Sensor, der W�rmewerte liefert.},}

\glossary{name={HTML},description={Hyper Text Markup Language: Eine Sprache zum
Erstellen von Webseiten.},}

\glossary{name={API},description={Application Programming Interface: Eine
Programmierschnittstelle.},}

\glossary{name={Sensor},description={Bauteil eines Roboters zum Erfassen seiner
Umgebung.},}

\glossary{name={Aktuator},description={Bauteil eines Roboters zum Manipulieren
seiner Umgebung.},}

\glossary{name={XML},description={Extendable Markup Language: Eine
Auszeichnungssprache zur Darstellung hierarchisch strukturierter Daten in Form
von Textdateien},}

\glossary{name={Text-To-Speech},description={Eine Methode, mit der man W�rter
und Texte �ber die Lautsprecher ausgeben kann.},}

%\glossary{name={Shortcut},description={Eine Abk�rzung zum Aufrufen von Teilen
%eines Programms, z.B. CRTL+Q zum Beenden eines Programms.},}

% \glossary{name={Gamepad},description={Ein Ger�t, welches oft zum Steuern vom
% Computerspielen benutzt wird.},}

\glossary{name={I18n},description={Kurzform f�r Internationalisierung},}

%\glossary{name={Drag-and-Drop},description={Eine Methode zum Bewegen grafischer
%Elemente mit Hilfe einer Maus.},}

\glossary{name={Icon},description={Ein kleines Bild.},}

% \glossary{name={TCP/IP},description={Transmission Control Protocol/Internet
% Protocol: Das TCP/IP-Referenzmodell beschreibt den
% Aufbau und das Zusammenwirken von Netzwerkprotokollen der
% Internet-Protokolle.},}
% 
% \glossary{name={SMTP},description={Simple Mail Transfer Protocol, Ein einfaches
% Mailtransportprotokoll.},}
% 
% \glossary{name={MBROLA},description={Ein Programm zur Sprachausgabe.},}

\glossary{name={PNG},description={Portable Network Graphics, Bildstandard.},}

\glossary{name={JPG},description={Auch JPEG genannt, Bildstandard nach der
Joint Photographic Experts Group.},}

% \glossary{name={E-Mail},description={Ein elektronischer Brief.},}
% 
% \glossary{name={Cc},description={Carbon Copy, hier eine Kopie einer E-Mail.},}
