\section{Helferlein}

\xirp~stellt f�r den leichteren Umgang mit \seegls{SWT} ein paar Hilfsklassen bereit:
\begin{itemize}
  \item \index{Manager!ColorManager}\codeQuote{ColorManager} (\longRefSec{sec:swt:dispose})
  \item \index{Manager!FontManager}\codeQuote{FontManager} (\longRefSec{sec:swt:dispose})
  \item \index{Manager!ImageManager}\codeQuote{ImageManager} (\longRefSec{sec:swt:dispose})
  \item \index{SWT!SWTUtil}\codeQuote{SWTUtil} 
  \item \index{SWT!PointUtil}\codeQuote{PointUtil}: Methode zum addieren,
  dividieren etc mit \codeQuote{org.eclipse.swt.graphics.Point}
\end{itemize}

\subsection{SWTUtil}

Klasse mit allgemeinen statischen Hilfsmethoden f�r \seegls{SWT}:
\begin{itemize}
  \item \index{SWT!SWTUtil!asyncExec()}\codeQuote{asyncExec(Runnable)}: \longRefSec{sec:swt:thread}
  \item \index{SWT!SWTUtil!syncExec()}\codeQuote{syncExec(Runnable)}: \longRefSec{sec:swt:thread}
  \item \index{SWT!SWTUtil!centerDialog()}\codeQuote{centerDialog(Shell)}: Zentriert die �bergebene \codeQuote{Shell} auf dem Bildschirm.
  \item \index{SWT!SWTUtil!setGridLayout()}\codeQuote{setGridLayout(Composite,
  int, boolean)}: Setzt ein \codeQuote{GridLayout} f�r das gegebene \codeQuote{Composite}
  \item \index{SWT!SWTUtil!resetMargins()}\codeQuote{resetMargins(GridLayout)}:
  Entfernt R�nder vom gegebenen \codeQuote{GridLayout}
  \item
  \index{SWT!SWTUtil!resetSpacings()}\codeQuote{resetSpacings(GridLayout)}: 
  Entfernt Abst�nde zwischen Zeilen und Spalten vom gegebenen 
  \codeQuote{GridLayout}
  \item \index{SWT!SWTUtil!setGridData()}\codeQuote{setGridData(...)}: Setzt
  die \codeQuote{GridData} f�r ein \codeQuote{Composite}.
  \item \index{SWT!SWTUtil!convertPoint()}\codeQuote{convertPoint(...)}:
  Koordinaten z.B. von Mausklicks werden relativ zum Parent-Widget angegeben und
  m�ssen daher teilweise umgerechnet werden. Hierzu kann diese Methode benutzt
  werden oder aber direkt \codeQuote{toDisplay()} des \codeQuote{Composite}.
  \item
  \index{SWT!SWTUtil!blockDialogFromReturning()}\codeQuote{blockDialogFromReturning(Shell)}: siehe
  unten
  \item \index{SWT!SWTUtil!showBusyWhile()}\codeQuote{showBusyWhile(Shell,
  Runnable)}: Zeigt einen wartenden Mauszeiger solange wie der �bergebene Task l�uft.
  \item \index{SWT!SWTUtil!packTable()}\codeQuote{packTable(Table)}: Ruft
  \codeQuote{pack} f�r alle Tabellenspalten der Tabelle auf.
  \item \index{SWT!SWTUtil!checkStyle()}\codeQuote{checkStyle(int, int)}: \longRefSec{sec:checkstyle}
  \item \index{SWT!SWTUtil!secureDispose()}\codeQuote{secureDispose(...)}:
  Pr�ft mit \codeQuote{swtAssert()} und \codeQuote{SWT!dispose()}\seegls{disposed} den
  Parameter wenn erfolgreich.
  \item \index{SWT!SWTUtil!swtAssert()}\codeQuote{swtAssert(...)}: Pr�ft ob
  der �bergebene Parameter nicht \codeQuote{null} und nicht \seegls{disposed}
  ist.
  \item \index{SWT!SWTUtil!rotate()}\codeQuote{rotate(...)}: Dreht einen
  Punkt um seinen Mittelpunkt in einem gegeben Winkeln (im Bogenma�).
  \item \label{itm:swtutil:openfile}\index{SWT!SWTUtil!openFile()}\codeQuote{openFile(...)}: Versucht die
  angegebene \index{Datei!�ffnen}Datei mit dem im System f�r dessen Endung registrierten Editor zu �ffnen.
\end{itemize}

\subsubsection{blockDialogFromReturning()}
\index{SWT!SWTUtil!blockDialogFromReturning()}
Damit ein erstellter Dialog erste bei einem Tastendruck nicht aber direkt nach
dem �ffnen wieder schlie�t, kann \codeQuote{blockDialogFromReturning()} genutzt
werden.

Im folgenden ein Beispiel welches in Zeile 68
\index{SWT!SWTUtil!centerDialog()}\codeQuote{centerDialog()} und in Zeile 71
\codeQuote{blockDialogFromReturning()} benutzt.

\begin{java}[numbers=left,caption=Ein einfacher Dialog mit R�ckgabe,label=code:swt:util:block]
import org.eclipse.swt.SWT;
import org.eclipse.swt.events.SelectionAdapter;
import org.eclipse.swt.events.SelectionEvent;
import org.eclipse.swt.widgets.Dialog;
import org.eclipse.swt.widgets.Display;
import org.eclipse.swt.widgets.Shell;

import de.unibremen.rr.xirp.ui.util.SWTUtil;
import de.unibremen.rr.xirp.ui.widgets.custom.XButton;
import de.unibremen.rr.xirp.ui.widgets.custom.XLabel;
import de.unibremen.rr.xirp.ui.widgets.custom.XShell;
import de.unibremen.rr.xirp.ui.widgets.custom.XStyledSpinner;
import de.unibremen.rr.xirp.ui.widgets.custom.XButton.XButtonType;

public class SimpleDialog extends Dialog {

	private static final int HEIGHT = 130;
	private static final int WIDTH = 300;
	private XShell dialogShell;
	private int returnValue = -1;

	public SimpleDialog(Shell parent) {
		super(parent);
	}

	public int open() {
		dialogShell = new XShell(getParent( ), SWT.DIALOG_TRIM |
				SWT.APPLICATION_MODAL);
		dialogShell.setText(getText( ));
		dialogShell.setSize(WIDTH, HEIGHT);
		dialogShell.setMinimumSize(WIDTH / 2 + 50, HEIGHT / 2);
		dialogShell.setText("SimpleDialog");

		SWTUtil.setGridLayout(dialogShell, 2, true);

		XLabel label = new XLabel(dialogShell, SWT.NONE);
		label.setText("Zahl eingeben");
		SWTUtil.setGridData(label, true, true, SWT.BEGINNING, SWT.CENTER, 1, 1);

		final XStyledSpinner theta = new XStyledSpinner(dialogShell, SWT.NONE);
		SWTUtil.setGridData(theta, true, true, SWT.FILL, SWT.CENTER, 1, 1);

		XButton ok = new XButton(dialogShell, XButtonType.OK);
		SWTUtil.setGridData(ok, true, false, SWT.FILL, SWT.CENTER, 1, 1);
		ok.addSelectionListener(new SelectionAdapter( ) {

			@Override
			public void widgetSelected(SelectionEvent e) {

				returnValue = theta.getSelection( );
				dialogShell.close( );
			}
		});

		XButton cancel = new XButton(dialogShell, XButtonType.CANCEL);
		SWTUtil.setGridData(cancel, true, false, SWT.FILL, SWT.CENTER, 1, 1);
		cancel.addSelectionListener(new SelectionAdapter( ) {

			@Override
			public void widgetSelected(SelectionEvent e) {
				dialogShell.close( );
			}
		});

		dialogShell.setDefaultButton(ok);
		dialogShell.pack( );
		dialogShell.layout( );
		SWTUtil.centerDialog(dialogShell);
		dialogShell.open( );

		SWTUtil.blockDialogFromReturning(dialogShell);

		return returnValue;
	}

	public static void main(String[] args) {
		final Display display = new Display( );
		final Shell shell = new Shell(display);
		shell.pack( );
		shell.open( );

		SimpleDialog dialog = new SimpleDialog(shell);
		int value = dialog.open( );

		System.out.println(value);

		while (!shell.isDisposed( )) {
			if (!display.readAndDispatch( )) {
				display.sleep( );
			}
		}
		display.dispose( );
	}
}
\end{java}
\lstset{
numbers=none,
}

\subsubsection{checkStyle()}
\label{sec:checkstyle}
\index{SWT!SWTUtil!checkStyle()}

Mit \codeQuote{checkStyle()} l�sst sich �berpr�fen ob ein bestimmter Stil z.B.
von \seegls{SWT} gesetzt ist.

Im folgenden Beispiel ist dies gezeigt:

\begin{java}[caption=Pr�fen ob Stil gesetzt ist, label=code:swt:util:checkstyle]
public static void main(String[] args) {
	int combinedStyle = SWT.APPLICATION_MODAL | SWT.DIALOG_TRIM;
	boolean applicationModal = SWTUtil.checkStyle(combinedStyle,
			SWT.APPLICATION_MODAL);
	boolean systemModal = SWTUtil.checkStyle(combinedStyle,
			SWT.SYSTEM_MODAL);
	System.out.println("Modalit�t");
	System.out.println("Applikation: " + applicationModal);
	System.out.println("System: " + systemModal);
}
\end{java}