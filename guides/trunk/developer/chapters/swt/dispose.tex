\section{dispose}
\label{sec:swt:dispose}
\seegls{SWT} benutzt zur Darstellung der Oberfl�chenelemente 
(\index{SWT!Widget|see {Widget}}Widgets) nach M�glichkeit die nativen
vom aktuellen Betriebssystem zur Verf�gung gestellten Elemente. Im Gegensatz zur
normalen Programmierung mit Java k�nnen diese Widgets nicht vom \seegls{Garbage
Collector} freigegeben werden, da dies vom Betriebssystem durchgef�hrt werden
muss. Daher hat jedes Element von \seegls{SWT} welches vom Betriebssystem bereitgestellt
wird eine Methode \index{SWT!dispose()}\codeQuote{dispose()} zum Freigeben.

Normalerweise muss diese Methode nicht programmatisch aufgerufen werden, da wenn
ein Widget \seegls{disposed} wird, automatisch auch die darauf liegenden
Kindelemente freigegeben werden und das \codeQuote{dispose()} einer
\codeQuote{Shell} beispielsweise durch das Schlie�en dieser aufgerufen wird.

Eine Ausnahme davon bilden die Klassen, die von
\codeQuote{org.eclipse.swt.graphics.Resource} erben. Werden diese von der
Applikation erstellt, m�ssen sie auch wieder von dieser \seegls{disposed}
werden, da sonst Speicherlecks entstehen w�rden.

\xirp~bietet daher f�r \index{Manager!ColorManager}Farben,
\index{Manager!FontManager}Schriften und \index{Manager!ImageManager}Bilder \codeQuote{Manager} an,
bei deren Benutzung das \seegls{disposen} entf�llt. Sie liegen alle im Paket
\codeQuote{de.unibremen.rr.xirp.ui.util.ressource} und hei�en:
\begin{itemize}
  \item \codeQuote{ColorManager}
  \item \codeQuote{FontManager}
  \item \codeQuote{ImageManager}
\end{itemize}