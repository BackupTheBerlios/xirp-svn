\section{Thread}
\label{sec:swt:thread}
\seegls{SWT} hat nur einen \index{SWT!Thread}Thread, welchem der Zugriff auf die \seegls{SWT}
Oberfl�chen Elemente erlaubt ist. Alle anderen Threads bekommen bei einem
Zugriff eine \codeQuote{SWTException} mit der Meldung \enquote{Invalid thread
access}. Um aus anderen Threads heraus Oberfl�chenelemente manipulieren zu k�nnen muss der Code, der auf ein Oberfl�chenelement
zugreift, mit einem \index{SWT!SWTUtil!asyncExec()}\codeQuote{SWTUtil.asyncExec()} 
oder \index{SWT!SWTUtil!syncExec()}\codeQuote{SWTUtil.syncExec()} gekapselt werden.
Der auszuf�hrende Code wird dabei in einem \codeQuote{Runnable} �bergeben und in
die \codeQuote{Queue} des \seegls{SWT}-Thread zur Ausf�hrung eingeh�ngt.

W�hrend \index{SWT!SWTUtil!syncExec()}\codeQuote{SWTUtil.syncExec()} wartet bis der
Code tats�chlich ausgef�hrt wurde, kehrt der Aufruf von
\index{SWT!SWTUtil!asyncExec()}\codeQuote{SWTUtil.asyncExec()} sofort zur�ck, w�hrend
der �bergebene Code erst in der Zukunft ausgef�hrt wird.

Ein Beispiel ist in \autoref{sec:sensor} auf \autopageref{sec:sensor} zu finden.

Wichtig ist den Code, der an \codeQuote{SWTUtil.asyncExec()} oder
\codeQuote{SWTUtil.syncExec()} �bergeben wird, so minimal und wenig zeitaufwendig
wie m�glich zu halten, da l�nger dauernde Codest�cke die ganze Oberfl�che lahm
legen k�nnen (siehe auch \autoref{sub:swtcalc} auf \autopageref{sub:swtcalc}).