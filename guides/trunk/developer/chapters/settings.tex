\chapter{Einstellungen}
\label{cha:settings}
\index{Einstellungen}

\seegls{Plugins} k�nnen Einstellungen haben, um zum Beispiel \seegls{IP} und
\seegls{Port} eines \seegls{Servers} einstellen zu k�nnen.

Ein \seegls{Plugin} stellt seine Einstellungen durch das �berschreiben von
\codeQuote{getSettings()} bereit. Ein kleine Beispiel hierzu findet sich in
\autoref{sec:plugin:settings} ab \autopageref{sec:plugin:settings}.

In diesem Kapitel sollen alle weiteren Funktionen der Einstellungen beschrieben 
werden.

Einstellungen k�nnen verschiedene Seiten haben, auf denen sich jeweils Optionen
befinden. Jede Option kann mehrere Werte haben. Die Einstellungen sind komplett
\index{Internationalisierung}internationalisierbar. Es k�nnen weiterhin
Standardwerte zugewiesen werden, so dass es m�glich ist Standardeinstellungen
wiederherzustellen. Weiterhin k�nnen Einstellungen die noch nicht gespeichert
sind r�ckg�ngig gemacht werden.

Die Darstellung an der Oberfl�che folgt durch verschiedene \seegls{Renderer}.

Alle Klassen die f�r die Erstellung von Einstellungen ben�tigt werden finden
sich im Paket \codeQuote{de.unibremen.rr.xirp.settings}.

\section{Erstellung}
\index{Einstellungen}
Mit der Klasse \codeQuote{Settings} beginnt die Erstellung von Einstellungen, 
da man zun�chst eine Instanz dieser Klasse ben�tigt. Sie ist die Basisklasse 
f�r alle Einstellungen. Gleichzeitig wird hier durch die Wahl des Konstruktors 
auch festgelegt ob die Einstellungen automatisch von \xirp~in eine Datei 
persistiert werden sollen, oder die 
\index{Einstellungen!Persistierung}Persistierung selber �bernommen wird.

Sollen die Einstellungen von \xirp~automatisch persistiert werden, so muss 
einer der Konstruktoren \codeQuote{Settings(PropertiesConfiguration, 
II18nHandler, String, String)} gew�hlt werden.

Das �bergebene \codeQuote{PropertiesConfiguration}-Objekt\footnote{Die
Erstellung von \codeQuote{PropertiesConfiguration}-Objekten ist in der
Dokumentation zu Commons Configuration unter
\href{\jakartacommonsurl}{\jakartacommonsurl} zu finden.} enth�lt dabei
die Informationen in welche Datei die Einstellungen gespeichert werden sollen.
Als Standard hierf�r kann \index{Manager!PluginManager}\codeQuote{PluginManager.getProperties( )}
�bergeben werden.

Danach folgen der \index{Internationalisierung!II18nHandler} 
Internationalisierungshandler (im \seegls{Plugin} als Feld \codeQuote{handler} 
verf�gbar) und zwei Schl�ssel. Der Erste ist die Basis f�r die 
\index{Einstellungen!Internationalisierung}�bersetzungen, der zweite wird f�r 
die eindeutige Zuordnung zu einem \seegls{Plugin} bei der 
\index{Einstellungen!Persistierung}Persistierung ben�tigt.

Der Erste Schl�ssel (Hauptschl�ssel) kann frei gew�hlt werden, f�r den Zweiten
muss \newline\codeQuote{getRealKey()} benutzt werden.

\begin{java}[caption=Erstellung eines Settings-Objekt]
Settings settings = new Settings(PluginManager.getProperties( ),
		handler,
		"MyPlugin.settings.main",
		getRealKey( ));
\end{java}

\subsection{Seiten}

Diesem Objekt k�nnen nun mit \codeQuote{addPage(String, String, String)} \index{Einstellungen!Seite}Seiten
hinzugef�gt werden.

Die drei Strings sind dabei alle Schl�ssel f�r die \index{Einstellungen!Internationalisierung}�bersetzungen: Ein
Seitenschl�ssel f�r die �berschrift dieser Seite, einer f�r eine Kurzbeschreibung der
Seite und einer f�r eine Langbeschreibung der Seite.

Kurz- und Langbeschreibung m�ssen nicht gesetzt werden, k�nnen also
\codeQuote{null} sein.

\begin{java}[caption=Erstellung einer Einstellungsseite]
Settings settings = new Settings(PluginManager.getProperties( ),
		handler,
		"Example.settings.main",
		getRealKey( ));
SettingsPage settingsPage = settings.addPage("page",
		"shortdescription",
		"description");
\end{java}

Die Schl�ssel f�r die \index{Einstellungen!Internationalisierung}�bersetzungen sehen dann so aus:
\begin{itemize}
  \item Einstellungsbeschreibung: <hauptschl�ssel>
  \item Seiten�berschrift: <hauptschl�ssel><seitenschl�ssel>
  \item Kurzbeschreibung Seite: <hauptschl�ssel><seitenschl�ssel><kurzschl�ssel>
  \item Langbeschreibung Seite: <hauptschl�ssel><seitenschl�ssel><langschl�ssel>
\end{itemize}

In diesem Fall also:
\begin{properties}
Example.settings.main=Beispiel
Example.settings.main.page=Erste Seite
Example.settings.main.page.description=Lange Beschreibung
Example.settings.main.page.shortdescription=Kurze Beschreibung
\end{properties}

\subsection{Optionen}

Der Seite k�nnen dann einzelne \index{Einstellungen!Optionen}Optionen mit \codeQuote{addOption(String,
OptionType)} hinzugef�gt werden. Das erste Argument ist dabei der eindeutige
Optionsschl�ssel, das zweite Argument der \index{Einstellungen!Optionen!Optionstyp}Optionstyp. Zur Verf�gung stehende
Optionstypen sind:
\begin{itemize}
  \item \codeQuote{CHECKBOX}: Werte werden als \index{Widget!Checkbox}Checkboxen in einer Gruppe angezeigt.
  \item \codeQuote{RADIOBUTTON}: Werte werden als \index{Widget!Radiobox}Radiobox in einer Gruppe angezeigt.
  \item \codeQuote{COMBOBOX}: Werte werden in einer \index{Widget!Combobox}Combobox dargestellt.
  \item \codeQuote{TEXTFIELD}: Der letzte Wert wird in einem \index{Widget!Textfield}Textfeld f�r
  manuellen Input dargestellt. Dies kann auch ein Passwortfeld sein.
  \item \codeQuote{COLOR}: Letzter Wert wird in Farbauswahl dargestellt.
  \item \codeQuote{SPINNER}: Letzter Wert wird als \index{Widget!Spinner}Spinner dargestellt.
  \item \codeQuote{UNKNOWN}: Unbekannt. Es muss ein \index{Einstellungen!Renderer}\seegls{Renderer} f�r die Darstellung
  gesetzt werden.
\end{itemize}

Als Beispiel also:
\begin{java}[caption=Erstellung einer Option]
Settings settings = new Settings(PluginManager.getProperties( ),
		handler,
		"Example.settings.main",
		getRealKey( ));
SettingsPage settingsPage = settings.addPage("page",
		"shortdescription",
		"description");
Option optionString = settingsPage.addOption("option",
		Option.OptionType.TEXTFIELD);
\end{java}

\subsection{Werte}

Einer \index{Einstellungen!Optionen}Option k�nnen dann mehrere \index{Einstellungen!Werte} Werte zugewiesen
werden:

\begin{itemize}
\item \codeQuote{addTranslatableNamedBooleanValue(String, SettingsState,
Object...)}:  F�gt einen \\�bersetzbaren String der Option hinzu. Das erste
Argument ist der Schl�ssel f�r die �bersetzung. Der zweite Parameter gibt den Standardzustand des
Wertes an: Selektiert oder nicht selektiert. Der dritte Parameter ist optional
und kann Argumente f�r die Ersetzung von Variablen in der �bersetzung enthalten.
\item \codeQuote{addNonTranslatableNamedBooleanValue(String, SettingsState)}:
F�gt den gegebenen \\String der Option hinzu. Dieser String wird nicht �bersetzt
sondern direkt angezeigt. Der zweite Parameter gibt den Standardzustand des
Wertes an: Selektiert oder nicht selektiert.
\item \codeQuote{addNumberValue(Number, SettingsState)}: F�gt die gegebene Zahl
der Option hinzu. Der zweite Parameter gibt den Standardzustand des
Wertes an: Selektiert oder nicht selektiert.
\item \codeQuote{addRGBValue(String, int, int, int)}: F�gt einen Farbwert einer
Option vom Typ \codeQuote{COLOR} oder \codeQuote{UNKNOWN} hinzu. Der erste
Parameter gibt dabei den Schl�ssel an unter welchem der Wert gespeichert wird.
Dieser Schl�ssel muss f�r diese Option eindeutig sein. Die n�chsten drei
Parameter definieren den Standardfarbwert in RGB und stehen in dieser
Reihenfolge f�r Rot, Gr�n, Blau im Bereich von [0;255]. Da f�r den Optionstyp
\codeQuote{COLOR} nur der ein Wert dargestellt wird, wird hier die Beschreibung
von der Option als Beschreibung des Wertes genutzt.
\item \codeQuote{addRGBValue(String, RGB)}: Wie \codeQuote{addRGBValue(String,
int, int, int)} nur mit einem \codeQuote{RGB}-Objekt f�r den Standardfarbwert.
\item \codeQuote{addSpinnerValue(double, double, double, int)}: Wenn der
Optionstyp \codeQuote{SPINNER} oder \codeQuote{UNKNOWN} ist wird der erste
Parameter als Standardwert, der zweite als Minimum, der dritte als Maximum und
der vierte als Schrittweite interpretiert. Der Wert wird mit einem
\codeQuote{StyledSpinner} mit dem Stil \codeQuote{DOUBLE} und wenn das Minimum
kleiner 0 ist mit dem Stil \codeQuote{NEGATIVE} (siehe
\autoref{sec:XStyledSpinner} auf \autopageref{sec:XStyledSpinner}) dargestellt.
\item \codeQuote{addSpinnerValue(int, int, int, int)}: Genau wie
\codeQuote{addSpinnerValue(double, double, double, int)} jedoch nicht mit
\codeQuote{DOUBLE} Stil.
\item \codeQuote{addStringValue(String, String)}: Wenn der
Optionstyp \codeQuote{TEXTFIELD} oder \codeQuote{UNKNOWN} ist wird der im
zweiten Argument gegebene Text als Standardwert f�r ein Textfeld hinzugef�gt.
Die Werte von Textfeldern k�nnen nicht �bersetzt werden, da ein freier Text
eingegeben werden darf. Der erste Parameter ist der Schl�ssel f�r die
Speicherung welcher innerhalb der Option eindeutig sein muss.
\item \codeQuote{addStringValue(String, String, int)}: Genau wie
\codeQuote{addStringValue(String, String)} nur ist der dritte Parameter die
maximale Anzahl an eingebbaren Zeichen f�r das Textfeld.
\item \codeQuote{addPasswordValue(String, String)}: Genau wie
\codeQuote{addStringValue(String, String)} nur wird in der Darstellung der
eingegebenen Text nur mit Sternen gekennzeichnet.
\item \codeQuote{addValue(IValue)}: F�gt einen beliebigen selbst definierten
Wert hinzu. Siehe \autoref{sec:settings:ownvalues} auf \autopageref{sec:settings:ownvalues}.
\end{itemize}

\subsection{Beispiele}

Mit der Methode \codeQuote{getExampleSettings()} l�sst sich ein Beispiel f�r
Einstellungen erstellen, welches f�r jeden vorhandenen \index{Einstellungen!Optionen!Optionstyp}Optionstyp eine neue
Seite hat, auf welcher jeder m�gliche Wert einmal hinzugef�gt wird.
\begin{java}
Settings settings = Settings.getExampleSettings( );
\end{java}

In den Abbildungen \ref{img:settings:checkpage} bis
\ref{img:settings:spinnerpage} sind die entstehenden Seiten zu sehen.

\kfig{settings_checkpage}{1}{Einstellungen: Optionstyp=CHECKBOX}{img:settings:checkpage}
\kfig{settings_radiopage}{1}{Einstellungen: Optionstyp=RADIOBUTTON}{img:settings:radiopage}
\kfig{settings_combopage}{1}{Einstellungen: 
Optionstyp=COMBOBOX}{img:settings:combopage} 
\kfig{settings_colorpage}{1}{Einstellungen:
Optionstyp=COLOR}{img:settings:colorpage} 
\kfig{settings_textfieldpage}{1}{Einstellungen:
Optionstyp=TEXTFIELD}{img:settings:textfieldpage}
\kfig{settings_spinnerpage}{1}{Einstellungen:
Optionstyp=SPINNER}{img:settings:spinnerpage}

\clearpage

Nun noch ein Codebeispiele zur Erzeugung einer Seite mit verschiedenen
\index{Einstellungen!Optionen!Optionstyp}Optionstypen und \index{Einstellungen!Werte}Werten:
\begin{java}[caption=Beispieleinstellungen,label=code:settings:example]
@Override
public Settings getSettings() {
	if (settings == null) {
		settings = new Settings(PluginManager.getProperties( ),
				handler,
				"Example.settings.main",
				getRealKey( ));

		SettingsPage settingsPage = settings.addPage("page",
				"shortdescription",
				"description");

		Option checkbox = settingsPage.addOption("checkbox",
				OptionType.CHECKBOX);
		checkbox.addTranslatableNamedBooleanValue("value1",
				SettingsState.SELECTED,
				"Ein Argument");
		checkbox.addNumberValue(20.5, SettingsState.NOT_SELECTED);

		Option radiobutton = settingsPage.addOption("radiobutton",
				OptionType.RADIOBUTTON);
		radiobutton.addNonTranslatableNamedBooleanValue("Nicht �bersetzbarer Wert",
				SettingsState.SELECTED);
		radiobutton.addNumberValue(20.5, SettingsState.NOT_SELECTED);

		Option combo = settingsPage.addOption("combobox",
				OptionType.COMBOBOX);
		combo.addNonTranslatableNamedBooleanValue("Nicht �bersetzbarer Wert",
				SettingsState.SELECTED);
		combo.addNumberValue(20.5, SettingsState.NOT_SELECTED);

		Option color = settingsPage.addOption("color", OptionType.COLOR);
		color.addRGBValue("colorvalue", 255, 0, 0);

		Option spinner = settingsPage.addOption("spinner",
				OptionType.SPINNER);
		spinner.addSpinnerValue(20.5, -10, 20, 1);

		Option spinnerInt = settingsPage.addOption("spinnerint",
				OptionType.SPINNER);
		spinnerInt.addSpinnerValue(15, -10, 20, 1);

		Option text = settingsPage.addOption("text", OptionType.TEXTFIELD);
		text.addStringValue("textkey", "Eingabefeld");

		Option pass = settingsPage.addOption("pass", OptionType.TEXTFIELD);
		pass.addPasswordValue("textkey", "Passwort");

	}
	return settings;
}
\end{java}

Die zugeh�rigen \index{Einstellungen!Optionen}�bersetzungen:
\begin{properties}
Example.settings.main=Testeinstellungen
Example.settings.main.page=Testeinstellungen
Example.settings.main.page.shortdescription=Kurze Beschreibung
Example.settings.main.page.description=Lange Beschreibung
Example.settings.main.page.checkbox=Checkboxoption
Example.settings.main.page.checkbox.value1=�bersetzter String: {0}
Example.settings.main.page.radiobutton=Radiobuttonoption
Example.settings.main.page.combobox=Combooption
Example.settings.main.page.color=Farboption
Example.settings.main.page.spinner=Spinneroption
Example.settings.main.page.spinnerint=Spinneroption (int)
Example.settings.main.page.text=Textoption
Example.settings.main.page.pass=Passwordoption
\end{properties}

\kfig{settings_examplepage}{1}{Einstellungsseite mit Beispielen}{img:settings:examplepage}

Diese Beispieleinstellungen sind in \autoref{img:settings:examplepage} zu sehen.
\section{Eigene Werte}
\index{Einstellungen!Werte!eigene}
\label{sec:settings:ownvalues}
Die Methode \codeQuote{addValue(IValue)} bietet die M�glichkeit zu einer Option\index{Einstellungen!Optionen}
auch selbst definierte Werte hinzuzuf�gen.

Um einen eigenen \index{Einstellungen!Wert}Wert zu definieren muss daher die Klasse \codeQuote{IValue}
implementiert, oder eine der schon bestehenden Werte erweitert werden.

Als Beispiel soll ein Wert implementiert werden, welcher Vor- und Nachname einer
Person enthalten kann.

Dazu muss zun�chst die Klasse \codeQuote{IValue} implementiert werden, welche
zus�tzlich \codeQuote{Observeable} erweitern muss, damit sp�ter eine
Synchronisation mit der anzeigenden Oberfl�che m�glich wird. Weiterhin erh�lt
die Klasse noch eine Klasse \codeQuote{PersonName} welche Name und Vorname
kapselt. 

\begin{java}[caption=Klasse PersonName]
public class PersonName {

	private String forename;
	private String surename;

	public PersonName(String forename, String surename) {
		this.forename = forename;
		this.surename = surename;
	}

	public String getForename() {
		return forename;
	}

	public String getSurename() {
		return surename;
	}

	public static PersonName parse(String string) {
		final String[] split = string.split(" ");
		String forename = "";
		String surename = "";
		if (split.length > 0) {
			forename = split[0];
			if (split.length > 1) {
				surename = split[1];
			}
		}

		return new PersonName(forename, surename);
	}

	@Override
	public String toString() {
		return forename + " " + surename;
	}
}
\end{java}

Die Klasse \codeQuote{PersonName} enth�lt bereits eine Methode welche einen mit
\codeQuote{toString()} generierten String wieder zur�ck in ein Objekt parsen
kann. Dies wird ben�tigt um die gespeicherten Einstellungen wieder auszulesen.

\begin{java}[caption=Eigener Wert f�r Einstellungen: PersonNameValue, label=code:settings:ownvalue]
import java.util.ArrayList;
import java.util.Observable;
import java.util.Observer;

import de.unibremen.rr.xirp.settings.IValue;
import de.unibremen.rr.xirp.settings.SettingsChangedEvent;
import de.unibremen.rr.xirp.settings.SettingsChangedListener;

public class PersonNameValue extends Observable implements IValue {

	private PersonName defaultValue;
	private PersonName currentValue;
	private PersonName savedValue;
	private String saveKey;
	private ArrayList<SettingsChangedListener> listeners = new ArrayList<SettingsChangedListener>( );

	public PersonNameValue(String key, String forename, String surename) {
		this.saveKey = key;
		this.defaultValue = new PersonName(forename, surename);
		this.currentValue = defaultValue;
		this.savedValue = currentValue;
	}

	@Override
	public void addObserverToValue(Observer obs) {
		this.addObserver(obs);
	}
	
	...
} 
\end{java}

Da die Klasse \codeQuote{Observable} erweitert kann der \codeQuote{Observer} der
Methode \codeQuote{addObserverToValue()} direkt zur Klasse hinzugef�gt werden.
Diese Methode wird von den \index{Einstellungen!Renderer}\seegls{Renderern} aufgerufen um sich auf �nderungen des
Objektes zu registrieren.

Nun folgt die Implementierung aller weiteren Methoden die vom Interface
bereitgestellt werden:
\begin{java}[caption=Eigener Wert f�r Einstellungen: PersonNameValue(2), label=code:settings:ownvalue2] 
@Override public String getDisplayValue() {
	return currentValue.toString( );
}

public PersonName getDisplayPerson() {
	return currentValue;
}

public void setCurrentPerson(PersonName person) {
	setCurrentPerson(person, false);
}

public void setCurrentPerson(PersonName person, boolean fromUI) {
	this.currentValue = person;

	// notify the observers that the current state has changed
	// and they should update the UI
	if (!fromUI) {
		setChanged( );
		notifyObservers( );
	}

	notifyChange( );
}

@Override
public String getSaveKey() {
	return saveKey;
}

@Override
public String getSaveValue() {
	return savedValue.toString( );
}

@Override
public void parseSavedValue(Object object) {
	setSavedValue(PersonName.parse(object.toString( )));
}

public void setSavedValue(PersonName person) {
	this.savedValue = person;
	setCurrentPerson(savedValue);
}

@Override
public boolean hasChanged() {
	return !currentValue.equals(savedValue);
}

@Override
public boolean isDefaultSelected() {
	return currentValue.equals(defaultValue);
}

@Override
public void reset() {
	setCurrentPerson(savedValue);
}

@Override
public void save() {
	this.savedValue = currentValue;
}

@Override
public void setToDefault() {
	setCurrentPerson(defaultValue);
}

@Override
public void setSelected(boolean selection) {
	setSelected(selection, false);
}

public void addLocaleChangeListener(SettingsChangedListener listener) {
	listeners.add(listener);
}

private void notifyChange() {
	for (SettingsChangedListener listener : listeners) {
		listener.settingsChanged(new SettingsChangedEvent(this));
	}
}

public void removeLocaleChangeListener(SettingsChangedListener listener) {
	listeners.remove(listener);
}

//Die weiteren Methoden treffen hier nicht zu

@Override
public void setSelected(boolean selection, boolean fromUI) {
	// Nothing to do
}

@Override
public boolean isSelected() {
	return true;
}

@Override
public boolean isCurrentlySelected() {
	return true;
}

@Override
public String getKey() {
	return null;
}

@Override
public Object[] getKeyArgs() {
	return null;
}
\end{java}

Die Methoden \codeQuote{parseSavedValue()} wird von der Optionsklasse aufgerufen
und �bergibt den \index{Einstellungen!Werte}Wert welcher einmal durch \codeQuote{getSaveValue()}
gespeichert wurde.

Die Methode \codeQuote{setCurrentPerson()} kommt zweimal vor. Dies liegt daran
dass ein Aufruf sowohl von der Oberfl�che als auch von innerhalb erfolgen kann
und dies unterschieden werden muss. Erfolgt der Aufruf nicht von der Oberfl�che
wird diese Beispielsweise �ber \codeQuote{notifyObservers( )} �ber die �nderung
informiert, da sie sich vorher mit \codeQuote{addOberserverToValue( )}
angemeldet hat.

Der Wert kann nun benutzt werden und wird im folgenden Beispiel einer
Checkbox-Option hinzugef�gt.
\index{Manager!PluginManager}
\begin{java}[caption=Benutzung von PersonNameValue]
@Override
public Settings getSettings() {
	if (settings == null) {
		settings = new Settings(PluginManager.getProperties( ),
				handler,
				"Example.settings.main",
				getRealKey( ));

		SettingsPage settingsPage = settings.addPage("page",
				"shortdescription",
				"description");

		Option check = settingsPage.addOption("personcheck",
				OptionType.CHECKBOX);
		PersonNameValue value = new PersonNameValue("personkey",
				"Max",
				"Musterman");
		check.addValue(value);
		check.addNumberValue(20.5, SettingsState.NOT_SELECTED);

	}
	return settings;
}
\end{java}
\section{Eigener Renderer}
\index{Einstellungen!Renderer!eigener}
Um die \index{Einstellungen!Werte}Werte von \codeQuote{PersonNameValue} (siehe
\autoref{code:settings:ownvalue} auf \autopageref{code:settings:ownvalue}) so 
anzuzeigen,
dass der Name auch ge�ndert werden kann ist ein eigener Renderer n�tig.

Eigene \seegls{Renderer} m�ssen \codeQuote{IOptionRenderer} implementieren. Diese bietet
auch die M�glichkeit eingegebenen Strings auf \index{Einstellungen!Validierung}Korrektheit (siehe
\autoref{sec:settings:valid} auf \autopageref{sec:settings:valid}) zu pr�fen. In der
abstrakten Implementierung \codeQuote{AbstractOptionRenderer} ist die Behandlung
eines Validierers schon implementiert. Daher wird hier auf die Erweiterung von
\codeQuote{AbstractOptionRenderer} zur�ckgegriffen:

\begin{java}[caption=Eigener Renderer f�r Einstellungen,label=code:settings:renderer]
import java.util.Observable;

import org.apache.log4j.Logger;
import org.eclipse.swt.SWT;
import org.eclipse.swt.events.ModifyEvent;
import org.eclipse.swt.events.ModifyListener;
import org.eclipse.swt.layout.GridLayout;
import org.eclipse.swt.widgets.Composite;

import de.unibremen.rr.xirp.settings.IValue;
import de.unibremen.rr.xirp.settings.Option;
import de.unibremen.rr.xirp.ui.util.SWTUtil;
import de.unibremen.rr.xirp.ui.widgets.custom.XComposite;
import de.unibremen.rr.xirp.ui.widgets.custom.XLabel;
import de.unibremen.rr.xirp.ui.widgets.custom.XText;
import de.unibremen.rr.xirp.ui.widgets.dialogs.preferences.renderer.AbstractOptionRenderer;
import de.unibremen.rr.xirp.util.Constants;

public class PersonNameRenderer extends AbstractOptionRenderer {

	private static Logger logClass = Logger.getLogger(PersonNameRenderer.class);

	private XText forename;
	private XText surename;

	public void render(Composite parent, Option option) {
		// parent ist nur zweispaltig, daher dreispaltiges Composite
		// selbst erstellen
		XComposite optionComp = new XComposite(parent, SWT.NONE);
		final GridLayout gridLayout = SWTUtil.setGridLayout(optionComp, 3, true);
		SWTUtil.resetMargins(gridLayout);
		SWTUtil.setGridData(optionComp, true, false, SWT.FILL, SWT.TOP, 2, 1);

		XLabel l = new XLabel(optionComp, SWT.NONE, option.getI18n( ));
		l.setTextForLocaleKey(option.getNameKey( ), option.getNameKeyArgs( ));
		SWTUtil.setGridData(l, true, false, SWT.LEFT, SWT.TOP, 1, 1);

		// Nur ein Wert kann dargestellt werden
		PersonNameValue last = null;
		for (IValue value : option.getValues( )) {
			if (value instanceof PersonNameValue) {
				last = (PersonNameValue) value;
			}
		}

		// Textfelder f�r Vor- und Nachname
		forename = new XText(optionComp, SWT.BORDER, true);
		SWTUtil.setGridData(forename, true, false, SWT.FILL, SWT.TOP, 1, 1);
		surename = new XText(optionComp, SWT.BORDER, true);
		SWTUtil.setGridData(surename, true, false, SWT.FILL, SWT.TOP, 1, 1);

		if (last != null) {
			final PersonNameValue currentSelection = last;
			// Observer auf den Wert hinzuf�gen
			currentSelection.addObserverToValue(this);

			// Wert darstellen
			final PersonName displayPerson = currentSelection.getDisplayPerson( );
			forename.setText(displayPerson.getForename( ));
			surename.setText(displayPerson.getSurename( ));

			// �nderungen in der Oberfl�che an den Wert �bertragen
			final ModifyListener modifyListener = new ModifyListener( ) {

				public void modifyText(@SuppressWarnings("unused")
				ModifyEvent e) {
					PersonName person = new PersonName(forename.getText( ),
							surename.getText( ));
					currentSelection.setCurrentPerson(person, true);
				}

			};
			forename.addModifyListener(modifyListener);
			surename.addModifyListener(modifyListener);

		}
		else {
			forename.setEnabled(false);
			surename.setEnabled(false);
			logClass.warn("Der PersonNameRenderer kann nur Werte vom Typ PersonNameValue darstellen." +
					Constants.LINE_SEPARATOR);
		}
	}

	public void update(Observable observable, @SuppressWarnings("unused")
	Object obj) {
		// �nderungen des Wertes in der Oberfl�che darstellen
		// z.B. bei Reset
		if (observable instanceof PersonNameValue) {
			PersonNameValue val = (PersonNameValue) observable;
			final PersonName displayPerson = val.getDisplayPerson( );
			this.forename.setText(displayPerson.getForename( ));
			this.surename.setText(displayPerson.getSurename( ));
		}
	}
}
\end{java}

\kfig{settings_renderer}{1}{Ein eigener Renderer f�r Einstellungen}{img:settings:renderer}

Es ist nur m�glich einen \index{Einstellungen!Werte}Wert f�r diese
\index{Einstellungen!Optionen}Option darzustellen. Dabei wird der Name 
der Person in zwei Textfeldern zur Eingabe bereitgestellt.

Nutzt man den \seegls{Renderer} f�r die Darstellung erh�lt man die Ansicht aus
\autoref{img:settings:renderer}: 
\index{Manager!PluginManager}
\begin{java}[caption=Den eigenen Renderer zur Darstellung nutzen,label=code:renderer:use]
@Override
public Settings getSettings() {
	if (settings == null) {
		settings = new Settings(PluginManager.getProperties( ),
				handler,
				"Example.settings.main",
				getRealKey( ));

		SettingsPage settingsPage = settings.addPage("page",
				"shortdescription",
				"description");

		Option option = settingsPage.addOption("person", OptionType.UNKNOWN);
		PersonNameValue value = new PersonNameValue("personkey",
				"Max",
				"Musterman");
		option.addValue(value);
		option.setRenderer(new PersonNameRenderer( ));
	}
	return settings;
}
\end{java}

\section{Validierung}
\label{sec:settings:valid}
\index{Einstellungen!Validierung}
Bei allen Darstellungsmethoden von Einstellungen ist es m�glich einen 
\codeQuote{Validator} zur Validierung von Eingaben hinzuzuf�gen. Sinn macht 
dies daher nur bei einem \seegls{Renderer} der auch tats�chlich M�glichkeiten 
zur Eingabe bereitstellt, wie z.B. der zuvor erstellte 
\newline\codeQuote{PersonNameRenderer} (siehe \autoref{code:settings:renderer} 
auf \autopageref{code:settings:renderer}).

Die \codeQuote{ModifyListener} des \index{Einstellungen!Renderer}\seegls{Renderer} m�ssen f�r die Unterst�tzung der
Validierung wie folgt ver�ndert werden:

\begin{java}[caption=Validierender Renderer,label=code:settings:renderer:validation]
// �nderungen in der Oberfl�che an den Wert �bertragen
forename.addModifyListener(new ModifyListener( ) {

	public void modifyText(@SuppressWarnings("unused")
	ModifyEvent e) {
		String newForename = forename.getText( );
		String newSurename = surename.getText( );

		// Wenn eingegebener Wert nicht zul�ssig
		// dann zur�cksetzen auf letzten Wert
		if (!validate(newForename)) {
			newForename = currentSelection.getDisplayPerson( )
					.getForename( );
			int pos = forename.getCaretPosition( );
			forename.setText(newForename);
			forename.setSelection(pos, pos);

		}

		PersonName person = new PersonName(newForename, newSurename);
		currentSelection.setCurrentPerson(person, true);
	}
});
surename.addModifyListener(new ModifyListener( ) {

	public void modifyText(@SuppressWarnings("unused")
	ModifyEvent e) {
		String newForename = forename.getText( );
		String newSurename = surename.getText( );

		// Wenn eingegebener Wert nicht zul�ssig
		// dann zur�cksetzen auf letzten Wert
		if (!validate(newSurename)) {
			newSurename = currentSelection.getDisplayPerson( )
					.getSurename( );
			int pos = surename.getCaretPosition( );
			surename.setText(newSurename);
			surename.setSelection(pos, pos);
		}

		PersonName person = new PersonName(newForename, newSurename);
		currentSelection.setCurrentPerson(person, true);
	}

});
\end{java}

Bei jeder Zeicheneingabe wird der neue Text mit dem Aufruf von
\codeQuote{validate()} mit dem gesetzten \index{Einstellungen!Validation}Validierer validiert. Ist der neue Text
nicht korrekt, so wird das Textfeld auf den alten Text zur�ckgesetzt.

So ist es nun m�glich zum Beispiel nur Buchstaben im Namen zu erlauben:
\index{Manager!PluginManager}
\begin{java}[caption=Validierung von Einstellungen,label=code:settings:validation]
@Override
public Settings getSettings() {
	if (settings == null) {
		settings = new Settings(PluginManager.getProperties( ),
				handler,
				"MyPlugin.settings.main",
				getRealKey( ));

		SettingsPage settingsPage = settings.addPage("page",
				"shortdescription",
				"description");

		Option option = settingsPage.addOption("person", OptionType.UNKNOWN);
		PersonNameValue value = new PersonNameValue("personkey",
				"Max",
				"Musterman");
		option.addValue(value);
		final PersonNameRenderer personNameRenderer = new PersonNameRenderer( );
		option.setRenderer(personNameRenderer);
		personNameRenderer.addValidator(new IValidator( ) {

			@Override
			public String checkString(String strg) {
				// wird nicht genutzt
				return strg;
			}

			@Override
			public boolean validate(String strg) {
				for (char chr : strg.toCharArray( )) {
					if (!Character.isLetter(chr)) {
						return false;
					}
				}
				return true;
			}

		});

	}
	return settings;
}
\end{java}

Eine Eingabe von Zahlen im Namen w�rde nun keinen Effekt mehr haben. Sie w�rden
nicht akzeptiert und den Namen nicht ver�ndern.

\subsection{Unterst�tzung}
Zur Zeit unterst�tzt von den von \xirp~bereitgestellten \seegls{Renderern} nur der f�r
den \index{Einstellungen!Optionen!Optionstyp}Optionstyp \codeQuote{TEXTFIELD} noch eine Validierung.

Dort wird bei jeder Eingabe \codeQuote{validate()} des Validators aufgerufen.
Ist die \index{Einstellungen!Validierung}Validierung erfolgreich so wird \codeQuote{checkString()} aufgerufen.
Hat sich der String dadurch ver�ndert, so wird das Textfeld auf diesen Wert
gesetzt. Schl�gt die Validierung fehl, so wird das Textfeld auf den letzten Wert
zur�ckgesetzt.

\section{Auslesen}
\index{Einstellungen!Listener}
Um �ber die �nderung von Einstellungen informiert zu werden kann man mittels
\newline\codeQuote{addSettingsChangedListener(SettingsChangedListener)} einen Listener
auf die Einstellungen registrieren welcher informiert wird wenn ge�nderte
Einstellungen gespeichert werden.

Von dem \codeQuote{Settings}-Objekt lassen sich dann mit \codeQuote{getPages()}
die einzelnen \index{Einstellungen!Seiten}Seiten der Einstellungen abrufen. Von diesen Seiten lassen sich
mit \codeQuote{getOptions()} die \index{Einstellungen!Optionen}Optionen jeder Seite abrufen. Mit
\codeQuote{getOption(String)} l�sst sich eine bestimmte Option zu einem
Schl�ssel abrufen.

Von den Optionen lassen sich mittels \codeQuote{getValues()} die \index{Einstellungen!Werte}Werte abrufen. 
Mit \codeQuote{getSelectedValues()} erh�lt man nur die selektierten Werte. F�r
Optionen die nur einen Wert haben k�nnen kann auch
\codeQuote{getSelectedValue()} benutzt werden.
Die Werte k�nnen dann mit \codeQuote{getDisplayValue()} ausgelesen werden. Bei
speziellen Werte wie denen des \index{Einstellungen!Optionen!Optionstyp}Optionstyp \codeQuote{COLOR} sollten die Werte
auf ihren speziellen Typ gecastet werden, damit direkt der Wert (z.B. RGB) 
abgerufen werden kann.

Werden nur wenige Optionen genutzt, so ist es am einfachsten sich die
Optionsobjekte direkt zu merken und dann von dort die Werte auszulesen.

Beispiel f�r einen \index{Einstellungen!Listener}Listener f�r die Einstellungen aus
\autoref{code:settings:example} auf \autopageref{code:settings:example}.

\begin{java}[caption=Einstellungen auslesen,label=code:settings:read]
settings.addSettingsChangedListener(new SettingsChangedListener( ) {

	@Override
	public void settingsChanged(SettingsChangedEvent evt) {
		StringBuilder b = new StringBuilder( );
		b.append("{");
		for (Iterator<IValue> it = checkbox.getSelectedValues( )
				.iterator( ); it.hasNext( );) {
			IValue value = it.next( );
			b.append(value.getDisplayValue( ));
			if (it.hasNext( )) {
				b.append(", ");
			}
		}
		b.append("}");
		String checked = b.toString( );
		String radio = radiobutton.getSelectedValue( )
				.getDisplayValue( );
		String combovalue = combo.getSelectedValue( )
				.getDisplayValue( );
		RGB rgb = ((RGBValue) color.getSelectedValue( )).getCurrentRGB( );
		double spinnerVal = ((SpinnerValue) spinner.getSelectedValue( )).getCurrentSpinnerValue( );
		int spinnerValInt = (int) ((SpinnerValue) spinnerInt.getSelectedValue( )).getCurrentSpinnerValue( );
		String textVal = text.getSelectedValue( ).getDisplayValue( );
		String passVal = pass.getSelectedValue( ).getDisplayValue( );

		System.out.println("Checkbox: " + checked);
		System.out.println("Radiobutton: " + radio);
		System.out.println("Combobox: " + combovalue);
		System.out.println("RGB: " + rgb);
		System.out.println("Spinner: " + spinnerVal);
		System.out.println("Spinner Int: " + spinnerValInt);
		System.out.println("Textfield: " + textVal);
		System.out.println("Password: " + passVal);
	}

});
\end{java}

Dies k�nnte zum Beispiel folgende Ausgabe erzeugen:

\begin{lstlisting}
Checkbox: {�bersetzter String: Ein Argument, 20.5}
Radiobutton: Nicht �bersetzbarer Wert
Combobox: 20.5
RGB: RGB {51, 153, 102}
Spinner: 17.0
Spinner Int: 19
Textfield: Ein Text
Password: MeinPasswort
\end{lstlisting}
\section{Internationalisierung}
\index{Einstellungen!Internationalisierung}
F�r die \index{Einstellungen!Seiten}Seiten und
\index{Einstellungen!Optionen}Optionen stehen Methoden zur Verf�gung die die �bersetzungen
um \index{Einstellungen!Internationalisierung!Argumente}Argumente erweitern:

\begin{itemize}
  \item \codeQuote{SettingsPage.setNameKeyArgs(Object...)}: Setzt Argumente f�r
  den Seitennamen.
  \item \codeQuote{SettingsPage.setLongKeyArgs(Object...)}: Setzt Argumente f�r
  die lange Beschreibung der Seite.
  \item \codeQuote{SettingsPage.setShortKeyArgs(Object...)}: Setzt Argumente f�r
  die kurze Beschreibung der Seite.
  \item \codeQuote{Option.setNameKeyArgs(Object...)}: Setzt Argumente f�r den
  Namen einer Option.
\end{itemize}

Allgemeine Informationen �ber �bersetzungen finden sich in \autoref{cha:i18n} ab
\autopageref{cha:i18n}.