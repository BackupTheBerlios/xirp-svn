\section{Validierung}
\label{sec:settings:valid}
\index{Einstellungen!Validierung}
Bei allen Darstellungsmethoden von Einstellungen ist es m�glich einen 
\codeQuote{Validator} zur Validierung von Eingaben hinzuzuf�gen. Sinn macht 
dies daher nur bei einem \seegls{Renderer} der auch tats�chlich M�glichkeiten 
zur Eingabe bereitstellt, wie z.B. der zuvor erstellte 
\newline\codeQuote{PersonNameRenderer} (siehe \autoref{code:settings:renderer} 
auf \autopageref{code:settings:renderer}).

Die \codeQuote{ModifyListener} des \index{Einstellungen!Renderer}\seegls{Renderer} m�ssen f�r die Unterst�tzung der
Validierung wie folgt ver�ndert werden:

\begin{java}[caption=Validierender Renderer,label=code:settings:renderer:validation]
// �nderungen in der Oberfl�che an den Wert �bertragen
forename.addModifyListener(new ModifyListener( ) {

	public void modifyText(@SuppressWarnings("unused")
	ModifyEvent e) {
		String newForename = forename.getText( );
		String newSurename = surename.getText( );

		// Wenn eingegebener Wert nicht zul�ssig
		// dann zur�cksetzen auf letzten Wert
		if (!validate(newForename)) {
			newForename = currentSelection.getDisplayPerson( )
					.getForename( );
			int pos = forename.getCaretPosition( );
			forename.setText(newForename);
			forename.setSelection(pos, pos);

		}

		PersonName person = new PersonName(newForename, newSurename);
		currentSelection.setCurrentPerson(person, true);
	}
});
surename.addModifyListener(new ModifyListener( ) {

	public void modifyText(@SuppressWarnings("unused")
	ModifyEvent e) {
		String newForename = forename.getText( );
		String newSurename = surename.getText( );

		// Wenn eingegebener Wert nicht zul�ssig
		// dann zur�cksetzen auf letzten Wert
		if (!validate(newSurename)) {
			newSurename = currentSelection.getDisplayPerson( )
					.getSurename( );
			int pos = surename.getCaretPosition( );
			surename.setText(newSurename);
			surename.setSelection(pos, pos);
		}

		PersonName person = new PersonName(newForename, newSurename);
		currentSelection.setCurrentPerson(person, true);
	}

});
\end{java}

Bei jeder Zeicheneingabe wird der neue Text mit dem Aufruf von
\codeQuote{validate()} mit dem gesetzten \index{Einstellungen!Validation}Validierer validiert. Ist der neue Text
nicht korrekt, so wird das Textfeld auf den alten Text zur�ckgesetzt.

So ist es nun m�glich zum Beispiel nur Buchstaben im Namen zu erlauben:
\index{Manager!PluginManager}
\begin{java}[caption=Validierung von Einstellungen,label=code:settings:validation]
@Override
public Settings getSettings() {
	if (settings == null) {
		settings = new Settings(PluginManager.getProperties( ),
				handler,
				"MyPlugin.settings.main",
				getRealKey( ));

		SettingsPage settingsPage = settings.addPage("page",
				"shortdescription",
				"description");

		Option option = settingsPage.addOption("person", OptionType.UNKNOWN);
		PersonNameValue value = new PersonNameValue("personkey",
				"Max",
				"Musterman");
		option.addValue(value);
		final PersonNameRenderer personNameRenderer = new PersonNameRenderer( );
		option.setRenderer(personNameRenderer);
		personNameRenderer.addValidator(new IValidator( ) {

			@Override
			public String checkString(String strg) {
				// wird nicht genutzt
				return strg;
			}

			@Override
			public boolean validate(String strg) {
				for (char chr : strg.toCharArray( )) {
					if (!Character.isLetter(chr)) {
						return false;
					}
				}
				return true;
			}

		});

	}
	return settings;
}
\end{java}

Eine Eingabe von Zahlen im Namen w�rde nun keinen Effekt mehr haben. Sie w�rden
nicht akzeptiert und den Namen nicht ver�ndern.

\subsection{Unterst�tzung}
Zur Zeit unterst�tzt von den von \xirp~bereitgestellten \seegls{Renderern} nur der f�r
den \index{Einstellungen!Optionen!Optionstyp}Optionstyp \codeQuote{TEXTFIELD} noch eine Validierung.

Dort wird bei jeder Eingabe \codeQuote{validate()} des Validators aufgerufen.
Ist die \index{Einstellungen!Validierung}Validierung erfolgreich so wird \codeQuote{checkString()} aufgerufen.
Hat sich der String dadurch ver�ndert, so wird das Textfeld auf diesen Wert
gesetzt. Schl�gt die Validierung fehl, so wird das Textfeld auf den letzten Wert
zur�ckgesetzt.
