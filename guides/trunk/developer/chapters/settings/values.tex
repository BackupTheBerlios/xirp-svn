\section{Eigene Werte}
\index{Einstellungen!Werte!eigene}
\label{sec:settings:ownvalues}
Die Methode \codeQuote{addValue(IValue)} bietet die M�glichkeit zu einer Option\index{Einstellungen!Optionen}
auch selbst definierte Werte hinzuzuf�gen.

Um einen eigenen \index{Einstellungen!Wert}Wert zu definieren muss daher die Klasse \codeQuote{IValue}
implementiert, oder eine der schon bestehenden Werte erweitert werden.

Als Beispiel soll ein Wert implementiert werden, welcher Vor- und Nachname einer
Person enthalten kann.

Dazu muss zun�chst die Klasse \codeQuote{IValue} implementiert werden, welche
zus�tzlich \codeQuote{Observeable} erweitern muss, damit sp�ter eine
Synchronisation mit der anzeigenden Oberfl�che m�glich wird. Weiterhin erh�lt
die Klasse noch eine Klasse \codeQuote{PersonName} welche Name und Vorname
kapselt. 

\begin{java}[caption=Klasse PersonName]
public class PersonName {

	private String forename;
	private String surename;

	public PersonName(String forename, String surename) {
		this.forename = forename;
		this.surename = surename;
	}

	public String getForename() {
		return forename;
	}

	public String getSurename() {
		return surename;
	}

	public static PersonName parse(String string) {
		final String[] split = string.split(" ");
		String forename = "";
		String surename = "";
		if (split.length > 0) {
			forename = split[0];
			if (split.length > 1) {
				surename = split[1];
			}
		}

		return new PersonName(forename, surename);
	}

	@Override
	public String toString() {
		return forename + " " + surename;
	}
}
\end{java}

Die Klasse \codeQuote{PersonName} enth�lt bereits eine Methode welche einen mit
\codeQuote{toString()} generierten String wieder zur�ck in ein Objekt parsen
kann. Dies wird ben�tigt um die gespeicherten Einstellungen wieder auszulesen.

\begin{java}[caption=Eigener Wert f�r Einstellungen: PersonNameValue, label=code:settings:ownvalue]
import java.util.ArrayList;
import java.util.Observable;
import java.util.Observer;

import de.unibremen.rr.xirp.settings.IValue;
import de.unibremen.rr.xirp.settings.SettingsChangedEvent;
import de.unibremen.rr.xirp.settings.SettingsChangedListener;

public class PersonNameValue extends Observable implements IValue {

	private PersonName defaultValue;
	private PersonName currentValue;
	private PersonName savedValue;
	private String saveKey;
	private ArrayList<SettingsChangedListener> listeners = new ArrayList<SettingsChangedListener>( );

	public PersonNameValue(String key, String forename, String surename) {
		this.saveKey = key;
		this.defaultValue = new PersonName(forename, surename);
		this.currentValue = defaultValue;
		this.savedValue = currentValue;
	}

	@Override
	public void addObserverToValue(Observer obs) {
		this.addObserver(obs);
	}
	
	...
} 
\end{java}

Da die Klasse \codeQuote{Observable} erweitert kann der \codeQuote{Observer} der
Methode \codeQuote{addObserverToValue()} direkt zur Klasse hinzugef�gt werden.
Diese Methode wird von den \index{Einstellungen!Renderer}\seegls{Renderern} aufgerufen um sich auf �nderungen des
Objektes zu registrieren.

Nun folgt die Implementierung aller weiteren Methoden die vom Interface
bereitgestellt werden:
\begin{java}[caption=Eigener Wert f�r Einstellungen: PersonNameValue(2), label=code:settings:ownvalue2] 
@Override public String getDisplayValue() {
	return currentValue.toString( );
}

public PersonName getDisplayPerson() {
	return currentValue;
}

public void setCurrentPerson(PersonName person) {
	setCurrentPerson(person, false);
}

public void setCurrentPerson(PersonName person, boolean fromUI) {
	this.currentValue = person;

	// notify the observers that the current state has changed
	// and they should update the UI
	if (!fromUI) {
		setChanged( );
		notifyObservers( );
	}

	notifyChange( );
}

@Override
public String getSaveKey() {
	return saveKey;
}

@Override
public String getSaveValue() {
	return savedValue.toString( );
}

@Override
public void parseSavedValue(Object object) {
	setSavedValue(PersonName.parse(object.toString( )));
}

public void setSavedValue(PersonName person) {
	this.savedValue = person;
	setCurrentPerson(savedValue);
}

@Override
public boolean hasChanged() {
	return !currentValue.equals(savedValue);
}

@Override
public boolean isDefaultSelected() {
	return currentValue.equals(defaultValue);
}

@Override
public void reset() {
	setCurrentPerson(savedValue);
}

@Override
public void save() {
	this.savedValue = currentValue;
}

@Override
public void setToDefault() {
	setCurrentPerson(defaultValue);
}

@Override
public void setSelected(boolean selection) {
	setSelected(selection, false);
}

public void addLocaleChangeListener(SettingsChangedListener listener) {
	listeners.add(listener);
}

private void notifyChange() {
	for (SettingsChangedListener listener : listeners) {
		listener.settingsChanged(new SettingsChangedEvent(this));
	}
}

public void removeLocaleChangeListener(SettingsChangedListener listener) {
	listeners.remove(listener);
}

//Die weiteren Methoden treffen hier nicht zu

@Override
public void setSelected(boolean selection, boolean fromUI) {
	// Nothing to do
}

@Override
public boolean isSelected() {
	return true;
}

@Override
public boolean isCurrentlySelected() {
	return true;
}

@Override
public String getKey() {
	return null;
}

@Override
public Object[] getKeyArgs() {
	return null;
}
\end{java}

Die Methoden \codeQuote{parseSavedValue()} wird von der Optionsklasse aufgerufen
und �bergibt den \index{Einstellungen!Werte}Wert welcher einmal durch \codeQuote{getSaveValue()}
gespeichert wurde.

Die Methode \codeQuote{setCurrentPerson()} kommt zweimal vor. Dies liegt daran
dass ein Aufruf sowohl von der Oberfl�che als auch von innerhalb erfolgen kann
und dies unterschieden werden muss. Erfolgt der Aufruf nicht von der Oberfl�che
wird diese Beispielsweise �ber \codeQuote{notifyObservers( )} �ber die �nderung
informiert, da sie sich vorher mit \codeQuote{addOberserverToValue( )}
angemeldet hat.

Der Wert kann nun benutzt werden und wird im folgenden Beispiel einer
Checkbox-Option hinzugef�gt.
\index{Manager!PluginManager}
\begin{java}[caption=Benutzung von PersonNameValue]
@Override
public Settings getSettings() {
	if (settings == null) {
		settings = new Settings(PluginManager.getProperties( ),
				handler,
				"Example.settings.main",
				getRealKey( ));

		SettingsPage settingsPage = settings.addPage("page",
				"shortdescription",
				"description");

		Option check = settingsPage.addOption("personcheck",
				OptionType.CHECKBOX);
		PersonNameValue value = new PersonNameValue("personkey",
				"Max",
				"Musterman");
		check.addValue(value);
		check.addNumberValue(20.5, SettingsState.NOT_SELECTED);

	}
	return settings;
}
\end{java}