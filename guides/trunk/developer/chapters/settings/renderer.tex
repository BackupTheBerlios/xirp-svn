\section{Eigener Renderer}
\index{Einstellungen!Renderer!eigener}
Um die \index{Einstellungen!Werte}Werte von \codeQuote{PersonNameValue} (siehe
\autoref{code:settings:ownvalue} auf \autopageref{code:settings:ownvalue}) so 
anzuzeigen,
dass der Name auch ge�ndert werden kann ist ein eigener Renderer n�tig.

Eigene \seegls{Renderer} m�ssen \codeQuote{IOptionRenderer} implementieren. Diese bietet
auch die M�glichkeit eingegebenen Strings auf \index{Einstellungen!Validierung}Korrektheit (siehe
\autoref{sec:settings:valid} auf \autopageref{sec:settings:valid}) zu pr�fen. In der
abstrakten Implementierung \codeQuote{AbstractOptionRenderer} ist die Behandlung
eines Validierers schon implementiert. Daher wird hier auf die Erweiterung von
\codeQuote{AbstractOptionRenderer} zur�ckgegriffen:

\begin{java}[caption=Eigener Renderer f�r Einstellungen,label=code:settings:renderer]
import java.util.Observable;

import org.apache.log4j.Logger;
import org.eclipse.swt.SWT;
import org.eclipse.swt.events.ModifyEvent;
import org.eclipse.swt.events.ModifyListener;
import org.eclipse.swt.layout.GridLayout;
import org.eclipse.swt.widgets.Composite;

import de.unibremen.rr.xirp.settings.IValue;
import de.unibremen.rr.xirp.settings.Option;
import de.unibremen.rr.xirp.ui.util.SWTUtil;
import de.unibremen.rr.xirp.ui.widgets.custom.XComposite;
import de.unibremen.rr.xirp.ui.widgets.custom.XLabel;
import de.unibremen.rr.xirp.ui.widgets.custom.XText;
import de.unibremen.rr.xirp.ui.widgets.dialogs.preferences.renderer.AbstractOptionRenderer;
import de.unibremen.rr.xirp.util.Constants;

public class PersonNameRenderer extends AbstractOptionRenderer {

	private static Logger logClass = Logger.getLogger(PersonNameRenderer.class);

	private XText forename;
	private XText surename;

	public void render(Composite parent, Option option) {
		// parent ist nur zweispaltig, daher dreispaltiges Composite
		// selbst erstellen
		XComposite optionComp = new XComposite(parent, SWT.NONE);
		final GridLayout gridLayout = SWTUtil.setGridLayout(optionComp, 3, true);
		SWTUtil.resetMargins(gridLayout);
		SWTUtil.setGridData(optionComp, true, false, SWT.FILL, SWT.TOP, 2, 1);

		XLabel l = new XLabel(optionComp, SWT.NONE, option.getI18n( ));
		l.setTextForLocaleKey(option.getNameKey( ), option.getNameKeyArgs( ));
		SWTUtil.setGridData(l, true, false, SWT.LEFT, SWT.TOP, 1, 1);

		// Nur ein Wert kann dargestellt werden
		PersonNameValue last = null;
		for (IValue value : option.getValues( )) {
			if (value instanceof PersonNameValue) {
				last = (PersonNameValue) value;
			}
		}

		// Textfelder f�r Vor- und Nachname
		forename = new XText(optionComp, SWT.BORDER, true);
		SWTUtil.setGridData(forename, true, false, SWT.FILL, SWT.TOP, 1, 1);
		surename = new XText(optionComp, SWT.BORDER, true);
		SWTUtil.setGridData(surename, true, false, SWT.FILL, SWT.TOP, 1, 1);

		if (last != null) {
			final PersonNameValue currentSelection = last;
			// Observer auf den Wert hinzuf�gen
			currentSelection.addObserverToValue(this);

			// Wert darstellen
			final PersonName displayPerson = currentSelection.getDisplayPerson( );
			forename.setText(displayPerson.getForename( ));
			surename.setText(displayPerson.getSurename( ));

			// �nderungen in der Oberfl�che an den Wert �bertragen
			final ModifyListener modifyListener = new ModifyListener( ) {

				public void modifyText(@SuppressWarnings("unused")
				ModifyEvent e) {
					PersonName person = new PersonName(forename.getText( ),
							surename.getText( ));
					currentSelection.setCurrentPerson(person, true);
				}

			};
			forename.addModifyListener(modifyListener);
			surename.addModifyListener(modifyListener);

		}
		else {
			forename.setEnabled(false);
			surename.setEnabled(false);
			logClass.warn("Der PersonNameRenderer kann nur Werte vom Typ PersonNameValue darstellen." +
					Constants.LINE_SEPARATOR);
		}
	}

	public void update(Observable observable, @SuppressWarnings("unused")
	Object obj) {
		// �nderungen des Wertes in der Oberfl�che darstellen
		// z.B. bei Reset
		if (observable instanceof PersonNameValue) {
			PersonNameValue val = (PersonNameValue) observable;
			final PersonName displayPerson = val.getDisplayPerson( );
			this.forename.setText(displayPerson.getForename( ));
			this.surename.setText(displayPerson.getSurename( ));
		}
	}
}
\end{java}

\kfig{settings_renderer}{1}{Ein eigener Renderer f�r Einstellungen}{img:settings:renderer}

Es ist nur m�glich einen \index{Einstellungen!Werte}Wert f�r diese
\index{Einstellungen!Optionen}Option darzustellen. Dabei wird der Name 
der Person in zwei Textfeldern zur Eingabe bereitgestellt.

Nutzt man den \seegls{Renderer} f�r die Darstellung erh�lt man die Ansicht aus
\autoref{img:settings:renderer}: 
\index{Manager!PluginManager}
\begin{java}[caption=Den eigenen Renderer zur Darstellung nutzen,label=code:renderer:use]
@Override
public Settings getSettings() {
	if (settings == null) {
		settings = new Settings(PluginManager.getProperties( ),
				handler,
				"Example.settings.main",
				getRealKey( ));

		SettingsPage settingsPage = settings.addPage("page",
				"shortdescription",
				"description");

		Option option = settingsPage.addOption("person", OptionType.UNKNOWN);
		PersonNameValue value = new PersonNameValue("personkey",
				"Max",
				"Musterman");
		option.addValue(value);
		option.setRenderer(new PersonNameRenderer( ));
	}
	return settings;
}
\end{java}
