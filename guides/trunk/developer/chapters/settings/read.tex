\section{Auslesen}
\index{Einstellungen!Listener}
Um �ber die �nderung von Einstellungen informiert zu werden kann man mittels
\newline\codeQuote{addSettingsChangedListener(SettingsChangedListener)} einen Listener
auf die Einstellungen registrieren welcher informiert wird wenn ge�nderte
Einstellungen gespeichert werden.

Von dem \codeQuote{Settings}-Objekt lassen sich dann mit \codeQuote{getPages()}
die einzelnen \index{Einstellungen!Seiten}Seiten der Einstellungen abrufen. Von diesen Seiten lassen sich
mit \codeQuote{getOptions()} die \index{Einstellungen!Optionen}Optionen jeder Seite abrufen. Mit
\codeQuote{getOption(String)} l�sst sich eine bestimmte Option zu einem
Schl�ssel abrufen.

Von den Optionen lassen sich mittels \codeQuote{getValues()} die \index{Einstellungen!Werte}Werte abrufen. 
Mit \codeQuote{getSelectedValues()} erh�lt man nur die selektierten Werte. F�r
Optionen die nur einen Wert haben k�nnen kann auch
\codeQuote{getSelectedValue()} benutzt werden.
Die Werte k�nnen dann mit \codeQuote{getDisplayValue()} ausgelesen werden. Bei
speziellen Werte wie denen des \index{Einstellungen!Optionen!Optionstyp}Optionstyp \codeQuote{COLOR} sollten die Werte
auf ihren speziellen Typ gecastet werden, damit direkt der Wert (z.B. RGB) 
abgerufen werden kann.

Werden nur wenige Optionen genutzt, so ist es am einfachsten sich die
Optionsobjekte direkt zu merken und dann von dort die Werte auszulesen.

Beispiel f�r einen \index{Einstellungen!Listener}Listener f�r die Einstellungen aus
\autoref{code:settings:example} auf \autopageref{code:settings:example}.

\begin{java}[caption=Einstellungen auslesen,label=code:settings:read]
settings.addSettingsChangedListener(new SettingsChangedListener( ) {

	@Override
	public void settingsChanged(SettingsChangedEvent evt) {
		StringBuilder b = new StringBuilder( );
		b.append("{");
		for (Iterator<IValue> it = checkbox.getSelectedValues( )
				.iterator( ); it.hasNext( );) {
			IValue value = it.next( );
			b.append(value.getDisplayValue( ));
			if (it.hasNext( )) {
				b.append(", ");
			}
		}
		b.append("}");
		String checked = b.toString( );
		String radio = radiobutton.getSelectedValue( )
				.getDisplayValue( );
		String combovalue = combo.getSelectedValue( )
				.getDisplayValue( );
		RGB rgb = ((RGBValue) color.getSelectedValue( )).getCurrentRGB( );
		double spinnerVal = ((SpinnerValue) spinner.getSelectedValue( )).getCurrentSpinnerValue( );
		int spinnerValInt = (int) ((SpinnerValue) spinnerInt.getSelectedValue( )).getCurrentSpinnerValue( );
		String textVal = text.getSelectedValue( ).getDisplayValue( );
		String passVal = pass.getSelectedValue( ).getDisplayValue( );

		System.out.println("Checkbox: " + checked);
		System.out.println("Radiobutton: " + radio);
		System.out.println("Combobox: " + combovalue);
		System.out.println("RGB: " + rgb);
		System.out.println("Spinner: " + spinnerVal);
		System.out.println("Spinner Int: " + spinnerValInt);
		System.out.println("Textfield: " + textVal);
		System.out.println("Password: " + passVal);
	}

});
\end{java}

Dies k�nnte zum Beispiel folgende Ausgabe erzeugen:

\begin{lstlisting}
Checkbox: {�bersetzter String: Ein Argument, 20.5}
Radiobutton: Nicht �bersetzbarer Wert
Combobox: 20.5
RGB: RGB {51, 153, 102}
Spinner: 17.0
Spinner Int: 19
Textfield: Ein Text
Password: MeinPasswort
\end{lstlisting}