\chapter{Sprachausgabe}
\index{Sprachausgabe|(}
\label{ref:speech}
Das folgende Kapitel besch�ftigt sich mit der Sprachausgabe die in 
\xirp~integriert wurde. Es werden die folgenden Themen genauer betrachtet.

\begin{itemize}
  \item Voraussetzungen
  \item Konfiguration
  \item Standard-Stimmen
  \item MBROLA-Stimmen
\end{itemize}

\newpage

\section{Voraussetzungen}
Es gibt zwei Voraussetzungen, um die Sprachausgabe nutzen zu k�nnen.

\begin{itemize}
  \item Soundkarte
  \item Plugin welches die Sprachausgabe nutzt
\end{itemize}

\index{Soundkarte}
\index{Plugins}
\index{Einstellungen}
Nat�rlich muss, um die Sprachausgabe h�ren zu k�nnen eine Soundkarte im System 
installiert sein. Des Weiteren muss ein Plugin geladen werden, welches Texte 
mittels des Sprachausgabesystems ausgibt. Momentan gibt es nur eine Stelle, an 
der die Anwendung selbst die Sprachausgabe benutzt: Im \enquote{�ber 
\xirp~2}-Dialog. Dort gibt es die M�glichkeit sich einen Informationstext �ber 
die Anwendung vorlesen zu lassen. Hierzu muss die Sprachausgabe in den 
Einstellungen aktiviert sein. Der Text der ausgegeben werden kann ist momentan 
nur auf Englisch verf�gbar\footnote{In Version 3.0.0 ist geplant, dass auch 
dieser Text �bersetzt wird.}.

\section{Konfiguration}
\index{Sprachausgabe!Konfiguration}
\index{Einstellungen}
In den Einstellungen der Anwendung kann unter der Kategorie 
\enquote{Speech-Einstellungen} die Sprachausgabe konfiguriert werden. Dort kann 
die zu verwendende Sprache gew�hlt und die Sprachausgabe de- und aktiviert 
werden. Welche Sprachen dort zur Auswahl stehen h�ngt von der momentan 
in\xirp~eingestellten Sprache und den insgesamt installierten Sprachen ab.

\section{Standard-Stimmen}
\index{Stimmen}
\xirp~liefert zwei Standard-Stimmen:

\index{Stimmen!kevin}
\index{Stimmen!kevin16}
\begin{itemize}
  \item kevin
  \item kevin16
\end{itemize}

Beide Stimmen sind m�nnliche Stimmen. Die Ausgabequalit�t ist bei der Stimme 
\enquote{kevin16} besser als die der Stimme \enquote{kevin}. Die 
Voreingestellte Stimme ist \enquote{kevin}.

\section{MBROLA-Stimmen}
\index{MBROLA}
Mit Hilfe der Software und der vorhandenen Sprachdateien des 
\href{http://tcts.fpms.ac.be/synthesis/mbrola.html}{MBROLA-Projektes}\footnote{Projekthomepage 
von MBROLA:
\href{http://tcts.fpms.ac.be/synthesis/mbrola.html}{http://tcts.fpms.ac.be/synthesis/mbrola.html}}
ist es m�glich zus�tzliche Stimmen zu installieren.

\subsection{Was ist MBROLA?}
Das Ziel des Projektes wird von den Initiatoren (TCTS Lab of the Facult� 
Polytechnique de Mons (Belgien)) wie folgt beschrieben.

\begin{quote}
The aim of the MBROLA project [\ldots] is to obtain a set of speech synthesizers
for as many languages as possible, and provide them free for non-commercial 
applications. \cite{MBROLA}
\end{quote}

\subsection{Installation der Stimmen}
Um zus�tzliche Stimmen zu installieren muss zun�chst im Hauptordner von
\xirp~ein Ordner \texttt{voice} angelegt werden. In diesen Ordner m�ssen dann
die MBROLA-Executable (Ohne Dateiendung: \texttt{mbrola}) und die Stimmen-Ordner
kopiert werden.
\par
Die MBROLA-Executable kann von
\begin{quote}
\href{http://tcts.fpms.ac.be/synthesis/mbrola/bin/pcdos/mbr301d.zip}
{http://tcts.fpms.ac.be/synthesis/mbrola/bin/pcdos/mbr301d.zip} 
\end{quote}
heruntergeladen werden.
Dies ist die DOS Version von MBROLA. Diese sollte unter Windows Systemen
benutzt werden. Linux Executables sind auf der Downloadseite\footnote{\href
{http://tcts.fpms.ac.be/synthesis/mbrola/mbrcopybin.html}
{http://tcts.fpms.ac.be/synthesis/mbrola/mbrcopybin.html}} ebenfalls
vorhanden. Die \texttt{mbrola.exe} die in der ZIP-Datei vorhanden ist muss in
\texttt{mbrola} umbenannt werden.
\par
Zus�tzliche Stimmen k�nnen von 
\href{http://tcts.fpms.ac.be/synthesis/mbrola/mbrcopybin.html}
{http://tcts.fpms.ac.be/synthesis/mbrola/mbrcopybin.html} heruntergeladen werden.
\par
\index{Einstellungen}
Wurde alles korrekt benannt und erstellt, sind die neuen Stimmen im
Einstellungsdialog vorhanden und ausw�hlbar.
\index{Sprachausgabe|)}
