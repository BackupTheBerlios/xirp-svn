%
% header.tex
%
\documentclass[%
	pdftex,%              PDFTex verwenden
	a4paper,%             A4 Papier
	twoside,%             Einseitig
	bibtotoc,%    Literaturverzeichnis nummeriert einf�gen
	idxtotoc,%            Index ins Verzeichnis einf�gen
	halfparskip,%         Europ�ischer Satz mit abstand zwischen Abs�tzen
	chapterprefix,%       Kapitel anschreiben als Kapitel
	headsepline,%         Linie nach Kopfzeile
	%footsepline,%         Linie vor Fusszeile
	12pt,%                 Gr�ssere Schrift, besser lesbar am bildschrim
	openright,
	pointlessnumbers
]{scrbook}

\usepackage{scrpage2}

\usepackage{textcomp}

\newcommand{\xirp}{$\chi$irp\index{xirp@$\chi$irp}}
\newcommand{\xirpname}{\enquote{eXtendable interface for robotic purposes}}
\newcommand{\version}{2.4.0}
\newcommand{\devguideurl}{http://developer.berlios.de/docman/display_doc.php?docid=1711&group_id=8442}
\newcommand{\devguideurlname}{xirp.berlios.de}
\newcommand{\seegls}[1]{\textuparrow{#1}}

\usepackage{scrpage2}

% \renewcommand{\partformat}{\partname~\thepart}
% \renewcommand{\chapterformat}{\chapapp~\thechapter}
% \renewcommand{\othersectionlevelsformat}[1]{\csname the#1\endcsname\enskip}
% 
% \renewcommand{\partmarkformat}{\thepart\enskip}
% \renewcommand{\chaptermarkformat}{\chapapp~\thechapter\enskip}
% \renewcommand{\sectionmarkformat}{\thesection\enskip}
% \renewcommand{\subsectionmarkformat}{\thesubsection\enskip}
% \renewcommand{\subsubsectionmarkformat}{\thesubsubsection\enskip}
% \renewcommand{\paragraphmarkformat}{\theparagraph\enskip}
% \renewcommand{\subparagraphmarkformat}{\thesubparagraphmarkformat\enskip}

\renewcommand{\chaptermark}[1]{\markboth{\chaptermarkformat{#1}}{\chaptermarkformat{#1}}}

\renewcommand{\chapterpagestyle}{scrheadings}
\renewcommand{\indexpagestyle}{scrheadings}

% kopf und fusszeilen
\pagestyle{scrheadings}

\clearscrheadfoot
\rohead[\headmark~\pagemark]{\headmark~\pagemark}
\lehead[\pagemark~\headmark]{\pagemark~\headmark}
\setheadtopline{.4pt}
\setheadsepline{.4pt}

\setcounter{secnumdepth}{5}
\setcounter{tocdepth}{5}

% Paket f�r die Indexerstellung.
\usepackage{makeidx}
%Formattierungen f�r Indexfile
%Erste Ebene
\newcommand*{\indexdelim}{\ \hspace{0pt plus 1fil}\penalty0\null\nobreak
  \dotfill~}
%Zweite Ebene
\newcommand*{\indexdelimi}{~\dotfill\penalty0\ }
%Dritte Ebene
\newcommand*{\indexdelimii}{~\dotfill\penalty0\ }

%Abschnitts�berschrift
\newcommand*{\indexsection}[1]{%
  \ifx\empty#1\empty\else
  \hspace{0pt plus 2fil}{{\usekomafont{sectioning} #1}}\hspace{0pt plus
    1fil}\nopagebreak
  \fi
}
% Paket f�r �bersetzungen ins Deutsche
\usepackage[french,ngerman]{babel}

% Pakete um Latin1 Zeichnens�tze verwenden zu k�nnen und die dazu
% passenden Schriften.
\usepackage[latin1]{inputenc}
\usepackage[T1]{fontenc}

% Paket f�r Quotes
\usepackage[babel,french=guillemets,german=swiss]{csquotes}

% Paket um die Symbole des TS1 Zeichensatzes verwenden zu k�nnen.
%\usepackage{textcomp}

% Paket zum Erweitern der Tabelleneigenschaften
\usepackage{array}

% Paket um Grafiken einbetten zu k�nnen
\usepackage[pdftex]{graphicx}

% Spezielle Schrift verwenden.
%\usepackage{goudysans}

% Spezielle Schrift im Koma-Script setzen.
%\setkomafont{sectioning}{\normalfont\bfseries}
%\setkomafont{captionlabel}{\normalfont\bfseries}
%\setkomafont{pagehead}{\normalfont\itshape}
%\setkomafont{descriptionlabel}{\normalfont\bfseries}

% Zeilenumbruch bei Bildbeschreibungen.
\setcapindent{1em}

% mathematische symbole aus dem AMS Paket.
%\usepackage{amsmath}
%\usepackage{amssymb}

% Type 1 Fonts f�r bessere darstellung in PDF verwenden.
%\usepackage{mathptmx}           % Times + passende Mathefonts
%\usepackage[scaled=.92]{helvet} % skalierte Helvetica als \sfdefault
%\usepackage{courier}            % Courier als \ttdefault

% Paket um Textteile drehen zu k�nnen
\usepackage{rotating}

% Package f�r Farben im PDF
\usepackage{color}
\input{rgb}
% Links innerhalb des PDF Dokuments
\definecolor{LinkColor}{rgb}{0,0,0.5}
\usepackage[unicode,
	pdftitle={Xirp - User Guide},
	pdfauthor={Matthias Gernand},
	pdfcreator={MiKTeX, LaTeX with hyperref and KOMA-Script},
	pdfsubject={Das Xirp Benutzerhandbuch},
	pdfkeywords={Xirp, Robotik, Universit�t, Bremen}]{hyperref}
\hypersetup{colorlinks=true,
	linkcolor=LinkColor,
	citecolor=LinkColor,
	filecolor=LinkColor,
	menucolor=LinkColor,
	%pagecolor=LinkColor,
	urlcolor=LinkColor}
	
% Paket um Listings sauber zu formatieren.
\usepackage{listings}
\lstloadlanguages{XML}

% Listing Definitionen
\lstset{language=XML,
	numbers=none,
	stepnumber=1,
	numbersep=5pt,
	numberstyle=\tiny,
	breaklines=true,
	breakautoindent=true,
	postbreak=\space,
	tabsize=2,
	basicstyle=\ttfamily\footnotesize,
	usekeywordsintag=true,
	markfirstintag=true,
	tagstyle=\color{SeaGreen},
	keywordstyle=\color{DarkBlue},
	commentstyle=\color{DarkRed},
	stringstyle=\color{BlueViolet},
	showspaces=false,
	showstringspaces=false,
	extendedchars=true,
	backgroundcolor=\color{LightGrey}}

\newenvironment{ListChanges}%
	{\begin{list}{$\diamondsuit$}{}}%
	{\end{list}}

\usepackage[number=none,style=altlist,toc=true]{glossary}
% Glossar erzeugen
\makeglossary
% Index erzeucgen
\makeindex

\addto\extrasngerman{%
\def\partautorefname{Teil}
\def\pageautorefname{\pagename}%
\def\figureautorefname{\figurename}%
\def\tableautorefname{\tablename}%
\def\appendixautorefname{\appendixname}%
\def\chapterautorefname{\chaptername}%
\def\sectionautorefname{Abschnitt}%
\def\subsectionautorefname{Abschnitt}%
\def\subsubsectionautorefname{Abschnitt}%
\def\paragraphautorefname{Abschnitt}%
\def\subparagraphautorefname{Abschnitt}%
\def\equationautorefname{Formel}%
\def\appendixautorefname{\appendixname}
\def\Itemautorefname{\itemname}
\def\Hfootnoteautorefname{\footnotename}
\def\AMSautorefname{\AMSname}
\def\theoremautorefname{\theoremname}
}

\renewcommand{\lstlistingname}{Auflistung}
\renewcommand{\lstlistlistingname}{Auflistungsverzeichnis}
\renewcommand{\glossaryname}{Glossar}

\setkomafont{caption}{\sffamily}
\setkomafont{captionlabel}{\sffamily\bfseries}
\setkomafont{footnote}{\sffamily}
\setkomafont{pagehead}{\sffamily}
\setkomafont{pagenumber}{\sffamily\bfseries}

\usepackage{multicol}
\renewcommand{\glossarypreamble}{\begin{multicols}{2}}
\renewcommand{\glossarypostamble}{\end{multicols}}

%
% EOF
%
