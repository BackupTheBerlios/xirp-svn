\glossary{name={Testerbot},description={Ein einfacher Server, welcher
Zufallswerte f�r diverse Sensoren liefert.},}

\glossary{name={Profile},description={Ein Profil wird in XML definiert. Es
spezifiziert welche(r) Roboter geladen werden soll und einige weitere
Optionen.},} 

\glossary{name={CSV},description={Comma Separated Values: Ein Datenformat um
tabellarische Daten in reinem text ablegen zu k�nnen.},}

\glossary{name={PDF},description={Portable Document Format: Ein Dokumentformat
zur systemunabh�ngigen Anzeige von Dokumenten.},}

\glossary{name={Client},description={Ein Computersystem, welches eine Verbindung
mit einem Server aufnimmt und Nachrichten mit diesem austauscht.},}

\glossary{name={Server},description={Ein Computersystem, welches Berechnungen
und Daten f�r seine Clients bereitstellt.},}

\glossary{name={Laserscanner},description={Eigentlich ein
Laser-Entfernungsmesser.},}

\glossary{name={Thermopile},description={Ein Sensor, der W�rmewerte liefert.},}

\glossary{name={HTML},description={Hyper Text Markup Language: Eine Sprache zum
Erstellen von Webseiten.},}

\glossary{name={API},description={Application Programming Interface: Eine
Programmierschnittstelle.},}

\glossary{name={Sensor},description={Bauteil eines Roboters zum Erfassen seiner
Umgebung.},}

\glossary{name={Aktuator},description={Bauteil eines Roboters zum Manipulieren
seiner Umgebung.},}

\glossary{name={XML},description={Extendable Markup Language: Eine
Auszeichnungssprache zur Darstellung hierarchisch strukturierter Daten in Form
von Textdateien},}

\glossary{name={Text-To-Speech},description={Eine Methode, mit der man W�rter
und Texte �ber die Lautsprecher ausgeben kann.},}

\glossary{name={Shortcut},description={Eine Abk�rzung zum Aufrufen von Teilen
eines Programms, z.B. CRTL+Q zum Beenden eines Programms.},}

\glossary{name={Gamepad},description={Ein Ger�t, welches oft zum Steuern vom
Computerspielen benutzt wird.},}

\glossary{name={I18n},description={Kurzform f�r Internationalisierung.},}

\glossary{name={Drag-and-Drop},description={Eine Methode zum Bewegen grafischer
Elemente mit Hilfe einer Maus.},}

\glossary{name={Icon},description={Ein kleines Bild.},}

\glossary{name={Hotkey},description={Ein Tastaturk�rzel in einer Anwendung.},}

\glossary{name={TCP/IP},description={Transmission Control Protocol/Internet
Protocol: Das TCP/IP-Referenzmodell beschreibt den
Aufbau und das Zusammenwirken von Netzwerkprotokollen der
Internet-Protokolle.},}

\glossary{name={IP-Adresse},description={Eine Internet Protkoll-Adresse, z.B.
192.168.1.1},}

\glossary{name={IPv4},description={Das Internet Protokoll in der Version 4.},}

\glossary{name={RS232},description={Eine serielle Schnittstelle},}

\glossary{name={SMTP},description={Simple Mail Transfer Protocol, Ein einfaches
Mailtransportprotokoll.},}

\glossary{name={MBROLA},description={Ein Programm zur Sprachausgabe.},}

\glossary{name={PNG},description={Portable Network Graphics, Bildstandard.},}

\glossary{name={JPG},description={auch JPEG genannt, Bildstandard nach der
Joint Photographic Experts Group.},}

\glossary{name={E-Mail},description={Ein elektronischer Brief},}

\glossary{name={Cc},description={Carbon Copy, hier eine Kopie einer E-Mail.},}

\glossary{name={Zoom},description={Vergr��ern oder Verkleinen von einer Grafik},}

\glossary{name={Port},description={Adresskomponente einer IP-Adresse. Ist n�tig
um Daten einer Anwendung zuordnen zu k�nnen.},}

\glossary{name={Carmen},description={Carnegie Mellon Robot Navigation Toolkit:
Sammlung von Software fur die Kontrolle mobiler Roboter, \textit{siehe}
\href{carmen.sourceforge.net}{carmen.sourceforge.net}},}

\glossary{name={IPC},description={Interprozesskommunikation, wird von Carmen zur
Kommunikation benutzt.},}

\glossary{name={Tray},description={System Tray od. Taskbar Notification Area,
ist ein Bereich auf der grafischen Benutzeroberfl�che vieler Desktop
Environments zur Anzeige von Nachrichten.},}

\glossary{name={Open Source},description={Open Source ist Software, die unter
einer von der Open Source Initiative (OSI) anerkannten Lizenz steht.},}

\glossary{name={CPL},description={Die Common Public License (CPL) ist eine
Open-Source-Lizenz.},}
